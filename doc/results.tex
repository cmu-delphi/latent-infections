\section{Results}
\label{sec:results}

\subsection{Infection estimates and cases-to-infections ratios across the \US states}
\label{sec:omitted-waves}

Prior to Omicron, the largest infection outbreaks were observed in the late 
summer and early fall of 2021 in Louisiana, Georgia, Idaho, and Montana (\Crefrange{fig:state_infect_est}{fig:six-states}). 
During this time, the states that have the
highest rate of infections on single day are Louisiana (476 infections per 100K,
on July 20, 2021) and Idaho (also 457 infections per 100K, on September 7,
2021). The period of lowest viral transmission was in the summer of
2020, where Vermont had fewer than 10 infections
per 100K per week from June to August, the longest such lull observed for any state. 

\begin{figure}[!tb]
\centering
\includegraphics[width=.99\linewidth]{adj-unadj-cases-plot-1.pdf} 
\caption{Estimates of the daily new infections per 100,000 population
for each \US state from June 1, 2020 to November 29, 2021 (dark blue line). The
blue shaded regions depict the 50, 80, and 95\% intervals for the estimates,
while the orange line represents the trailing 7-day average of reported cases
per 100,000.}
\label{fig:state_infect_est}
\end{figure}    

\begin{figure}[!tb]
\centering
    \includegraphics*[width=\linewidth]{six-decon-var-1.pdf}
    \caption{Panel A: Reported cases (orange) and estimates of daily new
    infections (dark blue) per 100K inhabitants. The blue shaded regions
    indicate 50, 80, and 95\% confidence bands.  
    Panel B: Deconvolved cases colored by variant per 100K inhabitants.}
    \label{fig:six-states}
\end{figure}

Nearly all states exhibit two major waves in infections--the Ancestral wave
began in the fall of 2020 and extended into the winter season, while the Delta wave
started in the late summer of 2021 and continued into mid-fall. 
In general, greater similarities in the strength and magnitude of outbreaks
emerge in small clusters of states that border each other. 
For instance, in the
Western states of Idaho and Montana or in the Southern states of North and South
Carolina present waves of infections that mirror each other in amplitude and timing.

While the major Ancestral, Alpha, and Delta waves tend to be visible for most
states, there are clear outbreaks in unreported infections that are not easily
detectable from cases alone in the falls of 2020 and 2021. For example, a wave
of infections is evident in North Dakota and South Dakota over the spring of
2021 that is virtually undetectable from the reported cases. Similarly, in the late summer
of 2021, the Delta wave is only faintly detectable from cases in a number of
Northeastern states, while infections suggest that it has already begun in earnest.  

Moreover, cases tend to 
severely underestimate infections during Delta for many states, more so
than in earlier waves (\Cref{fig:state_infect_est}). 
The most extreme was New Jersey, where about 6.3\% of
estimated infections were eventually reported as cases. Similarly low are
Maryland (7.3\%), and Nevada (8.4\%), and South Dakota (10.0\%). This issue
extends to most states: in 44 states fewer than 30\% of infections eventually
appear in case reports. The cases-to-infections ratio was larger in earlier waves, and
its effects were most apparent in different regions. During Alpha, Louisiana had the
lowest ratio of infections to cases (11.9\%) followed by California (13.6\%).
Such patterns are less apparent during the Ancestral wave, where Ohio and
Maryland had the lowest ratio of reported cases to infections at 21.4\% and
21.7\%, respectively. 

\Cref{fig:choro_inf_case_rates} shows that using cases as a proxy for infections can lead 
to misunderstandings in the states that are affected and the extent that they
are affected. For example, on October 20, 2020, while the case rates are elevated in a
handful of upper-Midwestern states (namely, North and South Dakota), the infection
rates are elevated to a similar extent in the surrounding states,
indicating a wider impact than is suggested by cases. 
On July 20, 2021, while
the map of case rates shows low and geographically consistent impact, infection
rates reveal that Texas, Louisiana, Georgia, and their neighbors are hotspots at
 that time. 
 
By focusing on states with elevated cases, infection outbreaks may be
overlooked. For instance, on August 27, 2021, Montana and Idaho have some of the
highest infection rates (\Cref{fig:choro_inf_case_rates}).
In contrast, their case rates are unremarkable, 
whereas the highest case rates tend to be localized in the Southeast. 
Moreover, infections outbreaks tend to precede case outbreaks, though 
the lead time can vary widely. Over the Delta wave,
infections in Montana and Idaho lead cases by about 41 and 6 days at the peaks 
(\Cref{fig:state_infect_est}). Such trends are also observed during the Ancestral wave,
where infections peak about 12 and 24 days earlier than cases for these same states.
The temporal discrepancies underscore the importance of clearly distinguishing
between infection and cases when assessing the disease burden 
and spread of COVID-19.

\begin{figure}[H]
\centering
\includegraphics[width=.99\textwidth]{choro-maps-1.pdf}
\caption{Choropleth maps of the state-level estimates of the daily new
infections per $100,000$ population (top row) and the daily new cases per
$100,000$ population (bottom row) for five select dates between June 1, 2020 and
November 29, 2021. Note that the first date was chosen as a baseline, while the
other dates were chosen because they present large counts of infections across
all states. In particular, the third and fifth dates present the largest number
of total infections across the 50 states within those calendar years.} %Note that
%the colors are scaled differently for infections and cases to enable relative comparisons.} 
\label{fig:choro_inf_case_rates}
\end{figure}    

\begin{figure}[H]
\centering
\includegraphics[width=.72\linewidth]{corr-plot-1.pdf} 
\caption{Spearman's rank correlation between each of cases and
infections with hospitalizations per 100,000. These are calculated for each lag,
state, and rolling window of 61 days before averaging. The vertical dashed lines
indicate the lags for which the highest average correlation is attained.}
\label{fig:correlations}
\end{figure}

\begin{figure}[H]
\centering
\includegraphics[width=\linewidth]{ihrs-1.pdf}
\caption{Time-varying IHR and CHR estimates for each state from June 1, 2020 to
November 29, 2021, obtained using the respective correlation maximizing lag from
\Cref{sec:lagged-correlations}. Note that the infection, case, and
hospitalization counts are subject to a center-aligned 7-day average to remove
spurious day of the week effects. Also note that the different starting points
across states are due to the availability of the hospitalization data.}
\label{fig:IHR_7dav}
\end{figure} 

\subsection{Insights from cross-correlations, IHRs and CHRs}
\label{sec:lagged-correlations}

The maximum average Spearman's correlation across states is
0.48, and occurs at a lag of 13 days (\Cref{fig:correlations}). In contrast, we find that the largest
average Spearman correlation for cases is 0.69 and occurs at a lag of 1 day.
That is, case reports are nearly contemporaneous to hospitalizations, while
infection estimates clearly precede them. 

As a counterpart to the correlation analysis, we compute the
time-varying infection-hospitalization ratios (IHRs) for each state using a 
13-day lag and case-hospitalization ratios (CHRs) with a 1-day lag for comparison 
\Cref{fig:IHR_7dav}). Overall, the relationship between infections and hospitalizations is
complex. It is characterized by intermittent spikes that punctuate longer periods where
the IHRs are relatively stable, remaining below 0.1 hospitalizations per
infection. 

Both IHRs and CHRs exhibit similar geospatial and temporal trends as those noted
for infections. Namely, states that are proximate (for example, North and South
Carolina) show similar temporal patterns in IHRs and CHRs. In addition, similar
spikes are evident across many states during waves of infections that are driven
by variants of concern. For example, some states exhibit a striking increase in
hospitalizations in mid-2021, which coincides with the rapid takeover of the
Delta variant \citep{hodcroft2021covariants}. 





