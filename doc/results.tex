\section{Results}
\label{sec:results}

By estimating the time series of COVID-19 infections per 100,000 inhabitants
for each \US state from June 1, 2020 to November 29, 2021, we observe rates of
infections that vary in intensity and disease burden across space and time
(\Crefrange{fig:state_infect_est}{fig:six-states}). Outbreaks in
infections precede those in cases and are consistently larger in magnitude
(we will use ``cases'' to mean ``reported cases'').

The largest per-capita outbreaks prior to Omicron were observed in the late
summer or early fall of 2021 in Louisiana, Georgia, Idaho, and Montana matching
the intuition that clusters of geographically proximate states are likely to
exhibit similar viral spread. During this time, the two states that have the
highest rate of infections on single day are Louisiana (476 infections per 100K,
on July 20, 2021, 2021) and Idaho (also 457 infections per 100K, on September 7,
2021). The period of lowest viral transmission is observed in the summer of
2020. From June 2020 to the end of August, Vermont saw less than 10 infections
per 100K per week, the longest such lull for any state.



\subsection{Infection estimates reveal waves missed by reported cases}
\label{sec:omitted-waves}

Relative to reported cases, examining estimated infections reveals a rather
different pattern. \Cref{fig:state_infect_est} shows estimates of the number of
daily new infections per 100,000 inhabitants for each \US state from June 1,
2020 to November 29, 2021 compared with reported cases. 

Nearly all states exhibit at least two major waves of infections---the first
begins in the fall of 2020 and extends into the winter season, while the second
starts in the late summer of 2021 and proceeds into mid-fall. These
represent major waves driven by the Ancestral and Delta variants, respectively.
In general, greater similarities in the strength and magnitude of outbreaks
emerge in small clusters of states that border each other. For instance, in the
Western states of Idaho and Montana or in the Southern states of North and South
Carolina the crests and troughs in the waves of infections appear in sync with each
other. Compared to the other states, consistently low rates of infections are
attained in the Northeastern states of Vermont, New Hampshire, and Maine, even
during the aforementioned Ancestral and Delta waves.

While the major Ancestral, Alpha, and Delta waves tend to be visible for most
states, there are clear outbreaks in unreported infections that are not easily
detectable from cases alone in the falls of 2020 and 2021. For example, a wave
of infections is evident in North Dakota and South Dakota over the spring of
2021 that is virtually undetectable from the reported cases. In the late summer
of 2021, the Delta wave is only faintly detectable from cases in a number of
Northeastern states (New York, Massachusetts, Connecticut, and New Hampshire),
and yet the infections suggest that it has already begun in earnest. 

\subsection{The cases-to-infections ratio varies by state and variant}
\label{sec:case-infection-ratio}

\begin{figure}[!tb]
\centering
\includegraphics[width=.99\linewidth]{adj-unadj-cases-plot-1.pdf} 
\caption{Estimates of the number of daily new infections per 100,000 population
for each \US state from June 1, 2020 to November 29, 2021 (dark blue line). The
blue shaded regions depict the 50, 80, and 95\% intervals for the estimates,
while the orange line represents the trailing 7-day average of reported cases
per 100,000.}
\label{fig:state_infect_est}
\end{figure}    

While it is clear from \Cref{fig:state_infect_est} that cases underestimate the
true burden of infections for every state, the degree to which this problem
persists varies widely across states and variants. For the Ancestral wave, the
largest discrepancies are more frequently in the Midwest: states such as
Illinois, Indiana, and Ohio. For the Delta wave, some of the largest
discrepancies between cases and infections are visible in the Western states of
Idaho and Montana, the Southeastern states of Louisiana and Georgia, and the
Midwestern states of Iowa and Nebraska. Early in the pandemic, such
discrepancies between cases and infections may be attributable to state-specific
issues with the reporting pipeline, while later, they more likely due to the
rise in asymptomatic infections across variants \citep{oph2022covid,
garrett2022high}. 

The ratio of cases to infections decreases with time. While the Delta wave is
somewhat apparent from the case counts for all states
(\Cref{fig:state_infect_est}), infection estimates suggest that case counts
severely underestimate infections during this period for many states, more so
than in earlier waves. The most extreme was New Jersey, where about 6.3\% of
estimated infections were eventually reported as cases. Similarly low are
Maryland (7.3\%), and Nevada (8.4\%), and South Dakota (10.0\%). This issue
extends to most states: in 44 states fewer than 30\% of infections eventually
appear in case reports. The case-report ratio was larger in earlier waves, and
its effects most apparent in different regions. During Alpha, Louisiana had the
lowest ratio of infections to cases (11.9\%) followed by California (13.6\%).
Such patterns are less apparent during the Ancestral wave, where Ohio and
Maryland had the lowest ratio of reported cases to infections at 21.4\% and
21.7\%, respectively. 

\Cref{fig:choro_inf_case_rates} displays the state-level daily new infections
and cases per 100,000 for five dates over June 2020 to November 2021, allowing a
closer examination of their spatial patterns. For instance, it shows
that on June 1, 2020, there is little difference between case and infection
rates across the states, while later on, the differences become more pronounced.
Furthermore, using cases as a proxy for infections can lead to
misunderstandings in the states that are affected and the extent to which they
are affected. For some days, the spatial extent of infections is understated by
cases. For example, on October 20, 2020, while case rates are elevated in a
handful of upper-Midwestern states (namely, North and South Dakota), infection
estimates are elevated to a similar extent in the surrounding states,
emphasizing a wider impact than is indicated by cases. On July 20, 2021, while
the map of case rates shows low and geographically consistent impact, infection
rates reveal that Texas, Louisiana, Georgia, and their neighbors are hotspots at
that time. 

By focusing on states with elevated cases, infection outbreaks may be
overlooked. For instance, on August 27, 2021, Montana and Idaho have some of the
highest infection rates. In contrast, the case rates are unremarkable for these
two states, whereas the highest case rates tend to be localized in the Southeast. However, the opposite occurs as well: on December
17, 2020, Tennessee and California have the highest case rates but infections
are largely similar to other states.

\begin{figure}[!tb]
\centering
\includegraphics[width=.99\textwidth]{choro-maps-1.pdf}
\caption{Choropleth maps of the state-level estimates of the number of daily new
infections per $100,000$ population (top row) and the daily new cases per
$100,000$ population (bottom row) for five select dates between June 1, 2020 and
November 29, 2021. Note that the first date was chosen as a baseline, while the
other dates were chosen because they show large counts of infections across
all states. In particular, the third and fifth dates present the largest number
of total infections across the 50 states within those calendar years. Note that
the colors are scaled differently for infections and cases to enable relative comparisons.} 
\label{fig:choro_inf_case_rates}
\end{figure}    



    
\subsection{Infections, overall and by variant, emphasize earlier outbreaks}
\label{sec:infections-by-voc}

\Cref{fig:six-states} examines the infection estimates for a selection of states
more closely. The top panel shows infection estimates for these states, while
the bottom panel disaggregates their deconvolved cases based on the locally
circulating variant proportions at the time. These figures show times when the
total infections emphasize earlier outbreaks than are indicated by cases alone.
During the Ancestral wave, infections in Massachusetts, Idaho, Montana,
Louisiana, and Ohio peak earlier than cases. For these states, the infections
peak about 17 days earlier on average, with Massachusetts attaining the maximum
difference of 26 days. Such trends are also observed in the major Delta wave in
the states that present a prominent peak during this time such as Montana and
Louisiana, where infections lead cases by about 41 and 24 days, respectively.
The division by variant categories reveals the variant(s) that are behind these
waves. For example, in California alone, the Epsilon wave appears to coincide
with a second Ancestral wave. To give another example, we can see a
major increase in Alpha in Massachusetts over the spring of 2021. To a lesser
extent, this trend is apparent for all of the other states, save for California,
where Alpha is not a major driver of infections in comparison to the other
variants in circulation around that time. 

\begin{figure}[!tb]
\centering
    \includegraphics*[width=\linewidth]{six-decon-var-1.pdf}
    \caption{Panel A: Reported cases (orange) and estimates of daily new
    infections (dark blue) per 100K inhabitants. The blue shaded regions
    indicate 50, 80, and 95\% confidence bands.  
    Panel B: Deconvolved cases colored by variant per 100K inhabitants.}
    \label{fig:six-states}
\end{figure}


\subsection{Infections lead hospitalizations according to cross-correlations}
\label{sec:lagged-correlations}

We systematically investigate the temporal relationship between infections and
hospitalizations with Spearman's rank-correlation across different lags
(\Cref{fig:correlations}). The maximum average correlation across states is
0.48, occurring at a lag of 13 days. In contrast, we find that the largest
average Spearman correlation for cases is 0.69 and occurs at a lag of 1 day.
That is, case reports are nearly contemporaneous to hospitalizations, while
infection estimates clearly precede them. 

With respect to previous literature, the maximum correlation being attained at a
lag of 13 days is fairly consistent with estimates of the average time from
infection to hospitalization for cases reported in January, 2020 in Wuhan, China
(9.7 days) as well as with estimates from across the pandemic in the UK (ranging
from 8.0 to 9.7 days) \citep{linton2020incubation, ward2021understanding}.
Importantly, our 13 day lag for the \US\ also includes the impact of the
reporting pipeline, a delay omitted from the international estimates. 

\begin{figure}[!tb]
\centering
\includegraphics[width=.8\linewidth]{corr-plot-1.pdf} 
\caption{Spearman's rank correlation between each of cases and
infections with hospitalizations per 100,000. These are calculated for each lag,
state, and rolling window of 61 days before averaging. The vertical dashed lines
indicate the lags for which the highest average correlation is attained.}
\label{fig:correlations}
\end{figure}
    

The average correlation is consistently larger for cases than infections (with a
difference of about 0.21 at the peaks). This increase is likely due to two
reasons. First, many cases are detected contemporaneously with hospitalization:
people may first test positive only when they go to to the hospital for
treatment. Second, unreported infections tend to be less severe and less likely
to lead to hospitalization than those that are reported
\citep{sallahi2021using}.



\subsection{IHR estimates tend to be smaller and exhibit less pronounced spikes}
\label{sec:ihrs}

As a counterpart to the correlation analysis, we compute the time-varying
infection-hospitalization ratios (IHRs) for each state using the correlation
maximizing lag (13 days). We similarly compute the case-hospitalization ratios
(CHRs) using their correlation maximizing lag for comparison (1 day,
\Cref{fig:IHR_7dav}). For each state, the CHRs tend to be larger in comparison
to the IHRs. This is consistent with the claim that reported infections are more
likely to require hospitalization than unreported infections. 


\begin{figure}[!tb]
\centering
\includegraphics[width=\linewidth]{ihrs-1.pdf}
\caption{Time-varying IHR and CHR estimates for each state from June 1, 2020 to
November 29, 2021, obtained using the respective correlation maximizing lag from
\Cref{sec:lagged-correlations}. Note that the infection, case, and
hospitalization counts are subject to a center-aligned 7-day average to remove
spurious day of the week effects. Also note that the different starting points
across states are due to the availability of the hospitalization data.}
\label{fig:IHR_7dav}
\end{figure}


Both IHRs and CHRs exhibit similar geospatial and temporal trends as those noted
for infections. Namely, states that are proximate (for example, North and South
Carolina) show similar temporal patterns in IHRs and CHRs. In addition, similar
spikes are evident across many states during waves of infections that are driven
by variants of concern. For example, some states exhibit a striking increase in
hospitalizations in mid-2021, which coincides with the rapid takeover of the
Delta variant \citep{hodcroft2021covariants}. This finding aligns with previous
studies that found an increased risk in hospitalization due to Delta
\citep{twohig2022hospital, nyberg2022comparative}. Interestingly, when the
Ancestral variant dominates in 2020, there is a spike in the IHRs that rivals or
sometimes even surpasses that which is observed during Delta. This situation is
most readily observable in New England as well as in some Western states such as
Arizona, New Mexico, and Wyoming.

Overall, the relationship between infections and hospitalizations is
complicated. We observe intermittent spikes that punctuate longer periods where
the IHRs are relatively stable, remaining below 0.1 hospitalizations per
infection. While we computed the IHRs and CHRs for all states, it is important
to note that both likely vary within states and depend on confounding variables
such as age and the presence of major comorbidities
\citep{russell2023comorbidities}. Therefore, it would be beneficial to account
for such variables in their calculations by, for example, stratifying infections
and hospitalizations by age to produce age-specific estimates of the IHRs for
each state~\citep{fox2023disproportionate}.





