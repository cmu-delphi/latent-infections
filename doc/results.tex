\section{Results}
\label{sec:results}

By estimating the time series of COVID-19 infections per 100,000 inhabitants
for each \US state from June 1, 2020 to November 29, 2021, we observe rates of
infections that vary in intensity and disease burden across space and time
(Figures \ref{fig:state_infect_est}--\ref{fig:six-states}). Outbreaks in
infections precede those in reported cases and are reliably larger in magnitude.
But simply shifting cases back in time and increasing them by a constant
multiplicative factor fails to capture the spatio-temporal dynamics of the
pandemic. 

The largest per-capita outbreaks prior to Omicron were observed in the late
summer or early fall of 2021 in Georgia, Louisiana, Idaho, Montana, and Wyoming,
matching the intuition of similar viral spread in clusters of geographically
proximate states. During this time, the two states that have the highest rate of
infections on single day are Georgia (451 infections per 100K, on August 15,
2021) and Idaho (also 451 infections per 100K, on September 7, 2021). The period
of lowest viral transmission is observed in the summer and fall of 2020. During
this time, New Hampshire saw only about 1 new infection per million residents
per week \attn{really?}. From June 2020 to the end of August, Vermont saw only
about 10 infections per 100K per week, the longest such lull for any state.



\subsection{Infection estimates reveal waves missed by reported cases.}
\label{sec:omitted-waves}

Relative to reported cases, examining estimated infections reveals a
rather different pattern. \autoref{fig:state_infect_est} shows
estimates of the number of daily new infections per 100,000 inhabitants for each
\US state from June 1, 2020 to November 29, 2021 compared with reported cases,
and deconvolved cases---reported cases ``pushed back'' by the delays shown in
\autoref{fig:chain_events_onset_report}. 

Most states exhibit at least two major spikes in infections---the first starts
in the fall of 2020 and extends into the winter season, while the second starts
in the late summer of 2021 and proceeds into the mid-fall. These represent major
waves driven by the Ancestral and Delta variants, respectively. Similar patterns
of these major surges are observed in nearly all states, though to varying
degrees. In general, greater similarities in the strength and magnitude of
outbreaks are found to emerge in the clusters of states that border each other.

While the major Ancestral, Alpha, and Delta waves tend to be visible for most
states, there are clear outbreaks in unreported infections that are not easily
detectable from cases alone in the falls of 2020 and 2021. Additionally, a wave
of infections is present in the spring of 2021 for North Dakota and South
Dakota, preceeding the Alpha wave in other jurisdictions. For the specific date
of Oct. 20, 2020, \autoref{fig:choro_inf_case_rates} shows increased case
reports in a cluster of Northern-Midwest states like North Dakota, South Dakota,
and Wisconsin relative to others. However, the infection rates more clearly
emphasize that this phenomenon is not limited to these states, but extends out
to the surrounding states as well. In the fall of 2021, we can see that the
major Delta wave is only faintly detectable from cases in a number of eastern
states such as Maryland and Connecticut. It is clear that cases fail to
adequately capture the rise and fall in infections during this time.

\subsection{The cases-to-infections ratio varies by state and variant.}
\label{sec:case-infection-ratio}

While it is clear from \autoref{fig:state_infect_est} that cases underestimate
the true burden of infections for every state, the degree to which this problem
persists varies across states and variants. For the Delta wave, some of
the greatest discrepancies between cases and infections are visible in the
Western states of Idaho and Montana, the Southern states of Louisiana and
Georgia, and the Midwestern states of Iowa and Nebraska. In addition, we can see
that the Delta wave is only faintly detectable from cases in a number of Eastern
states (e.g., Maryland and Connecticut). The Ancestral wave is poorly
represented \attn{what does ``poorly represented'' mean} by cases in several
midwestern states (for example, Illinois, Indiana, and Ohio). Early
in the pandemic, such discrepancies between cases and infections may be
attributable to state-specific issues with the reporting pipeline, while later, they more likely due to the rise in asymptomatic infections
across variants \citep{oph2022covid, garrett2022high}. 

The ratio between cases and infections decreases with time. While the Delta wave
is somewhat apparent from the case counts for all states
(\autoref{fig:state_infect_est}), infection estimates suggest that case counts
severely underestimate infections during this time for many states, moreso than
in earlier waves. The most extreme was New Jersey, where about 4.6\% of
estimated infections were eventually reported as cases. Similarly low are
Maryland (7.4\%), Connecticut (8.0\%), and Florida (8.7\%). This issue extends
to most states: in 39 states fewer than 30\% of infections would eventually
appear in case reports. This ratio was less extreme in earlier waves, and its
effects most apparent in different regions. During Alpha, Louisiana had the
lowest ratio of infections to cases (11.7\%) followed by California (14.4\%).
Such patterns are even less apparent during the Ancestral wave, where Ohio and
Maryland had the lowest ratio of reported cases to infections at 22.0\%  and
22.3\%, respectively. 

\autoref{fig:choro_inf_case_rates} shows that on June 1, 2020, there is little
difference between case and infection rates across the states, while
later on, the differences become more pronounced.
%% Maybe this sentence and the next few belong more in Section 2.2
For example, on July 20, 2021, while the map of case rates shows low and
geographically consistent spread, infection rates reveal that Texas, Louisiana,
Georgia, and their neighbors are hotspots at that time. Generally, the spatial
extent of infections is often understated by cases. For example, on October 20,
2020, while case rates are elevated in a handful of upper-Midwestern states
(namely, North and South Dakota), they fail to reveal the impact on the
surrounding states: infection rates are similarly elevated in all of the
surrounding states. 

By focusing on wtates with elevated cases, infection outbreaks may be
overlooked. For instance, on August 27, 2021, Montana and Idaho have some of the
highest infection rates. In contrast, the case rates are unremarkable for these
two states, whereas the highest rates tend to be localized to the Southeastern
states. However, the opposite occurs as well: on December
17, 2020, Tennessee and California have the highest case rates but minimal
infections relative to other states.

\begin{figure}[!tb]
\centering
\includegraphics[width=.99\linewidth]{state_niauc_est_faceted_F24.pdf} 
\caption{Estimates of the number of daily new infections per 100,000
population for each \US state from June 1, 2020 to November 29, 2021
(dark blue line). The blue shaded regions depict the 50, 80, and 95\%
confidence intervals for the estimates, while the teal line represents
the number of new daily new deconvolved cases per 100,000, and the
dotted orange line represents the 7-day average of the new cases per
100,000.}
\label{fig:state_infect_est}
\end{figure}


\begin{figure}[!tb]
\centering
\includegraphics[width=.99\textwidth]{choro_inf_case_rates_F24.pdf}
\caption{Choropleth maps of the state-level estimates of the number of daily new
infections per $100,000$ population (top row) and the daily new cases per
$100,000$ population (bottom row) for five dates between June 1, 2020 to
November 29, 2021. Note that the first date was chosen as a baseline, while the
other dates were chosen due to having large counts of infections across all
states. In particular, the third and fifth dates present the largest number of
infections across the 50 states from each year.} 
\label{fig:choro_inf_case_rates}
\end{figure}    



    
\subsection{Infections, overall and by variant, emphasize earlier outbreaks.}
\label{sec:infections-by-voc}

\autoref{fig:six-states} examines the infection estimates for a selection of
states more closely. The top panel shows infection estimates for these states,
while the bottom panel separates their estimated deconvolved cases based on the
circulating variant proportions at the time. These figures show times when the
total infections and the deconvolved cases broken down by the variant categories
emphasize earlier outbreaks than would be indicated by cases alone. For example
the major Ancestral wave in California, Maryland, Idaho, Montana, and Ohio
peaks earlier for infections than cases. Such trends are similar with Delta,
though more obviously in Louisiana, Idaho and Montana than California, Maryland
and Ohio. The division by variant categories reveals the variant or variants that
are behind these waves. The crest-trough patterns of these by-variant depictions
align with infections rather than the cases by construction (as they are for
deconvolved cases which are re-scaled to get the infection estimates). 
\attn{These last two sentences are not sufficiently direct: they need to say something interesting, without too much mealy-mouthedness.}


\begin{figure}[!tb]
\centering
    \includegraphics[width=\linewidth]{state_niauc_est_6states_F24.pdf}\\
    \includegraphics[width=\linewidth]{state_decon_byvar_est_6states_F24.pdf}
    \caption{Top panel: Reported cases, deconvolved cases, and estimates of
    daily new infections (dark blue line) per 100K inhabitants. The blue shaded
    regions indicate the 50, 80, and 95\% confidence bands, while the background
    is shaded to indicate the dominant variant in circulation at the time.  
    Bottom panel: Deconvolved cases colored by variant per 100K inhabitants.}
    \label{fig:six-states}
\end{figure}


\subsection{The relationship between infections and hospitalizations is messy.}
\label{sec:lagged-correlations}

We systematically investigate the temporal relationship between infections and
hospitalizations with Spearman's rank-correlation across different lags,
shifting hospitalizations backward to align with infections
(\autoref{fig:correlations}). The maximum average correlation across states is
0.51, occurring at a lag of 13 days. In contrast, we find that the greatest
average Spearman correlation for cases is 0.69 and occurs at a lag of 1 day.
That is, case reports are nearly contemporaneous to hospitalizations, while
infection estimates clearly precede them. The maximum correlation at a lag of 13
days is in keeping with estimates of the average time from infection to
hospitalization for cases reported in January, 2020 in Wuhan, China (9.7 days)
as well as with estimates from across the pandemic in the UK (rangeing from 8.0
to 9.7 days) \citep{ward2021understanding}. Importantly, the 13 day lag in the
\US\ also includes the impact of the reporting pipeline, a delay omitted from
the international estimates. 

\begin{figure}[!tb]
\centering
\includegraphics[width=.9\textwidth]{adj_unadj_cases_hosp_lag_corr_F24.pdf} 
\caption{Spearman's correlation between each of cases, deconvolved cases, and
infections with hospitalizations per 100,000. These are calculated for each lag,
state and rolling window of 61 days before averaging. The vertical dashed lines
indicate the lags for which the highest average correlation is attained.
\attn{Fig still wrong}}
\label{fig:correlations}
\end{figure}
    

While both the infections and deconvolved cases are leading indicators of
hospitalizations and their trajectories are similar, the average correlation
they attain is different. In particular, the correlation is larger for
deconvolved cases than for infections (with a difference of about 0.18 at the
peaks). The increase is likely due to 2 issues. First, many cases are detected
contemporaneously with hospitalization: people first test positive once they go
to the emergency room for treatment. Second, unreported infections tend to be
less severe and less likely to lead to hospitalization than those that are
reported \citep{sallahi2021using}.



\subsection{Infection-hospitalization ratios are smaller and more stable.}
\label{sec:ihrs}

As a counterpart to the correlation analysis, we compute the time-varying
infection-hospitalization ratios (IHRs) for each state using the correlation
maximizing lag. We similarly compute the case-hospitalization ratios (CHRs)
using their correlation maximizing lag for for comparison
(\autoref{fig:IHR_7dav}). 

For each state, the CHRs tend to be larger and noiser relative to IHRs. This
supports our claim that reported infections are more likely to require
hospitalization than unreported infections. Both IHRs and CHRs exhibit similar
geospatial and temporal trends as those noted for infections above. Namely,
states that are proximate (for example, Ohio, Pennsylvania, and Virginia) tend
to exhibit similar temporal patterns in IHRs and CHRs. In addition, similar
spikes are evident across many states during waves of infections that are driven
by variants of concern. For example, many states exhibit a striking increase in
hospitalizations in mid-2021, coinciding with the rapid takeover of the Delta
variant \citep{hodcroft2021covariants}. This finding aligns with previous
studies that found an increased risk in hospitalization due to Delta
\citep{twohig2022hospital, nyberg2022comparative}. Similarly, during the fall of
2020 a spike in IHRs rivals or surpasses that observed during the time of Delta
(which is the case for states like New York or Wyoming). \attn{what is the
``Similarly'' referring to here?}

Overall, the relationship between infections and hospitalizations is
complicated. We observe intermittent spikes that punctuate longer periods where
the IHRs are relatively stabile, remaining below 0.1 hospitalizations per
infection. 
% There does not tend to be a strict upward or downward trajectory or even a mild
% waning pattern in the IHRs, as one might expect with later variants that are
% more infectious but result in fewer hospitalization \citep{lorenzo2022covid,
% blauer2022compare}. \attn{Is the IHR actually H/I? or are the units not quite
% right (e.g., H/1M / I/100K)?}
While we computed and compared CHRs and IHRs for all states, it is important to
note that both likely vary within states and depend on confounding variables
such as age and the presence of major comorbidities
\citep{russell2023comorbidities}. Therefore, it would be beneficial to account
for such variables in their calculations by, for example, stratifying infections
and hospitalizations by age to produce age-specific estimates of the IHRs for
each state~\citep{fox2023disproportionate}.



\begin{figure}[!tb]
\centering
\includegraphics[width=.99\linewidth]{IHR_7dav_F24.pdf}
\caption{Time-varying IHR and CHR estimates for each state from June 1, 2020
to November 29, 2021, obtained using the corresponding optimal lag from the
systematic lag analysis. Note that the infection, case, and hospitalization
counts are subject to a center-aligned 7-day average to remove spurious day
of the week effects. Also note that the different starting points across
states are due to the availability of the hospitalization data.}
\label{fig:IHR_7dav}
\end{figure}


