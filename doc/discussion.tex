\section{Discussion}

We retrospectively estimated daily incident infections for each \US state over
the period June 1, 2020 to November 29, 2021. Our estimates suggest both (a)
that the pandemic impacted states earlier and at a larger scale than is
indicated by reported cases and that (b) using cases as a proxy for 
infections can lead to different misunderstandings and mischaracterizations of
trends in infections. Moreover, we observe
outbreaks in infections that are difficult to detect from cases alone such as
the Delta wave in New Jersey, Connecticut, and Maryland. This suggests that
cases paint an incomplete picture of the pandemic, especially when outbreaks are
largely driven by unreported infections. Furthermore, since case reports
generally follow symptom and infection onsets, cases have a built-in temporal
bias. This is in addition to other biases from differences in reporting across
states such as temporary bottlenecks due influxes of data or more persistent
processing issues that increase the average time from case detection to report
\citep{wash2020dash, dunkel2020covid19}. Thus, while reported cases provide an
indication of the trajectory of the pandemic, it is a delayed and incomplete
version.

Our approach offers a number of advantages. By incorporating 
state-level case, line list, and variant circulation data, we are able to construct 
incubation and delay distributions that are state and spatio-temporally specific.
Time-varying and state-specific seroprevalence data allows the reporting ratio
estimates to similarly vary over space and time, a departure from existing work
\citep{unwin2020state, uga2020covid19}. Unlike previous approaches that use
a single delay distribution to generate estimates 
for all states \citep{chitwood2022reconstructing,
jahja2022real}, our work avoids this assumption of geographic
invariance. This assumption is far from realistic due to
differences in reporting pipelines, pandemic response, and variants in
circulation, among other issues. 

Another limitation of such approaches to estimate infections 
is that they often fail to account for reinfections. While reinfections constitute a small
portion of the total infections until the arrival of high immune escape variants (BA.1), 
disregarding them means
that the infection-reporting ratio will tend to be underestimated with
seroprevalence data alone. By accounting for reinfections as well as the waning of
seropositivity (See \autoref{sec:report-ratio}), we more accurately estimate
this ratio.
However, we acknowledge that in future work we could refine the incorporation of these
factors. Since the waning of immunity is likely to
be variant-dependent \citep{pooley2023durability}, then it follows that our model's
waning parameter may be better posed as a mixture of parameters for different
variants with weights determined by the proportion of the variants circulating
at the time in the state. 
%Relatedly, newer variants may escape detection
%\citep{nih2022assessing, fda2023sars}. While in a retrospective analysis, where
%finalized data is used, this is less likely to be an issue, detection escape
%would pose a problem for real-time estimates of infections.

The major reason why we chose an end date of November 29,
2021, and ultimately decided to not continue into the Omicron wave is because
the Omicron variants come with substantial increases in the risk of reinfection
in comparison to previous variants, likely due to increased immune escape
\citep{wei2024risk, pulliam2022increased, eythorsson2022rate}. Access to
reinfection data that is representative of each location under study is paramount
for extending the analysis. While it would be ideal to use the reinfection rates
over time for each \US state, many states do not publicly report reinfection
data over the entire time period under examination, if at all.


% While it would be ideal to use confirmed rates over time for each \US state,
% most states do not publicly report reinfection data over the entire time period
% we considered. So we have turned to suspected reinfection data over time for
% Clark County, USA, as that surveillance is among the most detailed and reputable
% that we have found for the United States. Nevertheless, using such localized
% data raises questions of representativeness and the applicability of such
% estimates to Nevada and all other states. Furthermore, this data has no
% information available beyond suspected third infections, which imposes an
% irremediable bias. However, based on the third infection data available there,
% we expect that the probability of being reinfected more than three times is
% likely very low for time frame considered and so the omission of these would
% impact our infection estimates to a minimal extent. 

% The vast majority of issues we encountered when trying to reconstruct the
% infection time series for each state are due to an absence or a lack of data.
% Such is the primary issue we had with the restricted line list. In comparison to
% the number of JHU cases (which we are treating as a gold standard) for the same
% release date, we noted there are about $10$ million cases that are unaccounted
% for in the CDC line list. Moreover, the missingness does not appear to be random
% and uniformly distributed across states. Rather it is unequally distributed,
% suggesting that the dataset is likely biased. However, more information on the
% cases that are missing versus present would be required to determine the extent
% the missing cases led to a nonrepresentative, and therefore, biased sample, and
% could be a topic of further study.

Using seroprevalence data to estimate the case-ascertainment ratio past December 2021
is subject to additional issues that inhibit us from extending the period of
analysis. Specifically, while most state-level data suggests that
reinfections still account for less than 20\% of reported cases during Omicron
\citep{ruff2022rapid, nyreinfect2021, hireinfect2022, wareinfect2022},
seropositivity rapidly reaches nearly 100\% of the population, precluding its
continued use. Due to such data-driven limitations, alternative data sources for estimating the
case-ascertainment ratio should be considered. For example, wastewater surveillance data
may be complementary to seroprevalence data, especially when testing is low,
or serve as a substitute when it is unavailable \citep{mcmanus2023predicting}.
% However, viral detection is inconsistent across
% locations due to temperature, per-capita water use, and in-sewer travel time
% \citep{mcmanus2023predicting, hart2020computational, li2023correlation}.
Sentinel surveillance streams for influenza-like illness or acute respiratory
infection may provide decent proxies for COVID-19 incidence, especially when
testing for mild cases of COVID-19 is diminishing or has ceased completely.
Finally, alternative surveillance streams such as those from surveys, 
helplines, or medical records could potentially be
integrated if they offer a sufficiently strong signal of the disease intensity
over time \citep{reinhart2021open,ecdc2020strategies}.



% \attn{Some into Methods, some into Supplement}
% also runs the risk of being nonrepresentative of the
% intended population \citep{bajema2021estimated}. For example, in the blood donor
% dataset some states have region specific-estimates, which clearly do not stand
% for the entire state. Another source of systematic variation is in the
% characteristics of the individuals who opt for blood tests versus those who do
% not. For instance, there may be a healthy user bias, in which a number of those
% who opt for blood tests are generally more inclined to partake in proactive
% healthy behaviors (such as checking on basic health markers by taking an annual
% blood test) than those who do not \citep{parsley2018blood}. Alternatively, a
% number of individuals may be recommended for blood tests by their doctors due to
% signs of ill-health (ex. mineral deficiencies or underlying medical conditions).
% The extent that each such bias persists depends on the purpose of the blood test
% and whether it was used as a proactive or reactive medical tool. Since such
% information is unavailable to us, all we can conclude is that participant-driven
% sources of bias impact the seroprevalence samples to an undetermined extent.
% There are additional concerns about the performance of antibody testing for
% individuals with mild or asymptomatic disease as well as about the loss of
% immunity over time \citep{kaku2021performance, seow2020longitudinal,
% ibarrondo2020rapid}.
% 
% In this work, we do not attempt to directly address infection underascertainment
% due to the increase in asymptomatic infections across variants
% \citep{pho2023covid19}. We simply note that this would likely pose a greater
% problem later in the pandemic, particularly after the Delta era
% \citep{fan2022sars}. We hope that such infections would be largely represented
% by the seroprevalence and reinfection estimates, but there is undoubtedly
% increasing reliance on such estimates to be able to do this over time (owing to
% the simultaneous decline in the reporting cadence and the apparent rise in
% asymptomatic infections over time) \citep{oph2022covid, garrett2022high,
% blauer2022reduce, ren2021asymptomatic}. Consequently, there is an increasing
% uncertainty over time that is not captured by the model or the estimates.   


Overall, we adopt a deconvolution-based approach and devote
much of our efforts to tailoring our approach to the available data. A major
result of this is the development of a model for the waning of detectable
antibody levels and space-time-specific reporting ratios based on seroprevalence
data. In a way, our approach is built for the data rather than trying to force
the data to fit to an existing modelling framework. We are
encouraged by the plausibility of the resulting infection estimates
as well as their geospatial and temporal trends.
 


% Our approach is predicated upon having case, line list, viral circulation, and
% seroprevalence data for each state, all of which are readily available (or
% available upon request in the case of restricted line list data). As a result of
% this, we are able to demonstrate the feasibility of estimating COVID-19
% infections at the state level by using standard sources of data. 

% Our framework is quite versatile as it lends itself to more localized, county or
% community level estimates, or globalized, country-specific estimates.
% Fundamentally, to produce estimates of infections for different geographic
% regions, one would simply need to input the required data and re-run the code
% pipeline. In this way, one could readily adapt our approach to generate
% estimates for the provinces in Canada or the regions in England.

Well-informed, localized estimates of COVID-19 infections over time can provide
a clearer and more comprehensive understanding of the course of the pandemic.
Such estimates contribute important information on the timing and magnitude of
the disease burden for each location and they highlight trends that may not be
visible from reported case data alone. Therefore, our infection estimates
provide key information for the ongoing investigation on the true size and impact of
the pandemic.
