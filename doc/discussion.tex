\section{Discussion}

We retrospectively estimated daily incident infections for each \US state over
the period June 1, 2020 to November 29, 2021. Our  estimates suggest both (a)
that the pandemic impacted states earlier and at a larger scale than is
indicated by cases and that (b) examining cases alone hides some spatio-temporal
waves that become apparent by examining infections. We observe outbreaks in
infections that are difficult to detect from cases alone such as the Delta wave
in New Jersey, Connecticut, and Maryland. This suggests that cases paint an
incomplete picture of the pandemic, especially when outbreaks are largely driven
by unreported infections. Furthermore, since case reports generally follow
symptom and infection onsets, cases have a built-in temporal bias. This is in addition
to other biases from differences in reporting across states (such as temporary
bottlenecks due influxes of data or more persistent processing issues that
increase the average time from case detection to report \citep{wash2020dash,
dunkel2020covid19}. 
% Furthermore, no indication of uncertainty is provided for
% even the gold standard case estimates \citep{delphiepidata2020}. 
Thus, while reported cases provide an indication of the trajectory of the
pandemic, it is a delayed and incomplete version.

% Since case reporting is not
% consistent across time and states, case counts underestimate the true number of
% infections and, hence, the impact of the pandemic \citep{cdc2022estimated,
% simon2022inconsistent}. For example, some states report the number of
% individuals tested rather than the numbers of tests performed
% \citep{schechtman2020counting, chitwood2022reconstructing}. Additionally, while
% the definition of a confirmed COVID-19 case tends to be fairly uniform across
% the United States due to general adherence to the CSTE case definitions, state
% reporting standards have been known to vary \citep{cste2020, delphiepidata2020}.
% For instance, there may be inconsistencies across locations if some cases are
% labelled as confirmed based on positive antigen tests instead of PCR tests
% \citep{covidtracking2021}. 
% As well, the definition of a case and related terminology can change
% or evolve over time as more information becomes available.  

% Estimating the new number of infections by symptom or infection onset date would
% more closely align with the definition of incidence as we know it
% \citep{jahja2022real}.



% The remainder of our discussion consists of an in-depth look into the advantages
% and limitations of our approach and of other comparable approaches, followed by
% a high level summary of our work and its major contributions. 

Our approach offers a number of advantages.
% The development 
% of a way of modelling immunity and space-time-specific reporting ratios based on 
% seroprevalence data.
% Similar phrase on line 1268, though we may want to emphasize that point up here.
% To the best of our knowledge, no other modelling approach has been used to 
% reconstruct the infection time series
%for every state over as much of the COVID-19 pandemic as in this study. %%
%Furthermore, 
For instance, we aim to incorporate as much state-specific information as
possible when deriving our estimates. By using state-level case, line list, and
variant circulation data, we are able to construct incubation and delay
distributions that are specific to each state. Time-varying and state-specific
seroprevalence data allows the reporting ratio estimates to similarly vary over
space and time, a departure from existing work \citep{unwin2020state,
uga2020covid19}. Existing approaches that use the delay distribution to generate
infection estimates often only construct one delay distribution that is used for
all states \citep{chitwood2022reconstructing, jahja2022real}. That is, our work
avoids the assumption of geographic invariance, where it is assumed that
all states have the same patterns of delay from symptom onset to case report.
This assumption is
unlikely to be true due to differences in reporting pipelines, pandemic
response, and variants in circulation, among other issues. 

Another limitation of previous approaches to estimate latent infections is
that they do not to account for reinfections. While
reinfections represent a small fraction of total infections until
later in the pandemic, ignoring them means that the
infection-reporting ratio will tend to be underestimated with seroprevalence
data alone. By accounting for these as well as the waning of seropositivity (See
\autoref{sec:report-ratio}), we more accurately estimate this ratio.
% so they are not absolutely necessary to include in the
% earlier stages of the pandemic. Still, at no stage did infection confer lifelong
% immunity. Rather antibody levels and immunity are known to wane over time. And
% we believe it is important to account for such defining characteristics of the
% virus when tracking infections over time. Therefore, we account for reinfections
% and the waning of detectable antibody levels in our custom antibody prevalence
% model. 
However, we acknowledge that the extent to which each of these are
accounted for could be improved upon in future work. 
Since the waning of immunity is likely to be variant-dependent
\citep{pooley2023durability}, it follows that our model waning parameter may be
better posed as a mixture of parameters for different variants with weights
determined by the proportion of the variants circulating at the time in the
state. Related to this is the issue that newer variants may escape detection
\citep{nih2022assessing, fda2023sars}. While in a retrospective analysis where
finalized data is used this is less likely to be an issue, this could very well
pose a problem for real-time estimates of infections.

Regarding reinfections, a major reason why we chose an end date of November 29,
2021 and ultimately decided to not tread into Omicron territory is because the
Omicron variants come with substantial increase in the risk of reinfection in
comparison to previous variants as Omicron has been shown to have an increased
tendency towards immune escape \citep{wei2024risk, pulliam2022increased,
eythorsson2022rate}. So having quality reinfection data that is representative
of each location under study is of the utmost importance for the Omicron era. 

% While it would be ideal to use confirmed rates over time for each \US state,
% most states do not publicly report reinfection data over the entire time period
% we considered. So we have turned to suspected reinfection data over time for
% Clark County, USA, as that surveillance is among the most detailed and reputable
% that we have found for the United States. Nevertheless, using such localized
% data raises questions of representativeness and the applicability of such
% estimates to Nevada and all other states. Furthermore, this data has no
% information available beyond suspected third infections, which imposes an
% irremediable bias. However, based on the third infection data available there,
% we expect that the probability of being reinfected more than three times is
% likely very low for time frame considered and so the omission of these would
% impact our infection estimates to a minimal extent. 

% The vast majority of issues we encountered when trying to reconstruct the
% infection time series for each state are due to an absence or a lack of data.
% Such is the primary issue we had with the restricted line list. In comparison to
% the number of JHU cases (which we are treating as a gold standard) for the same
% release date, we noted there are about $10$ million cases that are unaccounted
% for in the CDC line list. Moreover, the missingness does not appear to be random
% and uniformly distributed across states. Rather it is unequally distributed,
% suggesting that the dataset is likely biased. However, more information on the
% cases that are missing versus present would be required to determine the extent
% the missing cases led to a nonrepresentative, and therefore, biased sample, and
% could be a topic of further study.

Using seroprevalence data to estimate the case-ascertainment ratio is subject to
a number of issues, and precludes us from pushing the period of analysis past
the Omicron wave in December 2021. While most state-level data suggests that
reinfections still account for less than 20\% of reported cases during Omicron
\attn{cites}, seropositivity rapidly reaches nearly 100\% of the population,
precluding its continued use. Due to these issues,
alternative data sources for
estimating the case-ascertainment ratio is necessary. 
% Intuitively, one might expect that
% leveraging data from multiple sources would likely lead to more accurate and
% stable estimates than those from using one source. 
For example, wastewater surveillance data
is may be complementary to seroprevalence data,
especially when testing is low \citep{mcmanus2023predicting}. However, 
% there has
% been limited success in predicting incidence using such data. The extent that
% wastewater concentration data is a useful in estimating COVID-19 incidence is
% unclear owing to problems with viral occurrence and detectability in wastewater
% that render 
viral detection is inconsistent across locations due to temperature,
per-capita water use, and in-sewer travel time \citep{mcmanus2023predicting,
hart2020computational, li2023correlation}. Sentinel surveillance streams for
influenza-like illness or acute respiratory infection may provide decent proxies
for COVID-19 incidence, especially when testing for mild cases of COVID-19 is
diminishing or has ceased completely. Finally, alternative surveillance streams
(potentially outside of public health) such as those from surveys, helplines, or
medical records could potentially be integrated if they provide at least a rough
indication of the disease intensity over time
\citep{reinhart2021open,ecdc2020strategies}.



% \attn{Some into Methods, some into Supplement}
% also runs the risk of being nonrepresentative of the
% intended population \citep{bajema2021estimated}. For example, in the blood donor
% dataset some states have region specific-estimates, which clearly do not stand
% for the entire state. Another source of systematic variation is in the
% characteristics of the individuals who opt for blood tests versus those who do
% not. For instance, there may be a healthy user bias, in which a number of those
% who opt for blood tests are generally more inclined to partake in proactive
% healthy behaviors (such as checking on basic health markers by taking an annual
% blood test) than those who do not \citep{parsley2018blood}. Alternatively, a
% number of individuals may be recommended for blood tests by their doctors due to
% signs of ill-health (ex. mineral deficiencies or underlying medical conditions).
% The extent that each such bias persists depends on the purpose of the blood test
% and whether it was used as a proactive or reactive medical tool. Since such
% information is unavailable to us, all we can conclude is that participant-driven
% sources of bias impact the seroprevalence samples to an undetermined extent.
% There are additional concerns about the performance of antibody testing for
% individuals with mild or asymptomatic disease as well as about the loss of
% immunity over time \citep{kaku2021performance, seow2020longitudinal,
% ibarrondo2020rapid}.
% 
% In this work, we do not attempt to directly address infection underascertainment
% due to the increase in asymptomatic infections across variants
% \citep{pho2023covid19}. We simply note that this would likely pose a greater
% problem later in the pandemic, particularly after the Delta era
% \citep{fan2022sars}. We hope that such infections would be largely represented
% by the seroprevalence and reinfection estimates, but there is undoubtedly
% increasing reliance on such estimates to be able to do this over time (owing to
% the simultaneous decline in the reporting cadence and the apparent rise in
% asymptomatic infections over time) \citep{oph2022covid, garrett2022high,
% blauer2022reduce, ren2021asymptomatic}. Consequently, there is an increasing
% uncertainty over time that is not captured by the model or the estimates.   


We adopt a relatively simple deconvolution-based approach and devote
much of our efforts to tailoring our approach to the available data. A major
result of this is the development of a way of to model the waning of detectable
antibody levels and space-time-specific reporting ratios based on seroprevalence
data. In a way, our approach is built for the data rather than trying to force
the data to fit to an existing approach. However, our model is only as good as
the quality and the quantity of the data provided to it. In our case, the lack
of data is both a barrier to entry and a continual roadblock. The assumptions we
are required to make as a consequence of this clearly limit the generalizability
and call into question the reliability of the results. So while we highlight
some interesting trends and numerical findings, these results are not
definitive, but rather exploratory and intended to stimulate discussion on the
challenging task of estimating infections. Despite these limitations, we are
encouraged by the ability to use routine data to produce sensible estimates of
infections in the United States and the plausibility of the apparent geospatial
and temporal trends. 
 


% Our approach is predicated upon having case, line list, viral circulation, and
% seroprevalence data for each state, all of which are readily available (or
% available upon request in the case of restricted line list data). As a result of
% this, we are able to demonstrate the feasibility of estimating COVID-19
% infections at the state level by using standard sources of data. 

% Our framework is quite versatile as it lends itself to more localized, county or
% community level estimates, or globalized, country-specific estimates.
% Fundamentally, to produce estimates of infections for different geographic
% regions, one would simply need to input the required data and re-run the code
% pipeline. In this way, one could readily adapt our approach to generate
% estimates for the provinces in Canada or the regions in England.

Well-informed, localized estimates of COVID-19 infections over time can help us
to have a more clear and comprehensive understanding of the course of the
pandemic. Such estimates contribute important information on the timing and
magnitude of disease burden for each location and they highlight trends that may
not be visible from case data alone. Therefore, our infection estimates provide
key information for the ongoing debate on the true size and impact of the
pandemic.
