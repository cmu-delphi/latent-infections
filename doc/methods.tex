\section{Methods}
\label{sec:methods}


In what follows, we provide details on how we estimate the daily incident
infections for each state over the considered time period of June 1, 2020 to
November 29, 2021 and the data we used to achieve this.
\autoref{fig:cases_to_infect_flowchart} provides a visual summary of the major
analysis tasks, which can be summarized as follows: First, we estimate the
time-varying delays from positive specimen to report date using a line list and
use these to deconvolve reported cases to the date of the positive specimen
collection. Then, we estimate the delay from symptoms to positive specimen,
combine this with variant specific infection-to-symptoms delays, and use this
delay to push back to the date of infection. The resulting infection onset
estimates are aggregated across the variant categories and adjusted to account
for the unreported infections by using state-specific, time-varying
seroprevalence data in an antibody prevalence model. 

\subsection{From reported cases to positive specimen collection}
\label{sec:step1}

Deconvolution ``pushes back'' reported cases to the likely date of positive
specimen collection. An important aspect of our methods is that deconvolution is
not the same as a simple shift, rather it involves the distribution of delays
(specific to each state and date), as estimated from de-identified patient-level
line list data on COVID-19 cases from the CDC \citep{cdc2020caserestr}. Simply
shifting cases back in time would fail to reflect the fact that some cases take
much longer to be reported than others 
(see Supplementary Methods \Cref{supp:convol} for further details).

We will start by describing how the model for deconvolution infers the likely
dates of positive specimen collection from reported cases before describing how
the line list was used to estimate the necessary delay distribution. Together,
these are the ingredients for Step 1 in \Cref{fig:cases_to_infect_flowchart}.
Define $y_{\ell,t}$ to be the number of new cases reported in location $\ell$ at
time $t$, as disseminated by the John Hopkins Center for Systems Science and
Engineering (JHU CSSE)\citealp{dong2020interactive} and retrieved with the
COVIDcast API \citep{reinhart2021open}. Let $\pi_{\ell,t}(k)$ be the probability
that cases with positive specimen collection on time $t-k$ are reported at $t$. Then, we
model $y_{\ell,t}$ as Gaussian with mean 
\begin{align}
  \label{eq:cases-model}
  \mathbb{E}[y_{\ell,t} \mid x_{\ell,s}, s \leq t ] =\sum_{k} \pi_{\ell,t-k}(k)x_{\ell,t-k},
\end{align}
which is a probability weighted sum of the number of positive specimens
collected $k$ days earlier, $x_{\ell,t-k}$.
We estimate $\mathbf{x}_\ell = \{x_{\ell,1},\ldots,x_{\ell,T}\}$ by minimizing
the negative log-likelihood with a penalty that encourages smoothness in time.
Thus, our estimator is given by
\begin{align}
  \label{eq:cases-deconvolved}
  \widehat{\mathbf{x}}_\ell = \argmin_{\mathbf{x}}\ \sum_{t}
  \bigg( y_{\ell,t} -  \sum_k \pi_{\ell,t-k}(k) x_{t-k} \bigg)^2 
  + \lambda\ \sum_t \big|x_t - 4x_{t-1} + 6x_{t-2} -4x_{t-3}+ x_{t-4}\big|.
\end{align}
The two parts of this optimization problem trade data fidelity (the sum of
squared errors) with smoothness in the resulting estimates (the fourth order
differences of $\mathbf{x}$). The tuning parameter $\lambda$ determines the
relative importance of these competing goals. The solution to this minimization
problem is an adaptive piecewise cubic polynomial \citep{tibshirani2014adaptive,
tibshirani2022divided} and can be accurately computed with ease
\citep{ramdas2016fast,jahja2022real}. We select $\lambda$ with cross-validation
to minimize the out-of-sample reconvolution error. 
%Additional details about convolution are given in the Supplement, \Cref{supp:convol}.
% with $3$-fold cross validation
% \citep{jahja2022real} in which every third day is reserved for testing, and the
% value that results in the smallest out-of-sample mean squared error is chosen.

%\subsection{Estimating delay distributions from private line lists}
%\label{sec:delaystop}

To estimate the $\pi_{\ell,t}(k)$ for all states $\ell$, times $t$, and delays
$k$, we use the CDC line list \citep{cdc2020casepub,
cdc2020caserestr}. The line list contains three key dates of interest for many
cases that will eventually appear in case reports: symptom onset,
positive specimen collection, and report to the CDC. Handling missingness and
imputation in these dates requires careful attention, and so additional details and
justifications are provided in Supplementary Methods \Cref{supp:linelist-details}.
Define $z_{\ell,t}$ to be a case report occurring at time $t$ in
location $\ell$, and let $\pi_{\ell,t}(k)$ be the probability that
$z_{\ell,t}$ has a positive specimen collected $k$ days earlier. We assume that
all positive specimen results will be reported within 60 days and that no test will be
reported on the same date as it was collected.
Under these assumptions, let $N_{\ell,t}$ be the total number of
$z_{\ell,r}$ with positive specimen collection date $r$ in a window $r \in
[t-75+1, t+60]$ around the time of interest $t$. 
Then, we compute the observed probability mass function (pmf)
\begin{align}
  \label{eq:line-list-delay}
    \tilde{p}_{\ell,t}(k) = \frac{1}{N_{\ell,t}} \big(\textrm{\# $z_{\ell,r}$ with positive specimen at $r-k$}\big)\indicator(0<k\leq 60),
\end{align}
where $\indicator(Z) = 1$ if $Z$ is true and 0 otherwise. We also compute a
similar national pmf, $\tilde{p}_{t}(k)\indicator(0<k\leq 60)$, without
restricting to location $\ell$. Next, let $\alpha_{\ell,t}$ be the ratio of
$N_{\ell,t}$ to the number of cases reported by JHU
CSSE\cite{dong2020interactive} in the window $[t-60+2, t+75]$. Then, compute
$p_{\ell,t} = \alpha_{\ell,t}\tilde{p}_{\ell,t} +
(1-\alpha_{\ell,t})\tilde{p}_t$. This construction was adopted to allow for more
reliance on the state estimate when a larger fraction the JHU cases reports
appear in the CDC line list (and vice versa). We calculate the mean $m_{\ell,t}$
and variance $v_{\ell,t}$ of the pmf $\{p_{\ell,t}(k)\}$ and estimate the
best-fitting gamma distribution by solving the moment equations $m_{\ell,t} =
\alpha_{\ell,t}\theta_{\ell,t}$ and $v_{\ell,t}=
\alpha_{\ell,t}\theta_{\ell,t}^2$ for the shape $\alpha_{\ell,t}$ and scale
$\theta_{\ell,t}$. Finally, we discretize the resulting gamma density to the
original support to produce an estimate $\widehat{\pi}_{\ell,t}(k)$ of the delay
distribution $\pi_{\ell,t}(k)$. Additional minor details and justification for
the delay distribution calculations are deferred to Supplementary Methods
\Cref{supp:delay-justifications}.
 
\subsection{From positive specimen collection to infection onset}
\label{sec:step2-and-3}

To continue, pushing positive specimen collection time back to infection onset
(Step 2 in \Cref{fig:cases_to_infect_flowchart}), we will use a procedure very
similar to that described above and specified in
\Cref{eq:cases-model,eq:cases-deconvolved}. However, because
the delays involve the time from infection to symptom onset, these must be
variant-specific. This means that both the probabilities and the observations
must be replaced with variant-specific quantities. 

For the observations, we use our estimates from \Cref{sec:step1},
$\widehat{\mathbf{x}}_\ell$, but we weight them corresponding to the mix of
variants in circulation. To estimate the daily proportions of the variants
circulating in each state, we use GISAID genomic sequencing data from
CoVariants.org \citep{hodcroft2021covariants, elbe2017data}, and estimate a
multinomial logistic regression model. This procedure is now standard
\citep{obermeyer2022analysis, annavajhala2021emergence, figgins2021sars}, so we
defer details to Supplementary Materials \Cref{sec:variant-proportions}. The resulting
estimated probability of variant $j$ is given by $\hat{v}_{j\ell,t}$.

To estimate variant-specific delays from infection to positive specimen
collection, we convolve the location-time-specific symptom-to-test distributions
(that are estimated from the CDC line list in the same way as in \Cref{sec:step1}), 
with variant-specific incubation periods.
The convolution of these yields a distribution $\tau_{j\ell,t}(k)$. 
Details on the convolution and its inputs are deferred to Supplementary Methods
\Cref{supp:delay-sops,sec:incubation,supp:details-conv}. 

Analogous to \Cref{eq:cases-model,eq:cases-deconvolved}, for each variant $j$,
we model the variant-specific, deconvolved cases as Gaussian with mean
\begin{align}
  \mathbb{E}\left[\widehat{v}_{j\ell,t}\widehat{x}_{\ell,t} \mid u_{j\ell,s}, s \leq t  \right] = \sum_k \tau_{j\ell,t-k}(k) u_{j\ell,t-k} 
\end{align}
and estimate $\mathbf{u}_{j\ell}$ by minimizing the negative loglikelihood with
a penalty to encourage smoothness:
\begin{align}
\widetilde{\mathbf{u}}_{j\ell} = \argmin_{\mathbf{u}}\ \sum_{t} 
\left( 
    \widehat{v}_{j\ell,t} \widehat{x}_{\ell,t} -  
    \sum_k \tau_{j\ell,t-k}(k) u_{t-k} 
\right)^2 
+ \lambda\ \sum_{t} \big|u_{t} - 4u_{t-1} + 6u_{t-2} -4u_{t-3}+ u_{t-4}\big|.
\end{align} 
We call the solution $\widetilde{\mathbf{u}}_{j\ell}$ the \emph{variant-specific
deconvolved cases} and emphasize that these are cases that will eventually
be reported to public health. Because this deconvolution is done separately for
each location and variant, we add them together at each time $t$, and
we denote the total deconvolved cases at location $\ell$ as
$\widehat{\mathbf{u}}_\ell = \sum_j \widetilde{\mathbf{u}}_{j\ell}$ (Step 3 in
\Cref{fig:cases_to_infect_flowchart}). Note that these deconvolved cases are now
indexed by the time of infection onset rather than case report.




\subsection{Inverse reporting ratio and the antibody prevalence model} 
\label{sec:report-ratio}

To capture the unreported infections, it is necessary to adjust these
deconvolved case estimates by the ratio of the true number of new infections to
the new reported infections (Step 4 in \Cref{fig:cases_to_infect_flowchart}). 
Seroprevalence of anti-nucleocapsid antibodies represents the percentage
of people who have at least one resolving or past infection \citep{cdc2020data},
so we use the change in subsequent seroprevalence measurements to estimate
\emph{all} new infections rather than just those eventually appearing as cases.

We use two major contemporaneous surveys to estimate the proportion of the
population with evidence of previous infection in each state over time: the
2020--2021 Blood Donor Seroprevalence Survey and the Nationwide Commercial Lab
Seroprevalence Survey \citep{cdc2021blood, cdc2021comm}. See Supplementary
Methods \Cref{supp:sero-details} for additional details. Each of these provides
seroprevalence estimates along with confidence intervals. In order to account
for different surveys occurring on different dates with roughly weekly
availability and measurement error, we treat actual seroprevalence $s_{\ell,m}$
as a latent variable available on Monday (using $m$ rather than $t$ to denote
Mondays). Therefore, the observed seroprevalence survey measurements $r^1_m$ and
$r^2_m$ are modelled as Gaussian,
\begin{align}
\label{eq:sero-measurements}
r^1_{\ell,m} \mid s_{\ell,m},\ w^1_{\ell,m} &\sim \textrm{N}(s_{\ell,m},\ w^1_{\ell,m}\sigma^2_{\ell,r}),\\
r^2_{\ell,m} \mid s_{\ell,m},\  w^2_{\ell,m}
  &\sim \textrm{N}(s_{\ell,m},\ w^2_{\ell,m}\sigma^2_{\ell,r}),
\end{align}
with source-specific measurement errors that scale proportional to the reported
confidence intervals, respectively $w^1_{\ell,m}$ and $w^2_{\ell,m}$.  

To complete the model, we suppose that latent seroprevalence is modeled as a
Guassian with mean given by a fraction of the previous seroprevalence
measurement at $m$ plus the reinfection-adjusted deconvolved cases multiplied by
the inverse reporting ratio at time $m$:
\begin{align}
  \label{eq:expect-sero}
\mathbb{E}[s_{\ell,m+1} \mid s_{\ell,m}] & = (1 -\gamma) s_{\ell,m} 
+ a_{\ell,m} (1 - z_{m}) \sum_{t\in[m,m+1]}\widehat{u}_{\ell,t},
\end{align}
where $\widehat{u}_{\ell,t}$ are deconvolved cases (from
\Cref{sec:step2-and-3}), $z_{m}$ is the fraction of reinfections, and
$a_{\ell,m}$ is the inverse reporting ratio. Note that $\gamma$ is the fraction
of people whose level of infection-induced antibodies falls below the detection
threshold between time $t$ and time $t+1$. The daily fraction of new infections
$z_t$ are based on surveillance work conducted by the Southern Nevada Health
District \citep{ruff2022rapid}, and these estimates are broadly similar to those
in other locations with available data  \citep{ruff2022rapid, nyreinfect2021,
hireinfect2022, wareinfect2022}. Finally, we specify the time-varying evolution
of the inverse reporting ratio as Gaussian with expectation,
\begin{align}
  \label{eq:report-ratio}
\mathbb{E}[a_{\ell,m+1} \mid a_{\ell,m},\ a_{\ell,m-1},\ a_{\ell,m-2}] = 3a_{\ell,m} - 3a_{\ell,m-1} + a_{\ell,m-2}.
\end{align}
This construction for \Cref{eq:report-ratio} results in estimates that vary
smoothly in time.

    
The antibody prevalence model specified by
\Crefrange{eq:sero-measurements}{eq:report-ratio} is a state space model with
latent variables $\mathbf{s}_{\ell}$ and $\mathbf{a}_{\ell}$. Writing it in this
way allows for convenient handling of missing or irregularly-spaced survey
measurements, extrapolation of the estimated latent quantities before and after
the period of observed seroprevalence measurements, and maximum likelihood
estimates of the latent variables and all unknown parameters. Details of
this methodology and the computation of the associated uncertainty measurements
are deferred to Supplementary Methods \Cref{supp:ssapm}.



\subsection{Lagged correlation to hospitalizations and time-varying IHRs} 
\label{sec:ihr-calculations}

From the COVIDcast API \citep{reinhart2021open}, we retrieve the daily number of
confirmed COVID-19 hospital admissions for each state that are collected by the
\US Department of Health and Human Services (HHS). We use our infection
estimates $\mathbf{\widehat{u}}_\ell$ to compute the lagged correlation with 
hospitalizations. The goal of this analysis is to find the lag between
infection and hospitalization rates that gives the highest average rank-based
correlation across \US states. To that end, we consider a wide range of possible
lag values ranging from 1 to 25 days. Zero and negative lags are not considered
because COVID-19 infection onset must precede hospitalization.
To remove day of the week effects, both the infection and hospitalization
signals are averaged over a 7-day, center-aligned, moving window before their
conversion to rates.

For each considered lag, we calculate Spearman's correlation between the state
infection and hospitalization rates for each observed between June 1, 2020 to
November 29, 2021 with a center-aligned rolling window of 61 days. We then
average these correlations across all states and times for each lag. 

The lag that leads to the highest average correlation is used to estimate the
time-varying IHRs for each state. The IHR is computed by dividing the number of
individuals who are hospitalized due to COVID-19 by the estimated total number
who were infected on the lagged number of days before. To stabilize these lagged
IHR estimates, we average these hospitalizations and infections within a window
of 31 days centered on the date of interest, rather than just using one pair of
dates for each computation.
