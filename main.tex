\documentclass{article}
% Packages used
\input{latex-head.tex}
\def\TimesFont{} 
\graphicspath{{gfx/}}

\newcommand{\beginsupplement}{
  \setcounter{table}{0}  
  \renewcommand{\thetable}{S\arabic{table}} 
  \setcounter{figure}{0} 
  \renewcommand{\thefigure}{S\arabic{figure}}
  \setcounter{section}{0} 
  \renewcommand{\thesection}{S\arabic{section}}
}
\font\supptitlefont=cmr12 at 16pt 
\newcommand{\attn }[1]{\textcolor{red}{ATTN: #1}}
     
\begin{document}
\title{Retrospective estimation of latent COVID-19 infections before Omicron in the \US}
\author{Rachel Lobay, Maria Jahja, Ajitesh Srivastava, Ryan J.\ Tibshirani, Daniel J.\ McDonald}
\date{Version: \today}
\maketitle

\begin{abstract}
The true timing and magnitude of COVID-19 infections (rather than reported
cases) are of interest to both the public and to public health, but these are
challenging to pin down for a variety of data-driven and methodological reasons.
Accurate estimates of latent COVID-19 infections can improve our understanding
of the size and scope of the pandemic and provide more meaningful and timely
quantification of disease patterns and burden. In this work, we estimate daily
incident \emph{infections} for each \US state. Rather than taking a model-based
approach, our methods operate directly on data. We first deconvolve reported
COVID-19 cases to their infection date using delay distributions estimated from
the CDC linelist. We combine these deconvolved cases with serology data to scale
up to unreported infections. Our results cover all states at the daily
frequency, incorporate variant-specific incubation periods, and account for
reinfections and waning antigenic immunity. This analysis also produces
estimates for other important quantities such as the number of deconvolved cases
specific to each variant and the infection-case-report ratio. We also discuss some
implications of our results: a disease burden that appears earlier and more
extensively than previously quantified; differential infection-hospitalization
ratio estimates. Our findings help to better understand the impact of the
pandemic in the \US prior to the onset of Omicron and its descendants. 

\end{abstract}

\section{Introduction}

Reported COVID-19 cases are a staple in tracking the pandemic at varying
geographic resolutions \citep{dong2020interactive, nyt2020corona,
wp2020tracking}. Yet, for every case that is eventually reported to public
health, several infections are likely to have occurred, likely much earlier. To
see why, it is important to understand \emph{whose} cases are being reported and
what differentiates them from the unreported cases as well as \emph{when} these
case reports happen. \autoref{fig:chain_events_onset_report} shows an
illustration of the path of a symptomatic infection that \emph{is} eventually
reported to public health. Using this figure, we can discern a number of sources
of bias in the reporting pipeline. For instance, diagnostic testing mainly
targets symptomatic individuals; thus, infected individuals exhibiting little to
no symptoms are omitted \citep{cdc2022estimated}. In addition, testing
practices, availability, and uptake vary temporally and spatially
\citep{pitzer2021impact, ecdc2020strategies, hitchings2021usefulness}. Finally,
cases provide a belated view of the pandemic's progression, because they are
subject to delays due to the viral incubation period, the speed and severity of
symptom onset, laboratory confirmation, test turnaround times, and eventual
submission to public health \citep{pellis2021challenges, wash2020dash}. For
these reasons, reported cases are a lagging indicator of the course of the
pandemic. Furthermore, they do not represent the actual number of new infections
that occur on a given day as indicated by exposure to the pathogen. Since there
was no large-scale surveillance effort in the United States that reliably
tracked symptom onset, let alone infection onset, ascertaining the onset of all
\emph{infections} is challenging.

\begin{figure}[!tb]
\centering
    \includegraphics[width=.99\textwidth]{Chain_of_events_onset_report.pdf} 
    \caption{Idealized chain of events from infection onset to case report date 
    for a symptomatic infection that is eventually reported to public health.}
    \label{fig:chain_events_onset_report}
\end{figure}

% Importantly, all of these issues that are present in local health authority
% data are also present in the gold standard for case data from the JHU CSSE
% \citep{dong2020interactive, guidotti2022worldwide} because JHU scrapes case
% data from the local health authority dashboards \citep{jahja2022real}.
% Furthermore, the cases shown on the JHU CSSE Coronavirus Resource Center
% \citep{jhucsse2020covid} are those that have been disseminated to the public
% on a given day. 
% Our approach to estimate latent infections takes case data and estimates the 
% following...

Explaining the course of the pandemic and investigating the effects of
interventions, the burden facing various subgroups, and drawing insights for
future pandemics is inhibited because the true spatial and temporal behaviour of
infections is unknown. While reported cases provide a convenient proxy of the
disease burden in a population, it is incomplete, delayed, and understates the
true size of the pandemic. Regardless of these difficulties, it is important to
the public and public health to perform a pandemic post-mortem and try to better
explain its implications---to attempt to capture the true size and impact of the
pandemic as much as we can. Estimates of daily incident infections are one such
way to measure this and can guide understanding of the pandemic burden over
space and time.

In this work, we provide a statistically rigorous, data-first reconstruction of
daily incident infections for each \US state from June 1, 2020 to November 29, 2021.
\attn{Should this be 30?}
Using state-level line list data, we construct time-varying delay distributions
for the time from symptom onset to positive specimen date and positive specimen
to case report date. We combine these with variant-specific incubation period
distributions to deconvolve daily reported COVID-19 cases back to their
infection onset. Finally, the resulting deconvolved cases are adjusted to
account for the unreported infections using seroprevalence and reinfection data
to estimate adjust for the waning of antibody detectability over time. We
examine some features of our infection estimates and the implications of using
them rather than reported cases in assessing the impact of the pandemic. We
produce simple time-varying infection-hospitalization ratios (IHRs) for each
state and compare those to similarly derived case-hospitalization ratios (CHRs).
While these analyses provide a glimpse into the utility of our infection
estimates, we believe that there is much more to be explored, and we hope that
our work (and the resulting publicly-available estimates) will prove an
important benchmark for others to undertake retrospective analyses.




\section{Results}
\label{sec:results}

An important aspect of our methods is that deconvolution is not the same as a
shift. \attn{We need a concise description of this somewhere, possibly with a
graphic.}


\subsection{Infection estimates reveal waves missed by reported cases}
\label{sec:omitted-waves}

Outbreaks in infections precede those in reported cases and are reliably larger
in magnitude. But simply shifting cases back in time and increasing them by some
factor fails to capture the spatio-temporal dynamics of the pandemic. 
Hence, relative to reported cases, examining estimated infections reveals a
rather different pattern. \autoref{fig:state_infect_est} shows
estimates of the number of daily new infections per 100,000 inhabitants for each
\US state from June 1, 2020 to November 29, 2021 compared with reported cases,
and deconvolved cases (reported cases ``pushed back'' by the delays shown in
\autoref{fig:chain_events_onset_report}). 

While the major Ancestral, Alpha, and Delta waves tend to be visible for most
states, there are clear outbreaks in unreported infections that are not easily
detectable from cases alone in the falls of 2020 and 2021. For example, a wave 
of infections is present in the spring of 2021 for North Dakota and South Dakota
which is not visible in reported cases alone. \attn{we should search for more of
these patterns.}

\subsection{Spatial-temporal implications are ignored}
\label{sec:ignored-patterns}

\attn{We need to expand the discussion of this figure along the lines we
discussed in our meeting. Choose the dates carefully to emphasize that the
spatial extent during peaks/troughs of waves is different if you look at
infections instead of cases.}

\autoref{fig:choro_inf_case_rates} shows that for the
earliest time of June 1, 2020, there is little discrepancy between case and
infection rates, while for the later times there are immense differences in the
rates, such that case rates tend to underrepresent infections to a great extent.



\subsection{Infection/case ratios vary by state and VOC}
\label{sec:case-infection-ratio}

\attn{The rest of this section uses a lot of space to say, basically, that
infections are bigger than cases. I think we should cut much of it (though see
also below). I think a
better use is to more explicitly emphasize how the case/infection ratio changes
with time and VOC. Let's make that more focused, direct. Note that this isn't
really ``underreporting'', they're still reporting all the tests, but those
tests capture different numbers of infections.}

With respect to variants of concern, consider the late 2020
Ancestral wave for the midwestern states of Illinois, Indiana, and Ohio. For the
major Delta wave, some of the greatest discrepancies between cases and
infections are visible in the western states of Idaho and Montana, the southern
states of Louisiana and Georgia, and the midwestern states of Iowa and Nebraska
(\autoref{fig:state_infect_est}). Earlier on in the pandemic, such discrepancies
between cases and infections may be more attributable to failures in the
reporting pipeline, while later on in the pandemic, they more likely due to the
rise in asymptomatic infections across variants \citep{oph2022covid,
garrett2022high}. 

Finally, while the main Delta wave is somewhat evident from the case counts for
all states (\autoref{fig:state_infect_est}), our estimates suggest that case
counts tend to severely underestimate infections during this time for many
states. The lowest of all states was in New Jersey, where about $4.6\%$ (95\%
confidence interval: $[1.9, 67.7]$) of the estimated infections were reported.
This was followed by Maryland with $7.4\%$ ($[2.7, 83.8]$), Connecticut with
$8.0\%$ ($[3.1, 25.8]$), and Florida with $8.7\%$ ($[4.8, 34.0]$). This
underreporting issue extends to most states as in $39$ states less than $30\%$
of infections were reported during this time. Only $4$ states of Alaska, Maine,
Vermont and Virginia reported at least $40\%$ infections. No states were found
to surpass $50\%$ for reported infections for this time.

Similar patterns were observed during the earlier period of Alpha domination,
where Louisiana had the lowest reported infections at $11.7\%$ (95\% confidence
interval: $[6.7, 31.5]$) and was followed by California at $14.4\%$ (95\%
confidence interval: $[7.7, 68.2]$). There were $23$ states that reported at
least $40\%$ and $22$ states that reported at least $50\%$ of their infections.

Such patterns were comparatively less apparent during the earlier and larger
period of Ancestral domination, where Ohio and Maryland held the lowest
percentages of reported infections at $22.0\%$ (95\% confidence interval:
$[16.2, 34.0]$) and $22.3\%$ (95\% confidence interval: $[14.8, 40.5]$),
respectively. During this time, $28$ states that reported at least $40\%$ and
$14$ states that reported at least $50\%$ of their infections. 

\begin{figure}[!tb]
    \centering
        \includegraphics[width=.99\linewidth]{state_niauc_est_faceted_F24.pdf} 
        \caption{Estimates of the number of daily new infections per 100,000
            population for each \US state from June 1, 2020 to November 29, 2021
            (dark blue line). The blue shaded regions depict the 50, 80, and 95\%
            confidence intervals for the estimates, while the teal line represents
            the number of new daily new deconvolved cases per 100,000, and the
            dotted orange line represents the 7-day average of the new cases per
            100,000 as of the same date.}
        \label{fig:state_infect_est}
    \end{figure}
    

\begin{figure}[!tb]
\centering
    \includegraphics[width=.99\textwidth]{choro_inf_case_rates_F24.pdf}
    \caption{Choropleth maps of the state-level estimates of the number of daily
    new infections per $100,000$ population (top row) and the daily new cases
    per $100,000$ population (bottom row) for three times over the June 1, 2020
    to November 29, 2021 period. The first date was chosen simply as a baseline,
    while the second and third dates were chosen based on the day that had the
    largest number of infections across the 50 states from each year.} 
    \label{fig:choro_inf_case_rates}
\end{figure}    



    
\subsection{Infections broken down by VOCs emphasize earlier outbreaks}
\label{sec:infections-by-voc}

\autoref{fig:six-states} examines the infection estimates for a selection of
states more closely. This set has the largest infection/case ratios. 
\attn{We need to say more about what these 2 figures show, uniquely. What's the
point of looking at them? I think the section heading I wrote is the point. But
we need to say this. The takeaways below are a bit too vague, I think.} The top
panel shows \attn{....} The bottom panel divides estimated
infections into buckets based on the circulating variant proportions at the
time. From these plots, it is clear that few variant categories tends to
dominate and drive infections at a time. The general progression in terms of
variant starts with the Ancestral category from 2020 up to early 2021, to the
Alpha variant in mid 2021, which eventually gets eclipsed by the Delta variant
in mid to late 2021. This supports our division of our results by the three main
variant-driven time periods.


\begin{figure}[!tb]
\centering
    \includegraphics[width=\linewidth]{state_niauc_est_6states_F24.pdf}\\
    \includegraphics[width=\linewidth]{state_decon_byvar_est_6states_F24.pdf}
    \caption{Top panel: Reported cases, deconvolved cases, and estimates of daily new infections (dark blue
    line) per 100K inhabitants. The blue shaded regions indicate the 50, 80, and 95\% confidence
    bands, while the background is shaded to indicate the dominant variant in
    circulation at the time. \attn{I don't like the rotated x-axis.} 
    Bottom panel: Deconvolved cases colored by variant per 100K inhabitants. \attn{Is
    this the correct units? or are these raw? They should be per 100K.}}
    \label{fig:six-states}
\end{figure}


\subsection{The relationship between infections and hospitalizations is messy}
\label{sec:lagged-correlations}

We systematically investigate the temporal relationship between infections and
hospitalizations with Spearman's rank-correlation across different lags,
shifting hospitalizations backward to align with infections.
(\autoref{fig:correlations}). The maximum average correlation
across states is 0.513, occurring at a lag of 13 days. In contrast,
we find that the greatest average Spearman correlation for cases is 0.691 and
occurs at a lag of 1 day. That is, we find that case report rates are nearly
contemporaneous to hospitalizations, while infection estimates clearly precede
them. 

The maximum correlation at a lag of 13 days is in similar to early estimates of the average time from
infection to hospitalization of 9.7 days (95\% CI: $[5.4, 17.0]$) for cases
reported in January, 2020 in Wuhan, China as well as with estimates from across
the pandemic in the UK that ranged from an average of 8.0 to 9.7 days
\citep{ward2021understanding}. 
% However, we should note the first study is based
% on a small sample size for outbreak cases reported well before our study start
% date. As well, both sets of estimates depend upon the healthcare system and the
% population structure, amongst other things \citep{ward2021understanding}.
% Nevertheless, their relative agreement with our estimate of 13 days for the \US
% states lends some credence to of our results. 

\begin{figure}[!tb]
\centering
\includegraphics[width=.45\textwidth]{infect_case_hosp_lag_corr_F24.pdf} 
\includegraphics[width=.45\textwidth]{adj_unadj_pi_no_inc_hosp_lag_corr_F24.pdf} 
\caption{Spearman's correlation between the different case/infection ratesand
hospitalization rates per 100,000. These are calculated for each
lag, state and rolling window of 61 days before averaging. 
The vertical dashed lines indicate the lags
for which the highest average correlation is attained. \attn{Let's do one panel
with cases, deconvolved cases, and infections.}}
\label{fig:correlations}
\end{figure}
    

% In terms of the average correlation produced, the deconvolved case estimates by
% infection onset and the deconvolved case estimates by positive specimen date
% reach almost the same maximum average correlation. While that is not a clear
% differentiator by itself, there is a clear time-based benefit of opting for the
% infection estimates by the date of infection onset over symptom onset because
% they provide similar information on hospitalizations about 6 days before the
% latter tends to occur.

Unsurprisingly, the deconvolved case and infection estimates achieve their
maximum correlation at the same lag \attn{I don't think so...}. And yet, the average correlation to
hospitalizations tends to be greater for the deconvolved case estimates than for
the infection estimates. This finding may
stem from a difference in disease severity between the reported and unreported
infections: unreported infections tend to be less severe and less likely to
lead to hospitalization than those that are reported.




\subsection{Estimating infection-hospitalization ratios}
\label{sec:ihrs}

As a counterpart to the correlation analysis, we compute the time-varying
infection-hospitalization ratios (IHRs) for each state using the correlation
maximizing lag. We similarly compute the
case-hospitalization ratios (CHRs) using their correlation maximizing lag for
for comparison (\autoref{fig:IHR_7dav}). 

For each state, the CHRs tend to be larger and noiser relative to
IHRs. This supports our claim that the reported infections are more
likely to require hospitalization than the unreported infections. Both the IHRs
and CHRs exhibit similar geospatial and temporal trends as are noted for
infections. Namely, states that are close in proximity (such as Ohio,
Pennsylvania, and Virginia) tend to exhibit similar patterns in the IHRs and
CHRs over time. In addition, there are similar spikes observed across many
states during waves of infections that are driven by prominent new variants. For
example, many states exhibit a striking spike in hospitalizations in mid-2021,
which coincides with the rapid takeover of the Delta variant during that time
\citep{hodcroft2021covariants}. This finding aligns with previous studies that
found an increased risk in hospitalizations with Delta in comparison to other
variants \citep{twohig2022hospital, nyberg2022comparative}. Similarly, during
the fall of 2020 there tends to be another spike in the IHRs that rivals or
surpasses that observed during the time of Delta (which is the case for states
like New York or Wyoming). 

There does not tend to be a strict upward or downward trajectory or even a
mild waning pattern in the IHRs, as one might expect with later variants that are
more infectious but result in fewer hospitalization
\citep{lorenzo2022covid, blauer2022compare}. Overall, we observe intermittent
spikes that punctuate longer periods where the IHRs tend to stablize slightly below 0.2
hospitalizations per infection. These spikes tend to align with the emergence of
new variants. \attn{0.2 seems wrong. Is the IHR actually H/I? or are the units
not quite right (e.g., H/1M / I/100K)?}

While we computed and compared CHRs and IHRs for all states, it is important to
note that both likely to vary within states and depend on confounding variables
such as age and the presence of major comorbidities
\citep{russell2023comorbidities}. Therefore, it would be beneficial to account
for such variables in their calculations by, for example, stratifying infections
and hospitalizations by age to produce age-specific estimates of the IHRs for
each state~\citep{fox2023disproportionate}.



\begin{figure}[!tb]
\centering
\includegraphics[width=.99\linewidth]{IHR_7dav_F24.pdf}
\caption{Time-varying IHR and CHR estimates for each state from June 1, 2020
to November 29, 2021, obtained using the corresponding optimal lag from the
systematic lag analysis. Note that the infection, case, and hospitalization
counts are subject to a center-aligned 7-day average to remove spurious day
of the week effects. Also note that the different starting points across
states are due to the availability of the hospitalization data.}
\label{fig:IHR_7dav}
\end{figure}


\subsection{Disease burden and viral transmission}

\attn{I'm not yet sure what to do with this. Some of it should go into one of 
the sections above.}

From reconstructing the time series of COVID-19 infections per $100,000$
population for each \US state from June 1, 2020 to November 29, 2021, we
observe rates of infections that vary in intensity and disease burden across
space and time (\autoref{fig:state_infect_est}, \autoref{fig:six-states}).  
Most states present at least two major spikes in infections - the first starts
in the fall of 2020 and extends into the winter season, while the second starts
in the late summer of 2021 and proceeds into the mid-fall. These represent major
waves driven by the Ancestral and Delta variants. Similar patterns in the major
surges of infections are observed in nearly all states, though to varying
degrees. In general, greater similarities in the strength and magnitude of
outbreaks are found to emerge in the clusters of states that border each other.

To avoid encroaching upon possible boundary issues with ending the estimation
during a time of volatility (the period of the Delta-Omicron transition), we
focus on the infection estimates prior to November 1, 2021. The largest observed
outbreaks prior to this time were observed in the late summer or early fall of
2021 in Georgia, Louisiana, Idaho, Montana, and Wyoming which suggests a similar
spread of the virus in small clusters of states that are in close geographic
proximity. During this time, the two states that have the attain the highest
rate of infections per 100,000 on single day are Georgia with about $451$
infections per 100,000 on August 15, 2021 (95\% confidence interval: $[334,
567]$) and Idaho with $451$ on September 7, 2021 (95\% confidence interval:
$[312, 590]$). These are closely followed by Montana with $432$ on September 8,
2021 (95\% confidence interval: $[282, 581]$), Louisiana with $431$ on July 20,
2021 (95\% confidence interval: $[252, 610]$), and Wyoming with $350$ on
November 13, 2020 (95\% confidence interval: $[256, 444]$).

Prior to the Delta wave, the state that has the highest rate of infections per
100,000 on single day is Louisiana with about 358 infections per 100,000 on July
3, 2021 (95\% confidence interval: $[177, 539]$), followed by Wyoming with 349
on November 13, 2020 (95\% confidence interval: $[407, 546]$), South Dakota with
342 infections per 100,000 on July 3, 2021 (95\% confidence interval:$[177,
539]$), and Illinois with 340 infections per 100,000 on July 3, 2021 (95\%
confidence interval: $[177, 539]$). During this time, 74\% of the top rates for
each state were observed in the late fall or winter of 2020.

The period of lowest viral transmission is observed in the summer and fall of
2020. During this time, the state of New Hampshire achieves the lowest weekly
rate of infections of 0.01 infections per 100,000 for the week of September 13,
2020. In the summer of 2020, Vermont maintains a rate under 10 infections per
100,000 from the week of June 1, 2020 to August 30, 2020, which is the longest
continuous stretch observed for any state.

From a brief inspection of the geo-contiguous states, we can observe similar
patterns in surges and periods of waning over time, suggesting that states who
share similarities in climate and topography performed similarly to each other.
More precisely, we can observe neighboring states such as New Hampshire and
Massachusetts or Idaho and Montana that present waves that mirror each other in
amplitude and timing. 

Interestingly, the two states that are geographically removed from the
contiguous United States, Alaska and Hawaii, tend to perform quite differently
from each other later in the pandemic. Alaska generally presents significantly
greater rates of infections than Hawaii especially during the Delta era. This
suggests that it is not so much the non-contiguity aspect as it is other
distinguishing factors that lead to lower infection rates.


%\subsection{Sensitivity analysis}
% This section is under construction
%The infection estimates exhibit modest changes under different assumptions about the
%variant-specific incubation periods, the construction of the delay distribution (the
%window size for the considered onset dates), the fraction of new infections over time, and
%the population estimates (see Supplementary Materials Section X). 
% Potentially compare ww a_t ratios over the time period as well (ie. do they fall in the confidence bands for a_t estimates
% from the ss model)?
% \attn {Add a sentence or two about what is meant by modest changes... Did any of these result in noticeably higher or lower (biased) estimates of infections? To what extent (ie. an additional X infections)? For what states?}
% Do these sensitivity analyses + update this link accordingly.

\section{Discussion}

We retrospectively estimated daily incident infections for each \US state over
the period June 1, 2020 to November 29, 2021. Our  estimates suggest both (a)
that the pandemic impacted states earlier and at a larger scale than is
indicated by cases and that (b) examining cases alone hides some spatio-temporal
waves that become apparent by examining infections. We observe outbreaks in
infections that are difficult to detect from cases alone such as the Delta wave
in New Jersey, Connecticut, and Maryland. This suggests that cases paint an
incomplete picture of the pandemic, especially when outbreaks are largely driven
by unreported infections. Furthermore, since case reports generally follow
symptom and infection onsets, cases have a built-in temporal bias. This is in addition
to other biases from differences in reporting across states (such as temporary
bottlenecks due influxes of data or more persistent processing issues that
increase the average time from case detection to report \citep{wash2020dash,
dunkel2020covid19}. 
% Furthermore, no indication of uncertainty is provided for
% even the gold standard case estimates \citep{delphiepidata2020}. 
Thus, while reported cases provide an indication of the trajectory of the
pandemic, it is a delayed and incomplete version.

% Since case reporting is not
% consistent across time and states, case counts underestimate the true number of
% infections and, hence, the impact of the pandemic \citep{cdc2022estimated,
% simon2022inconsistent}. For example, some states report the number of
% individuals tested rather than the numbers of tests performed
% \citep{schechtman2020counting, chitwood2022reconstructing}. Additionally, while
% the definition of a confirmed COVID-19 case tends to be fairly uniform across
% the United States due to general adherence to the CSTE case definitions, state
% reporting standards have been known to vary \citep{cste2020, delphiepidata2020}.
% For instance, there may be inconsistencies across locations if some cases are
% labelled as confirmed based on positive antigen tests instead of PCR tests
% \citep{covidtracking2021}. 
% As well, the definition of a case and related terminology can change
% or evolve over time as more information becomes available.  

% Estimating the new number of infections by symptom or infection onset date would
% more closely align with the definition of incidence as we know it
% \citep{jahja2022real}.



% The remainder of our discussion consists of an in-depth look into the advantages
% and limitations of our approach and of other comparable approaches, followed by
% a high level summary of our work and its major contributions. 

Our approach offers a number of advantages.
% The development 
% of a way of modelling immunity and space-time-specific reporting ratios based on 
% seroprevalence data.
% Similar phrase on line 1268, though we may want to emphasize that point up here.
% To the best of our knowledge, no other modelling approach has been used to 
% reconstruct the infection time series
%for every state over as much of the COVID-19 pandemic as in this study. %%
%Furthermore, 
For instance, we aim to incorporate as much state-specific information as
possible when deriving our estimates. By using state-level case, line list, and
variant circulation data, we are able to construct incubation and delay
distributions that are specific to each state. Time-varying and state-specific
seroprevalence data allows the reporting ratio estimates to similarly vary over
space and time, a departure from existing work \citep{unwin2020state,
uga2020covid19}. Existing approaches that use the delay distribution to generate
infection estimates often only construct one delay distribution that is used for
all states \citep{chitwood2022reconstructing, jahja2022real}. That is, our work
avoids the assumption of geographic invariance, where it is assumed that
all states have the same patterns of delay from symptom onset to case report.
This assumption is
unlikely to be true due to differences in reporting pipelines, pandemic
response, and variants in circulation, among other issues. 

Another limitation of previous approaches to estimate latent infections is
that they do not to account for reinfections. While
reinfections represent a small fraction of total infections until
later in the pandemic, ignoring them means that the
infection-reporting ratio will tend to be underestimated with seroprevalence
data alone. By accounting for these as well as the waning of seropositivity (See
Methods \autoref{sec:waning-immunity}), we more accurately estimate this ratio.
% so they are not absolutely necessary to include in the
% earlier stages of the pandemic. Still, at no stage did infection confer lifelong
% immunity. Rather antibody levels and immunity are known to wane over time. And
% we believe it is important to account for such defining characteristics of the
% virus when tracking infections over time. Therefore, we account for reinfections
% and the waning of detectable antibody levels in our custom antibody prevalence
% model. 
However, we acknowledge that the extent to which each of these are
accounted for could be improved upon in future work. 
Since the waning of immunity is likely to be variant-dependent
\citep{pooley2023durability}, it follows that our model waning parameter may be
better posed as a mixture of parameters for different variants with weights
determined by the proportion of the variants circulating at the time in the
state. Related to this is the issue that newer variants may escape detection
\citep{nih2022assessing, fda2023sars}. While in a retrospective analysis where
finalized data is used this is less likely to be an issue, this could very well
pose a problem for real-time estimates of infections.

Regarding reinfections, a major reason why we chose an end date of November 29,
2021 and ultimately decided to not tread into Omicron territory is because the
Omicron variants come with substantial increase in the risk of reinfection in
comparison to previous variants as Omicron has been shown to have an increased
tendency towards immune escape \citep{wei2024risk, pulliam2022increased,
eythorsson2022rate}. So having quality reinfection data that is representative
of each location under study is of the utmost importance for the Omicron era. 

% While it would be ideal to use confirmed rates over time for each \US state,
% most states do not publicly report reinfection data over the entire time period
% we considered. So we have turned to suspected reinfection data over time for
% Clark County, USA, as that surveillance is among the most detailed and reputable
% that we have found for the United States. Nevertheless, using such localized
% data raises questions of representativeness and the applicability of such
% estimates to Nevada and all other states. Furthermore, this data has no
% information available beyond suspected third infections, which imposes an
% irremediable bias. However, based on the third infection data available there,
% we expect that the probability of being reinfected more than three times is
% likely very low for time frame considered and so the omission of these would
% impact our infection estimates to a minimal extent. 

% The vast majority of issues we encountered when trying to reconstruct the
% infection time series for each state are due to an absence or a lack of data.
% Such is the primary issue we had with the restricted line list. In comparison to
% the number of JHU cases (which we are treating as a gold standard) for the same
% release date, we noted there are about $10$ million cases that are unaccounted
% for in the CDC line list. Moreover, the missingness does not appear to be random
% and uniformly distributed across states. Rather it is unequally distributed,
% suggesting that the dataset is likely biased. However, more information on the
% cases that are missing versus present would be required to determine the extent
% the missing cases led to a nonrepresentative, and therefore, biased sample, and
% could be a topic of further study.

Using seroprevalence data to estimate the case-ascertainment ratio is subject to
a number of issues, and precludes us from pushing the period of analysis past
the Omicron wave in December 2021. While most state-level data suggests that
reinfections still account for less than 20\% of reported cases during Omicron
\attn{cites}, seropositivity rapidly reaches nearly 100\% of the population,
precluding its continued use. Due to these issues,
alternative data sources for
estimating the case-ascertainment ratio is necessary. 
% Intuitively, one might expect that
% leveraging data from multiple sources would likely lead to more accurate and
% stable estimates than those from using one source. 
For example, wastewater surveillance data
is may be complementary to seroprevalence data,
especially when testing is low \citep{mcmanus2023predicting}. However, 
% there has
% been limited success in predicting incidence using such data. The extent that
% wastewater concentration data is a useful in estimating COVID-19 incidence is
% unclear owing to problems with viral occurrence and detectability in wastewater
% that render 
viral detection is inconsistent across locations due to temperature,
per-capita water use, and in-sewer travel time \citep{mcmanus2023predicting,
hart2020computational, li2023correlation}. Sentinel surveillance streams for
influenza-like illness or acute respiratory infection may provide decent proxies
for COVID-19 incidence, especially when testing for mild cases of COVID-19 is
diminishing or has ceased completely. Finally, alternative surveillance streams
(potentially outside of public health) such as those from surveys, helplines, or
medical records could potentially be integrated if they provide at least a rough
indication of the disease intensity over time
\citep{reinhart2021open,ecdc2020strategies}.



% \attn{Some into Methods, some into Supplement}
% also runs the risk of being nonrepresentative of the
% intended population \citep{bajema2021estimated}. For example, in the blood donor
% dataset some states have region specific-estimates, which clearly do not stand
% for the entire state. Another source of systematic variation is in the
% characteristics of the individuals who opt for blood tests versus those who do
% not. For instance, there may be a healthy user bias, in which a number of those
% who opt for blood tests are generally more inclined to partake in proactive
% healthy behaviors (such as checking on basic health markers by taking an annual
% blood test) than those who do not \citep{parsley2018blood}. Alternatively, a
% number of individuals may be recommended for blood tests by their doctors due to
% signs of ill-health (ex. mineral deficiencies or underlying medical conditions).
% The extent that each such bias persists depends on the purpose of the blood test
% and whether it was used as a proactive or reactive medical tool. Since such
% information is unavailable to us, all we can conclude is that participant-driven
% sources of bias impact the seroprevalence samples to an undetermined extent.
% There are additional concerns about the performance of antibody testing for
% individuals with mild or asymptomatic disease as well as about the loss of
% immunity over time \citep{kaku2021performance, seow2020longitudinal,
% ibarrondo2020rapid}.
% 
% In this work, we do not attempt to directly address infection underascertainment
% due to the increase in asymptomatic infections across variants
% \citep{pho2023covid19}. We simply note that this would likely pose a greater
% problem later in the pandemic, particularly after the Delta era
% \citep{fan2022sars}. We hope that such infections would be largely represented
% by the seroprevalence and reinfection estimates, but there is undoubtedly
% increasing reliance on such estimates to be able to do this over time (owing to
% the simultaneous decline in the reporting cadence and the apparent rise in
% asymptomatic infections over time) \citep{oph2022covid, garrett2022high,
% blauer2022reduce, ren2021asymptomatic}. Consequently, there is an increasing
% uncertainty over time that is not captured by the model or the estimates.   


We adopt a relatively simple deconvolution-based approach and devote
much of our efforts to tailoring our approach to the available data. A major
result of this is the development of a way of to model the waning of detectable
antibody levels and space-time-specific reporting ratios based on seroprevalence
data. In a way, our approach is built for the data rather than trying to force
the data to fit to an existing approach. However, our model is only as good as
the quality and the quantity of the data provided to it. In our case, the lack
of data is both a barrier to entry and a continual roadblock. The assumptions we
are required to make as a consequence of this clearly limit the generalizability
and call into question the reliability of the results. So while we highlight
some interesting trends and numerical findings, these results are not
definitive, but rather exploratory and intended to stimulate discussion on the
challenging task of estimating infections. Despite these limitations, we are
encouraged by the ability to use routine data to produce sensible estimates of
infections in the United States and the plausibility of the apparent geospatial
and temporal trends. 
 


% Our approach is predicated upon having case, line list, viral circulation, and
% seroprevalence data for each state, all of which are readily available (or
% available upon request in the case of restricted line list data). As a result of
% this, we are able to demonstrate the feasibility of estimating COVID-19
% infections at the state level by using standard sources of data. 

% Our framework is quite versatile as it lends itself to more localized, county or
% community level estimates, or globalized, country-specific estimates.
% Fundamentally, to produce estimates of infections for different geographic
% regions, one would simply need to input the required data and re-run the code
% pipeline. In this way, one could readily adapt our approach to generate
% estimates for the provinces in Canada or the regions in England.

Well-informed, localized estimates of COVID-19 infections over time can help us
to have a more clear and comprehensive understanding of the course of the
pandemic. Such estimates contribute important information on the timing and
magnitude of disease burden for each location and they highlight trends that may
not be visible from case data alone. Therefore, our infection estimates provide
key information for the ongoing debate on the true size and impact of the
pandemic.

\section{Methods}


In what follows, we provide details on how we estimate the daily incident
infections for each state over the considered time period of June 1, 2020 to
November 29, 2021 and the data we used to achieve this. 
% We start with a brief
% introduction to each data source used and follow this with a description of each
% major analysis task in the order they are performed.
\autoref{fig:cases_to_infect_flowchart} provides a visual summary of the data,
analysis tasks, and the relationships between them. The major analysis tasks
this figure aims to convey are as follows: First, we estimate variant-specific
incubation periods and two types of delay distributions for each day over the
considered time period. Next, each incubation period and symptom onset to
positive specimen delay distribution are joined using convolution to obtain
variant-specific infection onset to positive specimen distributions for each
time. Then two types of deconvolution are performed. We first deconvolve from
case report to positive specimen date. We then deconvolve from positive specimen
to report date by variant. The resulting infection estimates are aggregated
across the variant categories, and adjusted to account for the unreported
infections by using state-specific, time-varying seroprevalence data in an
antibody prevalence model. This lets us reach our ultimate goal of obtaining
daily incident infection estimates.
% Then, we can apply these estimates in a
% lagged correlation to hospitalizations to find the ``best'' lag between
% infection and hospitalization rates according to Spearman's rank-based correlation 
% and use this to compute time-varying IHRs for each state. 


\begin{figure}[!tb]
\centering
    \includegraphics[width=.99\textwidth]{Reported_cases_to_infect_flowchart.pdf} 
    \caption{Flowchart of the inputted data and major analysis steps required 
    to get from reported cases to incident infection estimates for each day 
    over June 1, 2020 to November 29, 2021 for a state. Data sources are coloured 
    in yellow, while data analysis steps are coloured in blue. The data sources that
    do not stem from an analysis step are literature estimates.}
    \label{fig:cases_to_infect_flowchart}
\end{figure}


% Reasons why we should start from June 1, 2020 
% A list of reasons on why I settled on these start and end dates: 
% 1. Reinfections officially started occurring 2020-06-01 according to the data we?re using
% 2. sero measurements start in July 2020 for most states or later
% 3. the prior values for the inverse of the ascertainment ratio estimates are as of 2020-06-01
% 4.  the earliest hospitalizations for each state (that we use in the correlation analysis) 
% are also almost always in July, 2020. 
% So I am not convinced we should extrapolate multiple months away from the start or end for sero. 
% One month in either direction seems reasonable & doesn't overreach.







\subsection{Estimating delay distributions from private line lists}
\label{sec:delaystop}

We obtain de-identified patient-level line list data on COVID-19 cases from the
CDC. Although there are both public and restricted versions of the dataset
available containing the same patient records \citep{cdc2020casepub,
cdc2020caserestr}, only the restricted dataset contains information on the state
of residence. The three key dates of interest are those for symptom onset,
positive specimen collection, and report to the CDC. Handling missingness and
imputation in these dates is somewhat complicated, and additional details and
justifications are deferred to \autoref{sec:linelist-details}.

We use the line list to estimate the delay distribution for the pairs symptom
onset to positive specimen and positive specimen to report. We provide the full
procedure for the latter, before giving a brief description below for the
former. First, define $z_{\ell,t}$ to be a case report occurring at time $t$ in
location $\ell$, and let $\pi_{\ell,t}(k)$ to be the probability that
$z_{\ell,t}$ has a positive specimen collected $k$ days earlier. We assume that
all positive specimens will be reported within 60 days and that no test will be
reported on the same date as it was collected, that is, $\pi_{\ell,t}(0) = 0$
and $\pi_{\ell,t}(k) = 0$ whenever $k > 60$. Let $N_{\ell,t}$ be the number of
$z_{\ell,s}$ with $s\in[t-75+1,t+60] = \mathcal{S}_t$ and positive specimen date
greater than $s-60$. Then, we first compute
\begin{align*}
    \tilde{p}_{\ell,t}(k) = \frac{1}{N_{\ell,t}}\sum_{s \in \mathcal{S}_t}
    \big(\textrm{\# $z_{\ell,s}$ with positive specimen at $s-k$}\big).
\end{align*}
Next we compute a similar national quantity $\tilde{p}_{t}(k) =
\frac{1}{N_{t}}\sum_{s \in \mathcal{S}_t} \big(\textrm{\# $z_{s}$ with positive
specimen at $s-k$}\big)$, without restricting to location $\ell$. Next, let
$\alpha_{\ell,t}$ be the ratio of $N_{\ell,t}$ to the number of cases reported
by JHU CSSE\cite{dong2020interactive} in the same window. Then, compute
$p_{\ell,t}(k) = \alpha_{\ell,t}\tilde{p}_{\ell,t}(k) +
(1-\alpha_{\ell,t})\tilde{p}_t(k)$. This construction allows for more reliance
on the state estimate when there are more CDC cases relative to JHU (and vice
versa). We calculate the mean $m_{\ell,t}$ and variance $v_{\ell,t}$ of
$\{p_{\ell,t}(k) : 0<k\leq 60\}$ and estimate a gamma distribution by solving
the moment equations $m_{\ell,t} = \alpha_{\ell,t}\theta_{\ell,t}$ and
$v_{\ell,t}= \alpha_{\ell,t}\theta_{\ell,t}^2$ for the shape $\alpha_{\ell,t}$
and scale $\theta_{\ell,t}$. Finally, we discretize the resulting gamma density
to the support set of 1 to 60 days to produce an estimate
$\{\widehat{\pi}_{\ell,t}(k): 0 < k \leq 60\}$ of the delay distribution
$\pi_{\ell,t}$.
 
Estimating the delay from symptom onset to positive specimen date follows the
same procedure with a few minor adjustments. First, we allow $k$ to range from
$-3$ to $21$ (rather than 1 to 60). These upper and lower bounds are based on
the largest delay values for the state-wide 0.05 and 0.95 quantiles. This is
reasonable because the median delay is very short at approximately 2 days, and
an asymptomatic individual may test positive following a known exposure, before
the onset of symptoms. Additional minor details are discussed in
\autoref{sec:delay-justifications}.



\subsection{Estimating the incubation period distributions} 

\attn{DJM: I stopped here, for now.}

One incubation period distribution is estimated for each variant under
consideration. The variants we consider are Alpha, Beta, Gamma, and Delta, which
are included because they are are designated to be variants of concern by WHO
based on their potential to cause new waves, dethrone the dominant variant, and
lead to changes in public health policy \citep{who2021tracking}. In addition, we
include the Epsilon (California) and Iota (New York) variants because of their
impact on those and the surrounding states \citep{yang2022investigation,
duerr2021dominance}. We relegate all other variants to be in the Other category
(which, for our purposes, is treated as a catch-all for all 2020 Ancestral
variants observed in the U.S.) This decision to include an Other category is, in
part, motivated by the lack of sequencing data for most states in 2020 as well
as the presence of an Others category in the sequencing data for that time. 

We obtain literature estimates of the variant-specific incubation periods for
the following eight variant categories: Ancestral, Alpha, Beta, Epsilon, Iota,
Gamma, Delta, and Omicron. For the Ancestral variants, we use the literature
estimates of the gamma distribution parameters \citep{tindale2020evidence}. For
the Alpha, Beta, Gamma, Delta and Omicron variants, we use the mean and standard
deviation of the number of days of incubation as reported in
\citet{tanaka2022shorter, grant2022impact, ogata2022shorter}.
Since the literature lacks reliable estimates for the incubation period of the
Epsilon and Iota variants, we use the incubation period for Beta because
Epsilon, Iota, and Beta are all children from the same parent in the
phylogenetic tree of the Nextstrain Clades (as depicted in
\citealp{hodcroft2021covariants}). 

We construct the incubation period distributions for the variants as gamma
distributions. These distributions are the same for all states and based on
literature estimates of the gamma parameters or the mean and standard deviation
of the incubation period (in which case the method of moments is used to fit a
gamma density). Then, we discretize each resulting density to the support set,
which is taken to be from 1 and 21 days. In other words, those are taken to be
the lower and upper limits for the number of days that the virus could be
incubating in someone. The implicit assumption for the lower bound is that there
must be at least one day between infection and symptom onset (which follows the
convention given in \citealp{phcan2021covid}). The assumption underlying the
upper bound is that 21 days is the maximum number of days that the virus could
be incubating in someone (which is reasonable based on
\citealp{zaki2021estimations} and \citealp{cortes2022sars}).

\subsection{Variant circulation proportions}

To estimate the daily proportions of the variants circulating in each state, we
obtain the GISAID genomic sequencing data counts from CoVariants.org
\citep{hodcroft2021covariants, elbe2017data}. Since these counts are biweekly totals, we
apply multinomial logistic regression using a third-order polynomial in time to
get estimates of the daily proportions for the eight variant categories
separately for each state.

\attn{write down the model and define notation. Then show the result for one
state: show the original proportions on the left, and the smoothed on the
right.}


\subsection{Convolutional estimates of the infection to positive specimen distributions} 

The previous two steps enabled us to estimate one incubation period per variant
and one symptom onset to positive specimen distribution for each state at each
time under consideration. We proceed to convolve each such pair of distributions
to get estimated infection to positive specimen distributions and, hence,
estimated time-varying probabilities for the delay from infection onset to
positive test specimen date for each state.

\subsection{Retrospective deconvolution}

The main goal for the retrospective deconvolution stage is to estimate the daily
number of new infections for each time using the dates that those cases were
eventually reported. To this end, there are two types of deconvolution
performed. The first is the deconvolution from report to positive specimen date
and the second is from positive specimen date to infection onset date. We
allocated the deconvolutions in this way to allow us to get the daily
deconvolved case estimates by variant. The intermediate of positive specimen
date was chosen because the variant proportion estimates are aligned to this
date. So for each state at each time, nine deconvolutions are performed in
total. The first is the deconvolution from report to positive specimen date,
followed by the eight deconvolutions from positive specimen to infection onset
for the eight different variant categories.

We will start by describing the first type of deconvolution performed from
report to positive specimen date in detail and then describe the second type in
terms of the changes made with respect to the first. For each state, we achieve
the first goal to estimate positive specimen dates for the cases by solving an
optimization problem. 
% Following the notation of \citet{jahja2022real} (save for suppressing the
% state/location subscript), 
For this problem, let $\cT$ represent the extended deconvolution period from
March 1, 2020 to March 1, 2023, which was used to minimize the effect of
boundary issues and to produce sufficient deconvolved case estimates for further
analysis. Let $\hat{p}_t$ be probabilities from the estimated positive specimen
to report date distribution for $t \in \cT$, $y_t$ the number of new cases
reported, and where $D^{(4)}$ is the discrete derivative matrix of order 4 such
that $D^{(4)}x$ yields all $4\th$-order differences of the vector $x$. From
these, we estimate the deconvolved case counts across time by solving for the
vector $x$ in
\begin{align*}
\minimize_{x}\ \sum_{t \in \cT} \left ( y_t -  \sum_{k = 1}^{d} \hat{p}_t(k)x_{t-k} 
\right )^2 + \lambda \|D^{(4)}x\|_1. 
\end{align*}

The above loss function decouples into two parts which trade data fidelity with
desired smoothness (that
encapsulate the classic bias-variance trade off). The first part represents
minimizing the sum of squared errors between the JHU reported cases and the
estimates, while the second part captures the smoothness of the estimates
(smaller values being more smooth). The tuning parameter $\lambda$ determines
the relative importance of these competing goals.

We solve this trend-filtering-regularized least squares deconvolution problem by
employing the ADMM algorithm from \citet{ramdas2016fast} that is described in
Appendix A of \citet{jahja2022real}. The solution to the problem is an adaptive
piecewise cubic polynomial \citep{tibshirani2014adaptive,
tibshirani2022divided}.

We select the tuning parameter, $\lambda$, by using $3$-fold cross validation
similar to \citet{jahja2022real} in which every third infection count is
reserved for testing. The tuning parameter that results in the smallest mean
squared error is selected.

The COVIDcast API \citep{reinhart2021open} is used to retrieve the daily number
of new confirmed COVID-19 cases for each state that are based on reports from
the John Hopkins Center for Systems Science and Engineering (JHU
CSSE)\citealp{dong2020interactive}. From the same API, we also retrieve the
daily number of confirmed COVID-19 hospital admissions for each state that are
collected by the \US Department of Health and Human Services (HHS). Both
datasets are updated as of June 6, 2022.

From this first type of deconvolution, we obtain case estimates by positive
specimen date for each state. The second type of deconvolution, where the goal
is to get estimates of the infection onset date for these cases, follows the
form of first type, save for two key modifications to the inputs. Firstly, we
utilize the results from the first deconvolution and, secondly, we must update
the probabilities to be the convolutional estimates of the infection to positive
specimen distributions. Thus, for a fixed variant category, $y_t$ is the number
of new cases deconvolved to positive specimen date multiplied by the estimated
proportion of the variant in circulation in the state at $t$, and $\hat{p}_t$
are the probabilities from the estimated infection onset to positive specimen
distribution for $t \in \cT$. With these modifications to the inputs, the
deconvolution proceeds in the exact same way as before. Since this deconvolution
is done separately for each variant category, we ultimately obtain deconvolved
case estimates by the date of infection onset that are separated by variant.

\subsection{Inverse reporting ratio and the antibody prevalence model} 

The infection estimates from retrospective deconvolution are derived solely from
the infection onset dates of the reported cases. To capture the unreported
infections, it is necessary to adjust these deconvolved case estimates by a
scaling factor that approximates the ratio of the true number of new infections
to the new reported infections. We refer to this quantity as the inverse
reporting ratio and denote it by $a_t$ for day $t$. Our new goal is to estimate
this quantity for every state at every time under consideration from June 1,
2020 to November 29, 2021.

The number of new reported infections is obtained from our deconvolved case
estimates. As for the true infections, since seroprevalence of anti-nucleocapsid
antibodies is used to estimate the percentage of people who have at least one
resolving or past infection \citep{cdc2020data}, we can use the change in
subsequent seroprevalence measurements to capture new infections, accounting for
those whose antibody levels fall below the detection threshold. We can adjust
the retrospective deconvolution estimates using a model that is based on such
seroprevalence estimates.

To estimate the proportion of the population in each state with evidence of
previous infection across time, we use two major seroprevalence surveys that
were led by the CDC: the 2020--2021 Blood Donor Seroprevalence Survey and the
Nationwide Commercial Lab Seroprevalence Survey \citep{cdc2021blood,
cdc2021comm}. In the former, the CDC collaborated with 17 blood collection
organizations in the largest nationwide COVID-19 seroprevalence survey to date
\citep{cdc2021blood}. The blood donation samples were used to construct monthly
seroprevalence estimates for nearly all states from July 2020 to December 2021
\citep{jones2021estimated}. In the latter survey, the CDC collaborated with two
private commercial laboratories and used blood samples to test for the
antibodies to the virus from people that were in for routine or clinical
management (presumably unrelated to COVID-19, \citealp{bajema2021estimated}). The
resulting dataset contains seroprevalence estimates for a number of multi-week
collection periods starting in July 2020 to February 2022. 

Both datasets are based on repeated, cross-sectional studies that aimed, at
least in part, to estimate the percentage of people who were previously infected
with COVID-19 using the percentage of people from a convenience sample who had
antibodies against the virus \citep{bajema2021estimated, cdc2020data,
jones2021estimated}. Adjustments were made in both for age and sex to account
for the demographic differences between the sampled and the target populations.
However, both datasets are incomplete and they differ in the number and the
timing of the data points for each state (\autoref{fig:sero_blood_comm_compar}).
Such limitations indicate that reliance upon only one seroprevalence survey is
inadvisable. For example, in the commercial dataset, the last estimate for North
Dakota is in September 2020. In the blood donor dataset, Arkansas does not have
estimates available until October 2020. 
% In addition, this blood donor dataset lacks
% measurements for any states in 2022 (as the corresponding survey ended in
% December 2021). 
% Finally, as can be seen from \autoref{fig:sero_blood_comm_compar},
%the final commercial seroprevalence measurement from 2022 shows a large
%increase relative to the immediately preceding measurement for each state. Since
%such an increase may signal unreliability or instability of the final estimates,
%we decided to remove them from our analysis. Note that North Dakota is the only
%state to which this exclusion does not apply as there are no commercial seroprevalence
%measurements beyond 2020.

\begin{figure}[!tb]
\centering
    \includegraphics[width=.99\textwidth]{sero_blood_comm_compar.pdf}
    \caption{A comparison of the seroprevalence estimates from the Commercial
    Lab Seroprevalence Survey dataset (yellow) and the 2020--2021 Blood Donor 
    Seroprevalence Survey dataset (blue). Note that the maximum and the minimum
    of the line ranges are the provided 95\% confidence interval bounds to 
    give a rough indication of uncertainty. \attn{This figure doesn't use space
    very well. Let's remove the gap between panels and make the facet labels
    (the state names) normal sized. The x-axis could just show 2021, 2022 (rather
    than so many ticks). Alternatively, another map layout? And I don't think
    ``Rate'' is correct for the y-axis.}}
    \label{fig:sero_blood_comm_compar}
\end{figure}

The date variables that come with the two seroprevalence datasets are different
and so the date variables that we are able to construct from them are not the
same. For the commercial dataset, we use the midpoint of the provided specimen
collection date variable. A major difference in the structure of the two
datasets is that the commercial dataset always has the seroprevalence estimates
at the level of the state, while the blood donor dataset can either have
estimates for the state or for multiple separate regions within the state. For
the blood donor dataset, we use the median donation date if the seroprevalence
estimates are designated to be for entire state. If they are instead for regions
in the state, since there is reliably one measurement per region per month, we
aggregate the measurements into one per month per state by using a weighted
average (to account for the given sample sizes of the regions). The median of
the median dates is taken to be the date for the weighted average.

The daily fraction of new infections are estimated from the provided incidence
of suspected reinfections over March 2020 to April 2022 in Clark County, which
is based on surveillance work conducted by the Southern Nevada Health District
(SNHD) and reported by \citet{ruff2022rapid}. The proportion of new cases per
week that are suspected reinfections are calculated by dividing the number of
suspected reinfections by all new PCR-identified cases during the same week. 
%Possible problem here – reinfections not include third infections (see comments
% in the discussion section about this)

 
To adapt to the sparseness in the seroprevalence data, we convert our daily data
to weekly by summing the reported infections and shifting the observed
seroprevalence measurements to the nearest Monday. If there are multiple
measurements in a week from a seroprevalence source, then the average is used.
We denote these changes by changing the time-based subscript from $t$ to $m$
where $m$ indicates the Monday relative to our June 1, 2020 start date.

For each state, let $s_m$ be the seroprevalence estimate on $m$, $w_m$ be the
corresponding inverse variance weight, and $C_{m-1}^m = \sum_{t = m-1}^{m} c_t$
be the total reported infections from $m-1$ to $m$ scaled by the state's
population. To account for reinfections, we multiply the change in reported
infections for $m$ by the corresponding fraction of new infections, $z_m$. 

Using these components, we construct the following model separately for each state
%\begin{align}
%\min_{a, \gamma}\frac{1}{2}\sum_{t \in T}w_t\left (s_t - (1 -\gamma)s_{t-1} 
%- a_t \Delta C_t z_t  \right )^2 + \frac{\lambda}{2} \|D^{(3)}a\|_2^2 \label{eq:waningpr}
%\end{align}
%  $w_t$ be the inverse variance weights derived from those estimates,
\begin{align}
s_m &= (1 -\gamma)s_{m-1} + a_m C_{m-1}^m z_m + \epsilon_m, \qquad \epsilon_m 
\sim N(0, w_m \sigma^2_\epsilon) \label{eq:waningpr}  \\
a_{m+1} &= 3a_{m} - 3a_{m-1} + a_{m-2} + \eta_m,\qquad\qquad \eta_m 
\sim N(0, \sigma^2_{\eta})   \nonumber
\end{align}
where $\gamma$ is the percentage of people whose level of infection-induced
antibodies falls below the detection threshold between time $m$ and time $m+1$.
Informally, we refer to $\gamma$ as the waning parameter and we call this model
the population antibody prevalence model. 
% By ``waning'' we are referring to the natural decline in the infection-induced antibodies over time. 
%Since the true course of immunity over time is unknown
% \citep{goldberg2022protection}, we take the straightforward approach 
% and model one $\gamma$ to try avoid making gratuitous
% or overly restrictive assumptions. 

We express the antibody prevalence model as a state-space model. This
representation allows for convenient handling of missing data, extrapolation
before and after the period of observed seroprevalence measurements, and maximum
likelihood estimates of $\gamma$ and $\sigma^2_\epsilon$. Details of this
methodology and the computation of the associated uncertainty measurements are
deferred to the Supplementary Materials \ref{supp:ssapm}.


\subsection{Lagged correlation to hospitalizations and time-varying IHRs} 

We use our infection estimates in a lagged correlation analysis with confirmed
COVID-19 hospitalizations. Our primary goal of this analysis is to find the lag
between infection and hospitalization rates that gives the highest average
rank-based correlation across \US states. To that end, we consider a wide range
of possible lag values ranging from 1 to 25 days. Zero and negative lags are not
considered because COVID-19 infection onset must precede hospitalization due to
the virus. To remove day of the week effects, both the infection and
hospitalization signals are subject to a 7-day moving average (center-aligned)
before their conversion to rates.

For each considered lag, we calculate the Spearman's correlation between the 
state infection and hospitalization rates for each observed day 
over the June 1, 2020 to November 29, 2021
time period %, and use the epi\_slide function \citep{mcdonald2023epipredict} to 
with a center-aligned rolling window of 61 days for each such computation.
We then calculate the average correlation across all states and times for each lag. 
The lag that leads to the highest average correlation is used to estimate 
the time-varying IHRs for each
state. To compute this for a given day, the number of individuals who are
hospitalized due to COVID-19 on a day are divided by the estimated total number
who were infected on the lagged number of days before.

\subsection{Ablation study for the lagged correlation analysis} 

To better understand the contribution of the intermediate steps to the lagged
correlation analysis, we carry out a brief ablation study in which we calculate
the lagged correlation using the following infection estimates: 1. those from
the deconvolution procedure under the assumption that the infection onset is the
same as the positive specimen date (i.e., excluding the positive specimen to
infection onset data and deconvolution); 2. those from the deconvolution
procedure under the assumption that the infection onset is the same as the
symptom onset date (excluding the incubation period data); 3. those from the
deconvolution procedure when utilizing all incubation period and delay data (the
deconvolved case estimates); 4. those from applying the antibody prevalence
model to produce estimates for both the reported and the unreported cases (the
infection estimates).

\section*{Data availability}
\section*{Code availability}

\clearpage
\bibliographystyle{unsrtnat}
\bibliography{bibliography.bib}
%\bibliographystyle{naturemag} %% Change back to numeric references in line with
%requirements for Nature Communications articles (see pg 3 fro here:
%https://www.nature.com/documents/ncomms-formatting-instructions.pdf)



\section*{Acknowledgements}

% Required Gisaid acknowledgement
We gratefully acknowledge all data contributors, i.e., the Authors and their
Originating laboratories responsible for obtaining the specimens, and their
Submitting laboratories for generating the genetic sequence and metadata and
sharing via the GISAID Initiative \citep{elbe2017data}, on which this research
is based.

\attn{Grants, Delphi, etc.}

\section*{Author contributions}

\section*{Competing interests}

The authors declare no competing interests.





\clearpage
% Eventually make the supplement a separate document with its own title page 
\beginsupplement
\title{\supptitlefont Online Supplement}
\maketitle

\section{Additional information about dataset used or estimation methodology}

\subsection{Table on the percent pairwise occurrence of events in the CDC line list}
\begin{table}[h!]
\begin{tabular}{|l|l|l|}
\hline
\textbf{Order of events}                              & \textbf{Percent pairwise occurrence}                                                                                    & \textbf{Handling}                                                                                                                                                                                                                                                                                                                                                          \\ \hline
IO $\rightarrow$ SO $\rightarrow$ PS $\rightarrow$ RE & \begin{tabular}[c]{@{}l@{}}PS $\geq$ SO: 97.1 \\ PS = SO: 33.6\\ PS \textgreater RE: 1.74\\  PS = RE: 14.6\end{tabular} & \begin{tabular}[c]{@{}l@{}}This is the idealized order of events and so we \\ built the current support sets for \\ SO $\rightarrow$ PS and PS $\rightarrow$ RE \\ delay distribution constructions around this \\ such that IO comes first by construction, \\ SO typically precedes PS, but may be the same \\ or come before, and RE comes after PS and SO\end{tabular} \\ \hline
IO $\rightarrow$ PS $\rightarrow$ SO $\rightarrow$ RE & \begin{tabular}[c]{@{}l@{}}PS \textless SO: 2.91 \\ SO $\leq$ RE: 99.3 \\ SO \textless RE: 86.1\end{tabular}            & \begin{tabular}[c]{@{}l@{}}Allowed for negative delays up to the largest \\ non-outlier value for the 0.05 quantile of delay\\ from PS to SO by state\end{tabular}                                                                                                                                                                                                         \\ \hline
IO $\rightarrow$ PS $\rightarrow$ RE $\rightarrow$ SO & \begin{tabular}[c]{@{}l@{}}RE \textless SO: 0.7 \\ RE \textless PS: 1.7\end{tabular}                                    & \begin{tabular}[c]{@{}l@{}}Nothing because current handling of the CDC \\ of the line list ensures that the most concerning \\ cases are handled where \\ SO = PO = RE, SO = RE and PO = RE\end{tabular}                                                                                                                                                                   \\ \hline
\end{tabular}
\caption{Percent pairwise occurrence for the different permutations of events considered in the restricted CDC line list. The abbreviation IO stands for infection onset, SO is symptom onset, PS is positive specimen, and RE is report date. We consider a restricted set of permutations because we assume that IO must come first and that PS must precede report date for a case to be legitimate. Finally, the underlying assumption for the percent pairwise occurrence calculations is that the cases must have both elements present (not missing).}
\label{tab:order_events_table}
\end{table}

\subsection{State space representation of the antibody prevalence model}\label{supp:ssapm} 

The antibody prevalence model from \autoref{eq:waningpr} is conceptualized
as a Gaussian state space model (as in \citealp{durbin2012time, helske2017kfas}).

In general, for $t = 1, \dots, n$, let $\alpha_t$ be the $m \times 1$ vector of latent
state processes at time $t$ and $y_t$ be the $p \times 1$ vector of observations
at time $t$. Under the assumption that $\eta$ is a $k \times 1$ vector, the
form of the linear Gaussian state space model is 
\begin{align}
y_t &= Z\alpha_t + \epsilon_t, \qquad \epsilon_t \sim N(0, H_t) \label{eq:ss1}\\
\alpha_{t+1} &= T_t\alpha_t + R_t\eta_t, \quad \eta_t \sim N(0, Q_t) \label{eq:ss2}
\end{align}
where $\alpha_1 \sim N(a_1, P_1)$ and 
there is independence amongst $\alpha_1$, $\epsilon_t$ and $\eta_t$
\citep{helske2017kfas, durbin2012time}. For notational
compactness, we let $\alpha = \left ( \alpha_1^\top, \dots, \alpha_n^\top \right )$
and $y = \left ( y_1^\top, \dots, y_n^\top \right )$.

The observation equation can be viewed as a linear regression model with the
time-varying coefficient $\alpha_t$, while the second equation is a first-order
autoregressive model, which is Markovian in nature \citep{durbin2012time}. 

The underlying idea behind the two equations is that we are assuming that the
system evolves according to $\alpha_t$ (as in the second equation), 
but since those states are
not directly observed, we turn to the observations $y_t$ and use their
relationship with $\alpha_t$ (as in the first equation) to drive the system
forward \citep{durbin2012time}. So the objective of state space modeling is to
obtain the latent states $\alpha$ based on the observations $y$ and this is
achieved through Kalman filtering and smoothing. 

Kalman filtering gives the following one-step-ahead predictions of the states
\begin{align*}
a_{t+1} &= \E[\alpha_{t+1}\given y_t, \dots, y_1] 
\end{align*} with covariance,
\begin{align*}
P_{t+1} &= \Var(\alpha_{t+1} \given y_t, \dots, y_1).
\end{align*}
Then, the Kalman smoother works backwards to the first time to give
\begin{align}
\hat{a}_t &= \E[\alpha_{t}\given y_n, \dots, y_1] \label{eq:hatat}\\
V_t &= \Var(\alpha_{t}\given y_n, \dots, y_1). \label{eq:Vt}
\end{align}
The filtering and smoothing steps are based on recursions that are described in
Appendix A of \citet{helske2017kfas} as we use the R package KFAS to estimate
our model.

% For our situation, the Kalman filter and smoothing approach offers a number of
% advantages over the penalized regression approach. Perhaps most notably,
% the parameters are estimated all at once (so cross validating for model
% parameter tuning is not necessary). Another major benefit is that it can handle 
% unevenly spaced time series (refer to \citealp{durbin2012time} for further details).

To express the antibody prevalence model in state space form, we define
 the components in Equations \ref{eq:ss1} and \ref{eq:ss2} as follows:

% Probably move the below specification to the appendix

\begin{alignat*}{3}
R &= \begin{bmatrix}
1 & 0  \\ 
0 & 1 \\ 
0 & 0 \\ 
0 & 0 
\end{bmatrix} &\qquad 
Z &= \begin{bmatrix}
1 & 0 & 0 & 0 \\ 
0 & 1 & 0 & 0 
\end{bmatrix} &\qquad 
H_m &= \begin{bmatrix} %%
w_{m,c}\sigma^2_o & 0 \\ 
0 & w_{m,b}\sigma^2_o
\end{bmatrix} \\
\alpha_m &= \begin{bmatrix}
s_{m}\\
a_m\\ 
a_{m-1}\\ 
a_{m-2}
\end{bmatrix} & 
T_m &= \begin{bmatrix}
 \gamma & C_{m-1}^m z_m & 0 & 0\\ 
 0 & 3 & -3 & 1 \\ 
 0 & 1 & 0 & 0\\ 
 0 & 0 & 1 & 0
\end{bmatrix}  & 
Q &= \begin{bmatrix} 
\sigma^2_s & 0  \\ 
0 & \sigma^2_a
\end{bmatrix} \\
a_1 &= \begin{bmatrix}
\tilde{s}_{1}\\ 
\tilde{a}_1\\ 
\tilde{a}_1 \\
\tilde{a}_1
\end{bmatrix} & 
P_{1} &= \begin{bmatrix}
\sigma^2_{\tilde{s}_{1}} & 0 & 0 & 0 \\ 
0 & \sigma^2_{\tilde{a}_1} & 0 & 0\\ 
0 & 0 & \sigma^2_{\tilde{a}_1} & 0 \\ 
0 & 0 & 0 & \sigma^2_{\tilde{a}_1}
\end{bmatrix} 
\end{alignat*}
where $\sigma^2_o$ is the variance of observations,
$\sigma^2_s$ is the variance of the seroprevalence estimates, 
and $\sigma^2_a$ is the trend variance. Since we expect the 
inverse ratios to be more variable than the seroprevalence estimates, 
we enforce that the estimate of $\sigma^2_a$ is a multiple of 
$\sigma^2_s$. Letting the subscripts $b$ and $c$ denote
the blood donor and commercial datasets, $w_{m,c}$ and $w_{m,b}$ are the
time-varying inverse variance weights computed from the commercial and blood
donor datasets, respectively. 

For each source, we compute the weights for the observed seroprevalence
estimates using the standard formula for the standard error of a proportion.
These weights are then re-scaled so they sum to the number of observed
seroprevalence measurements for the source. All days that are unobserved (i.e.,
lack seroprevalence measurements) are given weights of one. Finally, the ratio
of the average observed weights for the sources is used as a multiplier to scale
all of the weights for one source. For example, if the average weight of the
commercial source is double the average weight of the blood donor source (for an
arbitrary state), then we scale all of the weights in the commercial source
(including the ones) by two. The main purpose of this step is to ensure that
the source with a greater sample size contributes more weight in the model on
average. % Last sentence - Or is it simply that the source with larger
%weights on average will have more weight? 

The prior distribution for $\alpha_1$ is estimated using both data-driven constraints 
and externally sourced information. To obtain the initial value of the seroprevalence 
component, $\tilde{s}_{1}$, we extract the first observed seroprevalence measurement from 
each source, round down to two decimal places, and take the average to be $\tilde{s}_{1}$. 
The corresponding initial
variance estimate, $\sigma^2_{\tilde{s}_{1}}$, is taken to be the mean of the standard
errors of the two seroprevalence estimates. For all of the initial values of the trend 
components, we use the inverse of the ascertainment ratio estimate as of June 1, 2020 
for each state from Table 1 in \citet{unwin2020state} and denote this by $\tilde{a}_1$. The
initial variance estimate of $\sigma^2_{\tilde{a}_1}$ is based on the variance implied 
by the given inverse ascertainment ratio distribution.
% standard deviation implied by the interval in that table.  
% Update this last sentence if end up going with the standard deviation implied by the interval in Table 1
% instead of the variance implied by u_m ~ Beta(12,5) from the unwin2020state paper

The initial $\sigma^2_o$ is taken to be the average of the estimated variances
from the linear models for the sources where the observed seroprevalence
measurements are regressed on the enumerated dates. The initial value of 
the multiplier is set to be $100$ for all states. The $\sigma^2_s$ and $\gamma$ 
values are fixed and from averaging the estimated values for all states on the real line
(obtained under the starting conditions $\sigma^2_s = 0.000003$, $\gamma = 0.99$,
and $\sigma^2_o$ as described).

Following the maximum likelihood estimation of the two non-fixed parameters
we use the Kalman filtering and smoothing to obtain the
smoothed estimates of the weekly inverse reporting ratios and
their covariance matrices as shown in Equations \ref{eq:hatat} and \ref{eq:Vt}.
Forwards and backwards extrapolation is then used to estimate the ratios and covariance
outside of the observed seroprevalence range \citep{durbin2012time}, followed by linear 
interpolation to fill-in estimates for each day in our considered time period. 
After we obtain one vector of inverse reporting ratios for each state in this
way, we take each inverse reporting ratio and multiply it by the corresponding
deconvolved case estimate (that has undergone linear interpolation to correct
instances of $0$ reported infections) to obtain an estimate of new infections.
We are able to convert these numbers of infections to
infections per $100,000$ population by simple re-scaling (enabled by the fact
that normality is preserved under linear transformations).

The $50$, $80$, and $95\%$ confidence intervals are constructed by taking a
Bayesian view of the antibody prevalence model (refer to \ref{supp:bayeswaning} 
for the Bayesian specification of the model). 
That is, for each time, $t$, we obtain an estimate of the
posterior variance of $a_t$, apply the deconvolved case estimate as a constant
multiplier, and then use resulting variance to build a normal confidence
interval about the infection estimate. We additionally enforce that the lower
bound must be at least the deconvolved case estimate for the time under consideration.


\subsection{Bayesian specification of the antibody prevalence
model}\label{supp:bayeswaning} 
In brief, the antibody prevalence model where we let
$\beta = \left \{  \gamma, a_1,\dots, a_t \right \}$ and $X$ be the design
matrix, corresponds to a Bayesian model with prior 
\begin{align*}
    \beta \sim N \left( 0,  \frac{\sigma^2 }{ \lambda} \left( A^TD^TDA 
    \right)^{-1}  \right)
\end{align*} and likelihood 
\begin{align*}
    s|X,\beta \sim N \left( X\beta, \sigma^2W^{-1} \right),
\end{align*} where $A$ is indicator matrix save for the first column of $0$s 
(corresponding to $\gamma$), $D$ represents the discrete derivative matrix of 
order $3$, and $W$ is the inverse variance weights matrix. Then, the posterior 
on $a_t$ is normally distributed with mean 
\begin{align*}
    \left ( X^TWX + \lambda A^TD^TDA \right )^{-1}X^TWs
\end{align*} 
and variance 
\begin{align*}
    \sigma^2 (X^TWX + \lambda A^TD^TDA)^{-1}.
\end{align*}



\subsection{Ablation analysis of infection-hospitalization correlations}

We undertake an ablation study for the lagged correlation of infections, the
results of which are shown in \autoref{fig:correlations}. From
this, we can see that the deconvolved case or infection estimates from the
intermediate steps are all leading indicators of hospitalizations. However, the
degree that each such set of estimates lead hospitalizations depend on its
location in the sequence of steps and how close the estimates are to infection
onset. For example, the deconvolved cases by positive specimen date tend to
precede hospitalizations by about $11$ days, while those for the subsequent step
indicate that the deconvolved cases by symptom onset tend to precede
hospitalizations by a longer time of about $13$ days. Finally, after adding the
variant-specific incubation period data into the deconvolution and obtaining the
deconvolved case estimates, we can observe that the reported infections precede
hospitalizations by about $17$ days. 

\begin{figure}[!tb]
\centering
    \includegraphics[width=.8\textwidth]{adj_unadj_pi_no_inc_hosp_lag_corr_F24.pdf} 
    \caption{Lagged Spearman's correlation between the infection and
    hospitalization rates per 100,000 averaged for each lag across \US states
    and days over June 1, 2020 to November 29, 2021, and taken over a rolling
    window of 61 days. The infection rates are based on the counts for the
    deconvolved case and infection estimates as well as the reported infections
    by symptom onset and when the report is symptom onset. Note that each such
    set of infection counts is subject to a center-aligned 7-day averaging to
    remove spurious day of the week effects. The dashed lines indicate the lags
    for which the highest average correlation is attained.}
    \label{fig:adj_unadj_sym_hosp_lag_corr}
\end{figure}


\subsection{Scaling by population}

Annual estimates of the resident state
populations as of July 1 of 2020 and 2021 are taken from the December
2022 press release on the \US Census Bureau website \citep{uscensus2022annual}.
Unless otherwise specified, we use the July 1, 2020 estimates.


\subsection{Additional details on the date fields in the CDC linelist}
\label{sec:linelist-details}

Since the restricted dataset is updated
monthly and cases may undergo revision, we use a single version of it that was
released on June 6, 2022. We consider this version to be finalized in that it
well-beyond our study end date such that the dataset is unlikely to be subject
to further significant revisions.

\autoref{tab:order_events_table} presents the percent of pairwise occurrences
for the different possible permutations of events in the line list. Essentially,
most cases follow the idealized ordering shown by
\autoref{fig:chain_events_onset_report} and so we adhere to this construction as
much as possible.

 

We observe that the line list is prone to high percentages of missing data,
notably with respect to our variables of interest. Approximately 62.3\% of cases
are missing the symptom onset date, 55.4\% are missing positive specimen date,
and 8.96\% of cases are missing the report date. Relatedly,
cases with
missing report or positive specimen dates may be filled with their symptom onset date
\citet{jahja2022real}. So it is possible that all three variables may be
imputed with the same date for a case. However, we only actually deal with
select pairs of events; we do not use all three at once in our construction of
the delay distributions or anywhere else in our analysis. Therefore, we restrict
our investigation of missingness to the pairs of events.
\autoref{fig:prop-cc} suggests that this issue impacts states
differentially due to the inconsistent proportions of zero delay between
positive specimen and report date across states. 

Due to the contamination in the zero delay cases (the true extent of which is
unknown to us), we omit all such cases where the positive specimen and report
dates have zero delay from our analysis. We choose to allow for zero and
negative delay for symptom onset to report because correspondence with the CDC
confirms the distinct possibility that a person could test positive before
symptom onset and it is a reasonable ordering to expect if, for example, the
individual is aware that they have been exposed to an infected individual.

\begin{figure}[!tb]
\centering
\includegraphics[width=0.9\linewidth]{prop_cc_zero_delay.pdf}\\
\includegraphics[width=0.9\linewidth]{prop_cc_cdc_vs_jhu.pdf} 
\caption{Top panel: Proportion of complete cases with zero delay between
    positive specimen and report date in the restricted CDC line list dataset.
    Bottom panel: Complete case counts by state in the CDC line list versus the
    cumulative complete case counts from JHU CSSE as of June 6, 2022. All
    counts have been scaled by the 2022 state populations as of July 1, 2022
    from \citet{uscensus2022annual}.}
\label{fig:prop-cc}
\end{figure}

For the same release date, the restricted line list contains 74,849,225 cases
(rows) in total compared to 84,714,805 cases reported by the JHU CSSE; that is,
line list is missing about 10 million cases. The extent that this issue impacts
each state is shown in \autoref{fig:prop-cc}, from which it is clear
the fraction of missing cases is substantial for many states, often surpassing
50\% \citep{jahja2022real}. In addition, the probability of being missing does
not appear to be the same for states, so there is likely bias introduced from
using the complete case line list data. We consider such bias to be unavoidable
in our analysis due to a lack of alternative line list sources.

In the line list, we observe unusual jarring spikes in reporting in 2020
compared to 2021. Upon plotting by report date, we find that a few states are
contributing unusually large case counts on isolated days very late in the
reporting process (usually well beyond 50 days). We strongly suspect that these
large accumulations of cases over time are due breakdowns of the reporting
pipeline (which may be expected to occur more frequently in the year following
its instantiation than later in time). Such anomalies are not likely to be
reliable indicators of the delay from positive specimen to case report.
Therefore, we devise a simple, ad hoc approach to detect and prune these
reporting backlogs.

First, we obtain the part of the line list intended for the positive specimen to
case report delay estimation, where both such dates are present and where zero
and negative delay cases have been omitted. Then, for each of the three dates of
June 1, September 1, and December 1, 2020, we bin the reporting delays occurring
from 50 days up to the maximum observed delay. For each bin, we obtain the total
delay count for each state. We check whether each count on the log scale is at
least the median (for the bin) plus 1.5 times the interquartile range and retain
only those that exceed this criterion as potential candidates for pruning. Next,
we compute the counts by report date for each candidate state. If there is a
report date with a count greater than or equal to the pre-specified threshold,
then we remove those cases from the line list. Based on inspection and
intuition, we set the threshold to 2000 for the first two bins, and then lower
it to 500 for the remaining bins. A similar trial and error approach is used to
set the bin size (to 50 days).

\subsection{Justifications for delay distribution calculations}
\label{sec:delay-justifications}

Let $y_t$ denote the count of new cases reported at time $t$ and $x_t$ denote
the count of deconvolved cases with positive specimen at $t$. For all cases in
the line list that had both a positive specimen and a report date, we can count
the those that are reported at time $t$ by enumerating them according to
positive specimen date (similar to how symptom onset date was used in
\citealp{jahja2022real}):
\begin{align*}
y_t = \sum_{s=1}^{t} \sum_{i=1}^{x_s}\indicator \left ( \text{the }i\th\text{ positive specimen at }
    s \text{ gets reported at }t \right ).
\end{align*}
Taking the conditional expectation of the above yields
\begin{align*}
\E(y_t \mid x_s, s \leq t) = \sum_{s=1}^{t} \pi_t(s) x_s ,
\end{align*}
where $\pi_t(s) = \P(\text{case report at }t \mid \text{positive specimen date
at }s)$ for each $s \leq t$ are the delay probabilities and the $\{ \pi_t(s) : s
\leq t \}$ sequence comprises the delay distribution at time $t$. Notice that
there are no time restrictions placed on the positive specimen date, except that
it must have been between the start of the pandemic and the report date,
inclusive. This is unlikely to be a realistic assumption to make as $t$ moves
farther away from $s$. 

Thus, we make two key assumptions about these distributions. First, positive
specimen tests that are reported to the CDC are always reported within $d = 60$
days, which is true for the majority of the reported cases. Second, the
probability of zero delay is zero, which stems from the contamination of
zero-delay in the line list. As in \citet{jahja2022real}, we update the
conditional expectation formula to reflect these two assumptions: 
\begin{align*}
\E(y_t \mid x_s, s \leq t) = \sum_{k=1}^{60} p_t(k) x_{t-k}
\end{align*}
where for $k = 1, \dots, 60$,
\begin{align*}
p_t(k) = \P(\text{case report at }t \mid \text{positive specimen at }t-k).
\end{align*}


Thirdly, there are times where the empirical probability
was observed to be precisely 1 at zero delay and the proportion of CDC relative
to JHU cases used for the weight was also 1. Since we believe that having zero
delay for all cases is unrealistic and unlikely to be representative of all
cases for the state, we inject a small amount of variance manually by setting
the the CDC-to-JHU proportion to be the minimum shrinkage proportion observed
for the affected state (such instances were isolated to the state of New
Hampshire). Aside from these modifications, the construction of the delay
distribution proceeds in precisely the same manner as for positive specimen to
report date. 



\subsection{Possible investigation of reinfections}

Possible change to the paper based on Ajitesh's feedback - 
Main contribution is the model, shows without reinfection 
\& here's an extension that shows how to include reinfection data. 


\end{document}