\documentclass{article}
% Packages used
\input{latex-head.tex}
\def\TimesFont{} 
\graphicspath{{gfx/}}

\newcommand{\beginsupplement}{
  \setcounter{table}{0}  
  \renewcommand{\thetable}{S\arabic{table}} 
  \setcounter{figure}{0} 
  \renewcommand{\thefigure}{S\arabic{figure}}
  \setcounter{section}{0} 
  \renewcommand{\thesection}{S\arabic{section}}
}
\font\supptitlefont=cmr12 at 16pt 
\newcommand{\attn }[1]{\textcolor{red}{ATTN: #1}}
     
\begin{document}
\title{Retrospective estimation of latent COVID-19 infections over the pandemic in US states}
\author{Rachel Lobay, Maria Jahja, Ajitesh Srivastava, Ryan J.\ Tibshirani, Daniel J.\ McDonald}
\date{Version: \today}
\maketitle

\begin{abstract}
The true timing and magnitude of infections from the COVID-19 pandemic are of
interest to both the public and public health, but these are challenging to pin
down for a variety of data-driven and methodological reasons. Nonetheless, 
accurate estimates
of all latent infections can improve our understanding of the true size and
scope of the pandemic and provide an indication of disease patterns and burden
over time. In this work, we estimate the true daily incident infections for each
\US\ state by deconvolving reported COVID-19 cases using estimated
infection-onset-to-case-report distributions followed by a serology-based
procedure to adjust for the unreported infections. We find clear
variability in the timing and magnitude in the resulting estimates, indicating a
differential impact of the pandemic across states and revealing a disease burden
that appears earlier and more extensively than indicated by cases alone. Our
findings help to better understand the impact of the pandemic in the \US\ at the state level.
\end{abstract}

\section{Introduction}

Reported COVID-19 cases are a staple in tracking the pandemic at varying
geographic resolutions such as national, state and county levels
\citep{dong2020interactive, nyt2020corona, wp2020tracking}. Yet, for every case
that is eventually reported to public health, several infections are likely to
be missed. To see why, it is important to understand who's cases are being reported and
 what differentiates them from the unreported cases. Refer to
\autoref{fig:chain_events_onset_report} for an illustration of the path of a
symptomatic infection that is eventually reported to public health. 

Using this figure, we can discern a number of sources of bias in the reporting
pipeline. For instance, diagnostic testing mainly targets symptomatic
individuals; thus, infected individuals exhibiting little to no symptoms are
likely to be missed \citep{cdc2022estimated}. In addition, testing practices,
availability, and uptake vary across space and time \citep{pitzer2021impact,
ecdc2020strategies, hitchings2021usefulness}. Finally, cases provide a belated
view of the pandemic's progression because they are subject to delays due to the
viral incubation period, the speed and severity of symptom onset, laboratory
confirmation, test turnaround times, and submission to public health
\citep{pellis2021challenges, wash2020dash}. For these reasons, reported cases
are a lagging indicator of the course of the pandemic. Furthermore, they do not
represent the actual number of new infections that occur on a given day, as
indicated by exposure to the pathogen. Ascertaining infection onset is difficult
because there was no large-scale surveillance effort in the United States that
reliably tracked symptom onset, let alone infection onset.

\begin{figure}[!tb]
\centering
    \includegraphics[width=.99\textwidth]{Chain_of_events_onset_report.pdf} 
    \caption{Idealized chain of events from infection onset to case report date 
    for a symptomatic infection that is eventually reported to public health.}
    \label{fig:chain_events_onset_report}
\end{figure}

% Importantly, all of these issues that are present in local health authority
% data are also present in the gold standard for case data from the JHU CSSE
% \citep{dong2020interactive, guidotti2022worldwide} because JHU scrapes case
% data from the local health authority dashboards \citep{jahja2022real}.
% Furthermore, the cases shown on the JHU CSSE Coronavirus Resource Center
% \citep{jhucsse2020covid} are those that have been disseminated to the public
% on a given day. 
% Our approach to estimate latent infections takes case data and estimates the 
% following...

Explaining the course of the pandemic and investigating the effects of
interventions, the burden facing various subgroups, and drawing insights for
future pandemics is challenging because the true spatial and temporal behaviour
is unknown. While reported cases provide some understanding of the disease
burden in a population, it is incomplete, delayed, and understates the true size
of the pandemic. Regardless of these difficulties, it is important to the public
and public health to perform a pandemic post-mortem and try to better estimate the
true extent of its effect---to attempt to capture the true size and impact of
the pandemic as much as we can. Estimates of daily incident infections are one
such way to measure this and can guide public and professional understanding of
the pandemic burden over space and time.

In this work, we provide a statistically justified reconstruction of
daily incident infections for each U.S.\ state from March 9, 2020 to February 28, 2022.
We achieve this by first breaking the task of estimating infection onset
from report date into the more manageable parts of estimating the time from
infection to symptom onset and the time from symptom onset to report date (as
depicted in \autoref{fig:chain_events_onset_report}). Using state-level data, we
construct state-time-specific incubation period and
symptom-onset-to-case report delay distributions. We then use these estimated
incubation period distributions in conjunction with their delay distributions to
deconvolve daily reported COVID-19 cases to infection onset. The resulting
infection estimates are adjusted to account for unreported infections by
using seroprevalence data in a novel leaky immunity model that is defined by its
ability to account for the waning of detectable immunity.  
We examine some features of the infection estimates and the implications of
using them rather than reported cases in assessing the impact of the pandemic.
We apply our infection estimates to get time-varying infection-hospitalization
ratios (IHRs) for each state and compare those to similarly derived
case-hospitalization ratios (CHRs).
% finding major spikes that correspond to the introduction of prominent new variants in both, but 
% higher ratios overall for cases.
While these analyses provide a glimpse into the utility of our
infection estimates, we believe that there is much more to be explored, and we hope that
our work will prove an important benchmark for others to undertake retrospective
analyses.

\section{Methods}

In what follows, we provide details on how we estimate the daily incident
infections for each state over the considered time period of March 9, 2020 to
February 28, 2022 and the data we used to achieve this. We start with a brief
introduction to each data source used and follow this with a description of each
major analysis task in the order they are performed.
\autoref{fig:cases_to_infect_flowchart} provides a visual summary of the data,
analysis tasks, and the relationships between them. The five major analysis
tasks this figure aims to convey are as follows: First, we estimate the
incubation period and delay distribution for each day over the considered time
period for a given state. Next, we join each of these two parts together using
convolution to obtain a distribution from infection onset to case report for
each time. We use the resulting probability estimates along with daily reported
cases in retrospective deconvolution to estimate the infection onset dates for
the reported cases. We adjust these infection estimates to account for the
unreported infections by using state-specific, time-varying seroprevalence data
in a leaky immunity model to reach our ultimate goal of obtaining daily incident
infection estimates for each state. 
% Then, we can apply these estimates in a
% lagged correlation to hospitalizations to find the ``best'' lag between
% infection and hospitalization rates according to Spearman's rank-based correlation 
% and use this to compute time-varying IHRs for each state. 


\begin{figure}[!tb]
\centering
    \includegraphics[width=.99\textwidth]{Reported_cases_to_infect_flowchart.pdf} 
    \caption{Flowchart of the inputted data and major analysis steps required 
    to get from reported cases to incident infection estimates for each day 
    over March 9, 2020 to February 28, 2022 for a state. Data sources are coloured 
    in yellow, while data analysis steps are coloured in blue. The data sources that
    do not stem from an analysis step are literature estimates.}
    \label{fig:cases_to_infect_flowchart}
\end{figure}


% Reasons why we should potentially go for a time period of June 1, 2020 to February 28, 2022. 
% A list of reasons on why I settled on these start and end dates: 
% 1. Reinfections officially started occurring 2020-06-01 according to the data we?re using
% 2. sero measurements start about a month after in July, 2020 for most states and end 
% usually in early Feb. 2022 for almost all states, 
% 3. the prior values for the inverse of the ascertainment ratio estimates are as of 2020-06-01
% 4.  the earliest hospitalizations for each state (that we use in the correlation analysis) 
% are also almost always in July, 2020. 
% So I am not convinced we should extrapolate multiple months away from the start or end for sero. 
% One month in either direction seems reasonable & doesn't overreach.

\subsection{Data} 


The variant-specific incubation periods are taken to be the same for all states.
For the Ancestral variants, we build this using
literature estimates of the gamma distribution parameters \citep{tindale2020evidence}. 
For the Alpha, Beta, Gamma, Delta and Omicron variants, we use the
mean and standard deviation of the number of days of incubation as reported in
\citet{tanaka2022shorter, grant2022impact, ogata2022shorter}.\footnote{To clarify, 
we use the estimates for Alpha and Omicron from \citet{tanaka2022shorter}, those for
Beta and Gamma from \citet{grant2022impact}, and those for Omicron from
\citet{ogata2022shorter}.} Since the literature lacks reliable estimates for the incubation
period of the Epsilon and Iota variants, we use the
incubation period for Beta because Epsilon, Iota, and Beta are all
children from the same parent in the phylogenetic tree of the Nextstrain Clades
(as depicted in \citealp{hodcroft2021covariants}).

To estimate the daily proportions of the variants circulating in each state, we
obtain the GISAID genomic sequencing data counts from CoVariants.org
\citep{hodcroft2021covariants, elbe2017data}.\footnote{The complete list of
EPI\_SET Identifiers that were used to produce the CoVariants data are provided
in the Acknowledgements section of their website
\citep{hodcroft2021covariants}.} Since these counts are biweekly totals, we use
a simple convex optimization approach to interpolate daily numbers, where we
enforce that the counts in each interval must sum to the right boundary (the
biweekly total) and linear growth between the pairs of adjacent days. 

The COVIDcast API \citep{reinhart2021open} is used to retrieve the daily number
of new confirmed COVID-19 cases for each state that are based on reports from
the John Hopkins Center for Systems Science and Engineering (JHU CSSE,
\citealp{dong2020interactive}). From the same API, we also retrieve the daily
number of confirmed COVID-19 hospital admissions for each state that are
collected by the U.S.\ Department of Health and Human Services (HHS). Both
datasets are updated as of June 6, 2022.

We obtain de-identified patient-level line list data on COVID-19 cases from the
CDC. Although there are both public and restricted versions of the dataset
available containing the same patient records \citep{cdc2020casepub,
cdc2020caserestr}, the restricted dataset\footnote{The CDC does not take
responsibility for the scientific validity or accuracy of methodology, results,
statistical analyses, or conclusions presented.} is selected because it contains
information on the state of residence which is essential for constructing
state-specific delay distributions. Since the restricted dataset is updated
monthly and cases may undergo revision, we use a single version of it that was
released on June 6, 2022. We consider this version to be finalized in that it
well-beyond our study end date such that the dataset is unlikely to be subject
to significant revisions.

In this dataset, the two key variables of interest are the dates of symptom
onset and report to the CDC. However, we find that the line list is prone to
high percentages of missing data, notably with respect to our variables of
interest. Nearly 60\% of cases are missing the symptom onset date, while about
9\% of cases are missing the report date. In addition, we faced the
fundamental issue that \citet{jahja2022real} described, in which cases with
missing report dates may be filled with their symptom onset date.
\autoref{fig:prop_cc_zero_delay} suggests that this impacts states differentially
due to the inconsistent proportions of complete cases (those with
both onset and report date) that have zero delay between onset and report across
states. Due to this contamination in the zero delay cases (the true extent of
which is unknown to us), we omit all such cases from our analysis.

\begin{figure}[!tb]
\centering
    \includegraphics[width=.99\textwidth]{prop_cc_zero_delay.pdf}
    \caption{Proportion of complete cases with zero delay by state in the 
    restricted CDC line list dataset.}
    \label{fig:prop_cc_zero_delay}
\end{figure}

For the same release date, the restricted line list contains 74,849,225 cases
(rows) in total compared to 84,714,805 cases reported by the JHU CSSE; that is,
line list is missing about 10 million cases. The
extent that this issue impacts each state is shown in
\autoref{fig:prop_cc_cdc_vs_jhu}, from which it is clear the fraction of missing
cases is substantial for many states, often surpassing 50\%
\citep{jahja2022real}. In addition, the probability of being missing does not
appear to be the same for states, so there is likely bias introduced from using
the complete case line list data. We consider such bias to be unavoidable in our
analysis due to a lack of alternative line list sources.

\begin{figure}[!tb]
\centering
    \includegraphics[width=0.99\textwidth]{prop_cc_cdc_vs_jhu.pdf} 
    \caption{Complete case counts by state in the CDC line list versus the 
    cumulative complete case counts from JHU CSSE as of June 6, 2022.
     All counts have been scaled by the 2022 state populations as of
     July 1, 2022 from \citet{uscensus2022annual}.}
    \label{fig:prop_cc_cdc_vs_jhu}
\end{figure}

In the line list, we observe unusual jarring spikes in reporting in 2020 compared
to 2021. Upon plotting by report date, we find that a few states are
contributing unusually large case counts on isolated days very late in the
reporting process (usually well beyond 50 days). We strongly suspect that
these large accumulations of cases over time are due breakdowns of the reporting
pipeline (which may be expected to occur more frequently in the year following
its instantiation than later in time). Such anomalies are not likely to be reliable
indicators of the delay from symptom onset to case report. Therefore, we devise
a simple, ad hoc approach to detect and prune these reporting backlogs. 

For each of the four dates of March 1, June 1, September 1, and %%%%%
December 1, 2020, we bin the reporting delays occurring
from 50 days up to the maximum observed delay. Then, for each bin, we obtain
the total delay count for each state. We check whether each count on the
log scale is at least the median (for the bin) plus 1.5 times the
interquartile range and retain only those that exceed this criterion as potential
candidates for pruning. Next, we compute the counts by report date for each
candidate state. If there is a report date with a count greater than or equal to
the pre-specified threshold, then we remove those cases from the line list.
Based on inspection and intuition, we set the threshold to 2000 for the
first two bins, and then lower it to 500 for the remaining bins. 
A similar trial and error approach is used to set the bin size (to 50 days). % Area for sensitivity
% analysis? Also, could discuss the choice of using only 3 dates 

To estimate the proportion of the population in each state with evidence of
previous infection across time, we use two major seroprevalence surveys that
were led by the CDC: the 2020--2021 Blood Donor Seroprevalence Survey and the
Nationwide Commercial Lab Seroprevalence Survey \citep{cdc2021blood,
cdc2021comm}. In the former, the CDC collaborated with 17 blood collection
organizations in the largest nationwide COVID-19 seroprevalence survey to date
\citep{cdc2021blood}. The blood donation samples were used to construct monthly
seroprevalence estimates for nearly all states from July 2020 to December 2021
\citep{jones2021estimated}. In the latter survey, the CDC collaborated with two
private commercial laboratories and used blood samples to test for the
antibodies to the virus from people that were in for routine or clinical
management (presumably unrelated to COVID-19, \citealp{bajema2021estimated}). The
resulting dataset contains seroprevalence estimates for a number of multi-week
collection periods starting in July 2020 to February 2022. 

Both datasets are based on repeated, cross-sectional studies that aimed, at
least in part, to estimate the percentage of people who were previously infected
with COVID-19 using the percentage of people from a convenience sample who had
antibodies against the virus \citep{bajema2021estimated, cdc2020data,
jones2021estimated}. Adjustments were made in both for age and sex to account
for the demographic differences between the sampled and the target populations.
However, both datasets are incomplete and they differ in the number and the
timing of the data points for each state (\autoref{fig:sero_blood_comm_compar}).
Such limitations indicate that reliance upon only one seroprevalence
survey is inadvisable.
For example, in the commercial dataset, the last estimate for North Dakota is in
September 2020. In the blood donor dataset, Arkansas does not have estimates
available until October 2020. In addition, this blood donor dataset lacks
measurements for any states in 2022 (as the corresponding survey ended in
December 2021). Finally, as can be seen from \autoref{fig:sero_blood_comm_compar},
the final commercial seroprevalence measurement in 2022 shows a large
increase relative to the immediately preceding measurement for each state. Since
such an increase may signal unreliability or instability of the final estimates,
we decided to remove them from our analysis. 

\begin{figure}[!tb]
\centering
    \includegraphics[width=.99\textwidth]{sero_blood_comm_compar.pdf}
    \caption{A comparison of the seroprevalence estimates from the Commercial
    Lab Seroprevalence Survey dataset (yellow) and the 2020--2021 Blood Donor 
    Seroprevalence Survey dataset (blue). Note that the maximum and the minimum
    of the line ranges are the provided 95\% confidence interval bounds to 
    give a rough indication of uncertainty.}
    \label{fig:sero_blood_comm_compar}
\end{figure}

The date variables that come with the two seroprevalence datasets are different
and so the date variables that we are able to construct from them are
not the same. For the commercial dataset, we use the midpoint of the provided
specimen collection date variable. A major difference in the structure of the
two datasets is that the commercial dataset always has the seroprevalence
estimates at the level of the state, while the blood donor dataset can either have 
estimates for the state or for multiple separate regions within the state. For the 
blood donor dataset, we use the median donation date if the seroprevalence 
estimates are designated to be for entire state. If they are instead for regions in 
the state, since there is reliably one measurement per region per month, we 
aggregate the measurements into one per month per state by using a weighted 
average (to account for the given sample sizes of the regions). The median of the 
median dates is taken to be the date for the weighted average.

For adjusting our infection counts, annual estimates of the resident state
populations as of July 1 of 2020, 2021, and 2022 are taken from the December
2022 press release on the U.S.\ Census Bureau website \citep{uscensus2022annual}.

The daily fraction of new infections are estimated from the provided incidence
of suspected reinfections over March 2020 to April 2022 in Clark County, which
is based on surveillance work conducted by the Southern Nevada Health District
(SNHD) and reported by \citet{ruff2022rapid}. The proportion of new cases per
week that are suspected reinfections are calculated by dividing the number of
suspected reinfections by all new PCR-identified cases during the same week. 
%Possible problem here – reinfections not include third infections (see comments
% in the discussion section about this)

% Possible change to the paper based on Ajitesh's feedback - 
% Main contribution is the model, shows without reinfection 
% & here's an extension that shows how to include reinfection data. 

\subsection{Estimating the incubation period distribution} 

For each state at each time over March 9, 2020 to February 28, 2022, we estimate
the incubation period distribution from a finite and countable mixture of gamma
distributions to account for the gradual decline in the incubation period across
variants. The variants we consider are Ancestral (for our purposes, 
all 2020 variants observed in the U.S.), as well as Alpha, Beta, Gamma, Delta and Omicron,
which are designated as variants of concern by WHO based on their potential to
cause new waves, dethrone the dominant variant, and lead to changes in public
health policy \citep{who2021tracking}. In addition, we include the Epsilon
(California) and Iota (New York) variants because of their impact on those and
the surrounding states \citep{yang2022investigation, duerr2021dominance}. We
relegate all other variants to be in an Other category (so that the proportions
circulating in a state at a time always sum to one). This decision is, in part,
motivated by the lack of sequencing data for most states in 2020 as well as the
presence of an others category in the sequencing data for that time. 
% Perhaps include heatmap(s) from 
% https://docs.google.com/document/d/1WSa_CN5gXsa-5Ozuj6xZt2-knflL507QdwiiRscaW2o/edit
Then, for each variant in a state at a time, the proportion of the variant
circulating (the mixture weight) is multiplied by the corresponding component
gamma distribution for the incubation period. These distributions are the same
for all states and based on literature estimates of the gamma parameters or the
mean and standard deviation of the incubation period (in which case the method
of moments is used to fit a gamma density). Finally, we discretized the
resulting mixture density to the support set, which is taken to be from 1 and
21 days. In other words, those are taken to be the lower and upper limits for
the number of days that the virus could be incubating in someone. The implicit
assumption for the lower bound is that there must be at least one day between
infection and symptom onset (which follows the convention given in
\citealp{phcan2021covid}). The assumption underlying the upper bound is that 21
days is the maximum number of days that the virus could be incubating in someone
(which is reasonable based on \citealp{zaki2021estimations} and
\citealp{cortes2022sars}).

\subsection{Estimating the delay from symptom onset to report date} 

We use the restricted CDC line list to estimate the distribution between symptom
onset and report for each state at each time. We refer to this as the ``delay
distribution'' \citep[see
\autoref{fig:cases_to_infect_flowchart} or][]{jahja2022real}. More formally, let
$y_t$ denote the count of new cases reported at time $t$ and $x_t$ denote the
count of new infections with onset at $t$ for a state. For all cases in the line
list that had both an onset and a report date, we can count the those that are
reported at time $t$ by enumerating them according to onset (as in
\citealp{jahja2022real}):
\begin{align*}
y_t = \sum_{s=1}^{t} \sum_{i=1}^{x_s}\indicator \left ( \text{the }i\th\text{ infection at }
 s \text{ gets reported at }t \right ).
\end{align*}
Taking the conditional expectation of the above yields
\begin{align*}
\E(y_t \mid x_s, s \leq t) = \sum_{s=1}^{t} \pi_t(s) x_s ,
\end{align*}
where $\pi_t(s) = \P(\text{case report at }t \mid \text{infection onset at }s)$ for
each $s \leq t$ are the delay probabilities and the $\{ \pi_t(s) : s \leq t
\}$ sequence comprises the delay distribution at time $t$. Notice that
there are no time restrictions placed on the infection onset, save that it must
have been between the start of the pandemic and the report date, inclusive. This is
unlikely to be a realistic assumption to make as $t$ moves farther away from
$s$. 

Thus, we make two key assumptions about the delay distributions. First,
infections that are reported to the CDC are always reported within $d = 60$
days, which is true for the vast majority of reported cases. Second, the
probability of zero delay is zero, which stems from the contamination of
zero-delay in the line list. As in \citet{jahja2022real}, we update the
conditional expectation formula to reflect these two assumptions: 
\begin{align*}
\E(y_t \mid x_s, s \leq t) = \sum_{k=1}^{60} p_t(k) x_{t-k}
\end{align*}
where for $k = 1, \dots, 60$,
\begin{align*}
p_t(k) = \P(\text{case report at }t \mid \text{infection onset at }t-k).
\end{align*}

For each state, we estimate the delay distribution at each $t$ by using the empirical distribution of
all non-zero lags between the complete cases whose onset dates fall in the
center-aligned interval about $t$ designated by $[ t - 75 + 1, t + 60
]$). 

Now, the task of estimating the delay distribution for each state at each time
requires four distinct steps. First, we obtain the empirical
distribution of all lags (excluding zero) from all cases with onset dates
falling in the center-aligned interval. Next, we weight the state-specific
empirical distribution by the proportion of CDC cases to JHU cases. That is, we
compare the number of CDC cases used to create the empirical distribution to the
number of cases reported by JHU in the time window of $\left[t - 60 + 2, t +
75\right]$ (to correspond appropriately to the center-aligned interval for
the CDC cases). This proportion is used as the weight for the state's empirical
distribution, while the complement is used to weight the overall empirical
distribution that is formed from the data for all states. This construction
allows for more reliance on the state's distribution when there are more CDC
cases relative to JHU (and vice versa). After implementing the shrinkage method,
we fit a gamma density to the resulting empirical distribution by the method of
moments. Finally, we discretize the resulting density to the support set of 1
to 60 days.
 
\subsection{Convolutional estimate of infection-to-report distributions} 

From the incubation period and delay distribution estimation, we acquire one
delay and one incubation period distribution for each state at each time under
consideration. We then convolve each pair of distributions to get the estimated
infection-to-report distributions and, hence, the estimated probabilities for
the delay from infection onset to case report. 

\subsection{Retrospective deconvolution for reported cases}

The goal for retrospective deconvolution is to estimate the daily number of new
infections that occurred at each time using the dates that those cases were eventually
reported. For each state, we
achieve this goal by solving the an optimization problem. 
% Following the notation of \citet{jahja2022real} (save for suppressing the
% state/location subscript), 
Let $\cT$ represent the deconvolution period from June
1, 2020 to February 28, 2022. Let $\hat{p}_t$ be probabilities from the estimated
infection-to-report distribution for $t \in \cT$, $y_t$ the number of new cases
reported, and $D^{(4)}x$ yields all $4\th$-order differences of the vector $x$
(by using the discrete derivative matrix of order 4, $D^{(4)}$). From these,
we estimate the latent infection counts for the reported cases across time by
solving for the vector $x$ in
\begin{align*}
\minimize_{x}\ \sum_{t \in \cT} \left ( y_t -  \sum_{k = 1}^{d} \hat{p}_t(k)x_{t-k} 
\right )^2 + \lambda \|D^{(4)}x\|_1. 
\end{align*}

The above loss function decouples into two parts which trade data fidelity with
desired smoothness (that
encapsulate the classic bias-variance trade off). The first part represents
minimizing the sum of squared errors between the JHU reported cases and the
estimates, while the second part captures the smoothness of the estimates
(smaller values being more smooth). The tuning parameter $\lambda$ determines
the relative importance of these competing goals.

We solve this trend-filtering-regularized least squares deconvolution problem by
employing the ADMM algorithm from \citet{ramdas2016fast} that is described in
Appendix A of \citet{jahja2022real}. The solution to the problem is an adaptive
piecewise cubic polynomial \citep{tibshirani2014adaptive,
tibshirani2022divided}.

We select the tuning parameter, $\lambda$, by using $3$-fold cross validation as
in \citet{jahja2022real} in which every third infection count is
reserved for testing and imputed with the average of the two surrounding counts.
The tuning parameter that results in the
smallest sum of squared errors is ultimately chosen.

\subsection{Inverse reporting ratio and the leaky immunity model} 

The infection estimates from retrospective deconvolution are
derived solely from the infection onset dates of the reported cases. 
To capture the unreported infections, it is necessary to adjust the estimates by 
a scaling factor that approximates the ratio of the true number of new infections
to the new reported infections. We refer to this quantity as the
inverse reporting ratio and denote it by $a_t$ for time $t$. Our new goal is
to estimate this quantity for every state at every time under consideration. 

The number of new reported infections is obtained from our
deconvolved case estimates. As for the true
infections, since seroprevalence of anti-nucleocapsid antibodies is used to
estimate the percentage of people who have at least one resolving or past
infection \citep{cdc2020data}, we can use
the change in subsequent seroprevalence measurements to capture new infections, 
accounting for those who lose enough immunity to fall below the detection threshold.
% Thus, we adjust the retrospective deconvolution estimates using a model that is 
% based on such seroprevalence estimates. 
For each state, let $s_t$ be a seroprevalence estimate at time $t$, $w_t$ be the
inverse variance weights corresponding to those estimates, and $\Delta R_t$ be
the change in cumulative reported infections scaled by the state's population.
To account for reinfections, we multiply the change in reported infections at
time $t$ by the corresponding fraction of new infections, $n_t$. Using these
components, we construct a model separately for each state
%\begin{align}
%\min_{a, \gamma}\frac{1}{2}\sum_{t \in T}w_t\left (s_t - (1 -\gamma)s_{t-1} 
%- a_t\Delta R_t n_t  \right )^2 + \frac{\lambda}{2} \|D^{(3)}a\|_2^2 \label{eq:leakypr}
%\end{align}
%  $w_t$ be the inverse variance weights derived from those estimates,
\begin{align}
s_t = (1 -\gamma)s_{t-1} + a_t\Delta R_t n_t + \epsilon_t \label{eq:leakypr}
\end{align}
where $\epsilon \sim N(0, w_t\sigma^2_\epsilon)$, and $\gamma$ is the percentage
of people who lose immunity between time $t$ and time $t+1$.
Informally, we refer to $\gamma$ as the leaky parameter and we call this model leaky 
immunity model. By ``leaky'' we mean the decrease in detectability of antibodies 
due to the natural degradation of infection-induced
immunity over time. Since the true course of immunity over time is unknown
\citep{goldberg2022protection}, we take this rather straightforward approach 
and simply model a singular $\gamma$ to try
avoid making gratuitous or overly restrictive assumptions. 

To estimate this
model, we will express it as a state-space model with weekly observations and
enforce $a_t$ to follow a second-order autoregressive model. This representation
allows for convenient handling of missing data, extrapolation before and after
the period of observed seroprevalence measurements, and maximum likelihood
estimates of $\gamma$ and $\sigma^2_\epsilon$. Details of this methodology and
the computation of the associated uncertainty measurements are deferred to the
Appendix. 



\subsection{Lagged correlation to hospitalizations and time-varying IHRs} 

We use our infection estimates in a lagged correlation analysis with
confirmed COVID-19 hospitalizations. Our primary goal of this analysis is to
find the lag between infection and hospitalization rates that gives the highest
average rank-based correlation across US states. To that end, we consider a wide
range of possible lag values ranging from 1 to 25 days. Zero and negative
lags are not considered because COVID-19 infection onset must precede
hospitalization due to the virus. To remove day of the week effects, both the
infection and hospitalization signals are subject to a 7-day moving
average (center-aligned) before their conversion to rates.

For each considered lag, we calculate the Spearman's correlation between the 
state infection and hospitalization rates for each observed day 
over the March 9, 2020 to February 28, 2022
time period%, and use the epi\_slide function \citep{mcdonald2023epipredict} to 
with a center-aligned rolling window of 61 days for each such computation.
We then calculate the average correlation across all states and times for each lag. 
The lag that leads to the highest average correlation is used to estimate 
the time-varying IHRs for each
state. To compute this for a given day, the number of individuals who are
hospitalized due to COVID-19 on a day are divided by the estimated total number
who were infected on the lagged number of days before.

\subsection{Ablation study for the lagged correlation analysis} 

To better understand the contribution of the intermediate steps to the lagged 
correlation analysis, we carry out a brief ablation study in which we calculate the 
lagged correlation using the following infection estimates: 1. those from the
deconvolution procedure under the assumption that the report date is the 
same as the symptom onset date (i.e., excluding
the reporting delay data); 2. those from the deconvolution procedure when only the
reporting delay data is used (excluding the incubation period data); 3. those from
the deconvolution procedure when utilizing both the incubation period and delay data 
(the deconvolved case estimates); 4. those from applying the leaky model to produce
estimates for both the reported and the unreported cases (the infection estimates).

\section{Results}

This work estimates incident infections for each U.S.\ state over March 9,
2020 to February 28, 2022 and to illustrate the disease burden and viral
transmission dynamics at the level of the state across time. After converting the number
of infections to rates (infections per 100,000 population), we perform a
brief comparison between infection and case estimates within each state to see
to what extent that surges in infections are evident in cases alone
and point out instances where cases largely fail to capture surges in infections.
Then, we look at patterns in infections across the states and postulate what may be 
contributing to these trends based on defining characteristics such as state 
healthcare performance and geographical contiguity. 

\subsection{Infection estimates compared to reported cases}

Naturally, outbreaks in infections precipitate those in cases and are reliably larger in
magnitude (\autoref{fig:state_est_upto_dec1121} and
\autoref{fig:state_est_after_dec1121}).
Hence, our infection estimates indicate
that the pandemic had a differential impact across states earlier and at a larger scale
than is suggested by cases. While the major Ancestral, Delta, and Omicron waves tend to be
visible for most states, there are
clear outbreaks in unreported infections that are not easily detectable from cases alone in the 
falls of 2020 and 2021. For example, take the Alpha wave from mid-2021 in the 
midwestern states of Michigan, North Dakota, South Dakota, and Illinois
or the Delta wave from the fall of 2021 in the east coast states 
of Connecticut, Rhode Island, and Massachusetts. Earlier on in the pandemic, such
discrepancies may be more attributable to failures in the reporting pipeline, while
later on in the pandemic, they more likely due to the rise in asymptomatic infections
across variants \citep{oph2022covid, garrett2022high}. 

Finally, while the January 2022 Omicron wave is evident from the case counts for all states 
(\autoref{fig:state_est_after_dec1121}), our estimates suggest that case counts tend to severely 
underestimate infections during this time for many states. The lowest of all states was in Nevada, 
where about $9.1\%$ (95\% confidence interval: $[4.0, 100.0]$) of the infections were reported 
over January 2022. This was followed by Arizona with $12.2\%$ ($[4.8, 98.4]$), and Minnesota 
with $13.9\%$  ($[8.0, 53.5]$).  More broadly, in $25$ states
less than $50\%$ of infections were reported. Only $11$ states of Florida, New York, Colorado, New Jersey, 
South Dakota, Delaware, Alabama, Arkansas, North Carolina, New Hampshire, 
and Maryland reported at least $70\%$ infections over this time. 

% Could add more observations here
% Ex. In NY over some time, the total number of reported cases is 20% of the population, whereas the estimated infections is about 70%.
% The Alpha, Delta, and Omicron spikes would be good to explore more

\begin{landscape}
\thispagestyle{empty}
\begin{figure}[!tb]
    \centering
    \includegraphics[width=.99\linewidth]{state_niauc_before_dec11.pdf} 
    \caption{Estimates of the number of daily new infections per
     100,000 population for each US state from March 9, 2020 to December 11, 2021
      (dark blue line). The blue shaded regions depict the 50, 80, and 95\% confidence 
      intervals for the estimates, while the teal line represents the 
      number of new daily new deconvolved cases per 100,000, and the dotted 
      orange line represents the 7-day average of the new cases per 100,000 as 
      of the same date.}
    \label{fig:state_est_upto_dec1121}
\fillandplacepagenumber
\end{figure}
\end{landscape}

\begin{landscape}
\thispagestyle{empty}
\begin{figure}[!tb]
    \includegraphics[width=.98\linewidth]{state_niauc_after_dec11.pdf} 
    \caption{Estimates of the number of daily new infections per
     $100,000$ population for each US state from December 11, 2021 to February 28, 2022
      (dark blue line). The blue shaded regions depict the 50, 80, and 95\% confidence 
      intervals for the estimates, while the teal line represents the 
      number of new daily new deconvolved cases per $100,000$, and the dotted 
      orange line represents the 7-day average of the new cases per $100,000$ as 
      of the same date.}
    \label{fig:state_est_after_dec1121}
\fillandplacepagenumber
\end{figure}
\end{landscape}

We perform a lagged correlation analysis where we systematically investigate the
rank-based (i.e., Spearman's) correlation between our infection and confirmed
hospitalization rates per 100,000 population over a broad range of lag values
(\autoref{fig:infect_case_hosp_lag_corr}). Examining the correlation between infections and
hospitalizations shows that the maximum average correlation across states of 0.651 is 
observed at a lag of 14 days. In contrast, we find that the greatest average
rank-based correlations for cases with confirmed hospitalizations is achieved at a lag of
1. That is, we find that case report rates are nearly contemporaneous to
hospitalizations, while infection estimates clearly precede them. 

We undertake an ablation study for the lagged correlation of infections, the results of which 
are shown in \autoref{fig:adj_unadj_sym_hosp_lag_corr}. From this, we can see that the 
infection estimates from each intermediate 
step are all leading indicators of hospitalizations. However, the degree that each such set of estimates
lead hospitalizations depend on its location in the sequence of steps and how close the estimates 
are to infection onset. For example, the deconvolved cases by symptom onset 
tend to precede hospitalizations by about $11$ days, 
while those for the subsequent step indicate that the deconvolved case estimates by infection onset
precede hospitalizations by about $15$ days. When we solely rely on incubation period data 
in the deconvolution (i.e., under the assumption that the report date is the same as the date 
of symptom onset), we can see that the reported infections tend to precede
hospitalizations by about $6$ days. 

In terms of average correlation, the deconvolved case estimates by infection onset provide similar
information indicative to hospitalizations as the deconvolved case estimates by symptom onset
about $4$ days before the latter tends to occur, which highlights a time-based benefit of opting for
infection estimates by the date of infection onset over symptom onset. 

Unsurprisingly, the deconvolved case and infection estimates achieve their maximum correlation
at nearly the same lag. And yet, the average correlation to hospitalizations
tends to be greater for the deconvolved case estimates than for the infection estimates (and
the reported infections by symptom onset). If the goal is to find the part of this process that is most informative to
 hospitalizations, then it is clear that producing the deconvolved case estimates is the more informative step
and these estimates are potentially the most meaningful signal for future hospitalizations.
 This finding could indicate that the the unreported infections tend to be less severe and 
 less likely to lead to hospitalization than those that are reported.
%% Maybe move the second last last sentence to the discussion

As a counterpart to our lagged correlation analysis, we compute the time-varying IHRs 
for each state using the optimal lag for infection and hospitalization rates. We also included
the CHRs that are computed using the optimal lag for cases and hospitalizations for comparison.
(\autoref{fig:IHR_7dav}). 

\begin{figure}[!tb]
\centering
    \includegraphics[width=.8\textwidth]{infect_case_hosp_lag_corr_Nov2.pdf} 
    \caption{Lagged Spearman's correlation between infection and hospitalization
    rates per 100,000 as well as between case and hospitalization rates per
    100,000. The averages shown are for each lag, across US states and days
    over March 9, 2020 to February 28, 2022, and taken over a rolling window of
    61 days. Note that the infections, cases, and hospitalization counts are
    subject to a center-aligned 7-day averaging to remove spurious day of the
    week effects. The dashed lines indicate the lags for which the highest
    average correlation is attained.}
    \label{fig:infect_case_hosp_lag_corr}
\end{figure}

\begin{figure}[!tb]
\centering
    \includegraphics[width=.8\textwidth]{adj_unadj_sym_inc_hosp_lag_corr_Nov2.pdf} 
    \caption{Lagged Spearman's correlation between the infection and
    hospitalization rates per 100,000 averaged for each lag across US states
    and days over March 9, 2020 to February 28, 2022, and taken over a rolling
    window of 61 days. The infection rates are based on the counts for the
    deconvolved case and infection estimates as well as the reported infections
    by symptom onset and when the report is symptom onset. Note that each such
    set of infection counts is subject to a center-aligned 7-day averaging to
    remove spurious day of the week effects. The dashed lines indicate the lags
    for which the highest average correlation is attained.}
    \label{fig:adj_unadj_sym_hosp_lag_corr}
\end{figure}
While the IHRs tend to be less than 0.1 hospitalizations per infection, the CHRs tend to 
present a more amplified version of the CHRs for each state.
This supports our claim that the reported infections are more likely to require hospitalization
than the unreported infections. Both the IHRs and CHRs exhibit similar geospatial and 
temporal trends as are noted for infections. 
Namely, states that are close in proximity (such as Pennsylvania and Virginia) tend to exhibit similar patterns in
the IHRs over time. In addition, there are similar spikes observed across many states during 
waves of infections that are driven by prominent new variants. For example, many
states exhibit a striking spike in hospitalizations in mid-2021, which coincides with the rapid takeover 
of the Delta variant during that time \citep{hodcroft2021covariants}. This finding aligns with 
previous studies that found an increased risk in hospitalizations with Delta in comparison to other
variants \citep{twohig2022hospital, nyberg2022comparative}. Similarly, during the beginning of the
wave driven by the new Omicron variant in early 2022, there tends to be another spike in the IHRs
that rivals that observed during the time of Delta. However, this could be in part due to estimating near to the boundary.
Aside from this, there appears to be a slight waning pattern over time for many states where the
IHR is generally greater earlier than later in the pandemic over which time the virus mutates to variants that 
are generally more infectious, but that pose less of a risk to 
hospitalization \citep{lorenzo2022covid, blauer2022compare}.


%% Are the IHRs greater in states with poorer healthcare system performance?
%States such as _____ that are reputed to have lower healthcare performance leading up to the pandemic tended to 
%result in higher IHR estimates, notably in _____.

\begin{landscape}
\thispagestyle{empty}
\begin{figure}[!tb]
    \centering
   \includegraphics[width=.99\linewidth]{IHR_7dav_Nov5.pdf}
    \caption{Time-varying IHR and CHR estimates for each state from March 9, 2020
    to February 28, 2022, obtained using the corresponding optimal lag from the
    systematic lag analysis. Note that the infection, case, and hospitalization
    counts are subject to a center-aligned 7-day average to remove spurious day
    of the week effects. Also note that the different starting points across
    states are due to the availability of the hospitalization data.}
    \label{fig:IHR_7dav}
\fillandplacepagenumber
\end{figure}
\end{landscape}

\subsection{Disease burden and viral transmission}

\begin{figure}[!tb]
\centering
    \includegraphics[width=.9\linewidth]{state_niauc_est_6states.pdf}
    \caption{Estimates of the number of daily new infections per 100,000 for a
    sample of six US states from March 9, 2020 to February 28, 2022 (dark blue
    line). The blue shaded regions depict the 50, 80, and 95\% confidence
    intervals. The background is shaded to indicate the top variant in
    circulation at the time.}
    \label{fig:six_state_est}
\end{figure}

By reconstructing the time series of COVID-19 infections per $100,000$
population for each US state from March 9, 2020 to February 28, 2022, we observe
rates of infections that vary in intensity and disease burden across space and
time (\autoref{fig:state_est_upto_dec1121}, \autoref{fig:state_est_after_dec1121},
 \autoref{fig:six_state_est}, \autoref{fig:choro_inf_rates}). The
largest observed outbreaks at the beginning of the Omicron era (January
2022) in Nevada, Arizona, and Utah which suggests a similar spread of the virus in
states that are in close geographic proximity. During this time, the state that
has the highest rate of infections per 100,000 on single day is Nevada with
about $3289$ infections per 100,000 on January 9, 2022 (95\% confidence interval:
$[296, 7431]$), followed by Arizona with $2718$ on January 10, 2022 (95\%
confidence interval: $[346, 6748]$), and Utah with $2179$ on January 8, 2022
(95\% confidence interval: $[383, 6955]$).
Interestingly, Maine has a delayed major spike in mid-February 2022 relative to 
the other states (which tend to present their major Omicron spikes in January 2022).
On February 10, 2022, Maine presents the highest estimated infection rate on a day, 
having about $4464$ infections per 100,000 (95\% confidence interval: $[718, 8211]$).

Aside from Omicron, most states present smaller surges in infections
from around November 2020 to January 2021,
the late summer to fall of 2021 and leading up to the major surge in early 2022,
which likely represent well-known waves driven by the Ancestral and Delta
variants. 

During the Delta wave in the summer of 2021, the state that has the
highest rate of infections per 100,000 on single day is Louisiana with about
600 infections per 100,000 on July 30, 2021 (95\% confidence interval:
$[474, 728]$), followed by Georgia with 476 on August 8, 2021 (95\%
confidence interval: $[407, 546]$). These also represent the highest rates of
infections per 100,000 for any state prior to the domination of the Omicron
variants. Similar patterns in the major surges of infections are observed in
nearly all states, though to varying degrees. In general, greater similarities
in the strength and magnitude of outbreaks are observed in the clusters of
states that border each other and in those that have similar state health system
performance (in terms of indicators for the access, cost and the quality of the
state's healthcare as well as the health-related outcomes as noted in
\citet{radley2020}). 
% Maybe note the large confidence intervals for Omicron and much smaller for Delta

\begin{figure}[!tb]
\centering
    \includegraphics[width=.99\textwidth]{choro_inf_rates_Nov5.pdf}
    \caption{Choropleth maps of the state-level estimates of the number of 
    daily new infections per $100,000$ population for various times 
    over March 9, 2020 to February 28, 2022. These maps are generated from the \texttt{usmap} 
    package in R \citep{lorenzo2023usmap}.} 
    \label{fig:choro_inf_rates}
\end{figure}

The period of lowest viral transmission is observed in the spring of 2020. In April 2020,
the states of Vermont and Hawaii achieve the lowest rate of infections 
over the month of 5.67 and 6.72 infections per 100,000, respectively. These are followed by
Montana which achieves a rate of 10.6 infections per 100,000 in May 2020. In the spring of
2020, Montana maintains a rate under 10 infections per 100,000 from the week of April 12, 2020 to 
May 24, 2020. The states that consistently achieve the lowest rates
of infections tend to be those that demonstrate better pre-pandemic healthcare performance such as 
Vermont, New Hampshire, and Hawaii \citep{radley2020}.

From a brief inspection of the geo-contiguous states, we can observe similar patterns in
surges and periods of waning over time, suggesting that states who share similarities in
climate and topography performed similarly to each other. More precisely, we can observe
neighboring states such as New Hampshire and Massachusetts or Washington and Oregon that
present waves that mirror each other in amplitude and timing. 

Interestingly, the two states that are geographically removed from the
contiguous United States, Alaska and Hawaii, tend to perform quite differently
from each other later in the pandemic. Alaska generally presents significantly greater rates of
infections than Hawaii especially during the Omicron era. This suggests that
it is not so much the non-contiguity aspect as it is other distinguishing
factors that lead to lower infection rates.

%\subsection{Sensitivity analysis}
% This section is under construction
%The infection estimates exhibit modest changes under different assumptions about the
%variant-specific incubation periods, the construction of the delay distribution (the
%window size for the considered onset dates), the fraction of new infections over time, and
%the population estimates (see Supplementary Materials Section X). 
% Potentially compare ww a_t ratios over the time period as well (ie. do they fall in the confidence bands for a_t estimates
% from the ss model)?
% \attn {Add a sentence or two about what is meant by modest changes... Did any of these result in noticeably higher or lower (biased) estimates of infections? To what extent (ie. an additional X infections)? For what states?}
% Do these sensitivity analyses + update this link accordingly.

\section{Discussion}

We obtained retrospective estimates of daily incident infections for each US
state for March 9, 2020 to February 28, 2022. While all states present waves that
are associated with the major emerging variant of the time, the clear
variability in the magnitude of our estimates indicate that the intensity and
disease burden are heterogenous across states. Yet, there are similar epidemic
patterns in surges and periods of waning observed in clusters of neighbouring
states. As well, states with lower reputed healthcare performance tend toward
higher infection rates during major surges in infections. For instance, Nevada
and Arizona, which exhibited the highest rates of infections during the major
Omicron outbreak in January 2022, are both firmly located in the bottom quartile
in the assessment for access and affordability from the 2020 Scorecard on State
Health System Performance \citep{radley2020} and in the government estimates of
per capita personal health care spending by state for that year
\citep{centers2020health}. So perhaps the apparent difficulty in containing the
spread during the Omicron wave is at least in part due to the lack of healthcare
support. As noted in the report, both such states have a notably higher
proportion of people who are uninsured and adults without a usual source of
care. For example, the well-documented shortage of healthcare professionals
impacting these states \citep{do2023nevada, gong2019higher}. Indeed, the
proportion of adults without a usual source of care is estimated to be well
above the national average in both states for the pre-pandemic year of 2018
\citep{radley2020}. In addition, a lack of healthcare professionals means a lack
of medical guidance at the level of the individual and a lack of opportunity for
early testing/detection and treatment during the pandemic. This coupled with the typically
milder symptoms of Omicron in comparison to previous variants may have made it
more difficult for individuals to identify that they were infected with the
virus and contributed to the increase in spread in those states. Furthermore, a
lack of health insurance and the prospect of having to pay costly out-of-pocket
medical expenses may have deterred people from seeking testing
\citep{embrett2022barriers} and denied opportunities to get tested when visiting
a medical professional for reasons unrelated to COVID-19. In contrast, Hawaii,
Vermont and New Hampshire, which are reported to be top performers in healthcare
for the domains of access and affordability, prevention and treatment, avoidable
use and cost, healthy lives, and income disparity in \citet{radley2020}, exhibit
some of the lowest rates of infections during the pandemic and were routinely subject to
surges that were lower in intensity.
%That said, as this link to healthcare performance is largely speculative, a more in-depth analysis that takes into account others' assessments of healthcare performance is necessary to explore the plausible connection between state healthcare performance and infection rates during the height of the pandemic. 

Interestingly, the states with the highest rate of infections during the January
2022 Omicron wave, Utah, Arizona, and Nevada, are all in the top quartile for
the percent change in population from 2020 to 2022 based on the annual estimates
of the resident state populations from the US Census Bureau
\citep{uscensus2022annual}, suggesting that there may be a connection between
states that are among the fastest growing and increased infection rates during
that outbreak. Considering the larger time frame of 2010 to 2020 leading up to
the pandemic, the US Census Bureau reports that Utah holds the greatest percent
change in population at about 18.4\%, Nevada is ranked fifth at 15\%, and
Arizona is ranked $9\th$ at 11.9\% \citep{censusbureau2020}. It is reasonable
that the faster growing states place more strain on the healthcare system than
states that grow at a slower pace, limiting access to medical professionals and
to resources for diagnosing and testing. To elucidate this possible connection,
further exploration into the impact of state population growth on infection
waves and healthcare system performance during the pandemic is warranted.

It is reasonable that Montana experienced some of the lowest infection
rates in the spring of 2020 due having a lower population and population density 
as well as from having a larger area relative to other states. This combination of favourable
conditions likely contributed to lower infection rates earlier on in the pandemic.
Similarly, Hawaii and Alaska were both states that maintained rates of under 
10 infections per 100,000 for at least three weeks in the spring of 2020. It is 
plausible their geographic isolation in addition to the previous factors discussed for
Montana contributed to such lower infections near to the beginning of the pandemic.
Research by \citet{provenzano2020urban} and \citet{carozzi2022urban} on urban density and COVID-19 
in US counties found that the geographic connectivity and social connectedness of denser areas 
likely impacted the timing of outbreaks (so that denser locations were more likely to 
have outbreaks earlier on), but by the end of 2020, density had little to no impact on time-adjusted 
 COVID-19 cases.
 However, investigation beyond the first year of the pandemic is warranted.
 As well, this study only looks relation between density and cases, not infections 
 Since the relationship between density and infections has remained relatively unexplored, 
 further research should be done to elucidate such connections and how they change over time.

Our infection estimates suggest that the pandemic has an impact
in states earlier and at a larger scale than is indicated by cases. Since
case reporting is not consistent across time and states, case counts
underestimate the true number of infections and, hence, the impact of the
pandemic \citep{cdc2022estimated, simon2022inconsistent}. For example, some
states report the number of individuals tested rather than the numbers of tests
performed \citep{schechtman2020counting, chitwood2022reconstructing}.

We observe outbreaks in infections that are difficult to detect from cases
alone such as the Delta wave that prevailed from July to November 2021 in Connecticut, 
Rhode Island, Massachusetts, and New Jersey. This suggests that cases paint an incomplete 
picture of the pandemic, especially when outbreaks are largely driven by 
unreported infections. Furthermore, since case report dates follow symptom and infection
onset, cases are a fundamentally flawed indicator of disease burden because they
have a built-in temporal bias. This is in addition to other biases from
differences in reporting across states (such as temporary bottlenecks due
influxes of data or more persistent processing issues that increase the average
time from case detection to report \citep{wash2020dash, dunkel2020covid19}. So
while reported cases provide an indication of the trajectory of the pandemic, it
is a delayed and incomplete version. Estimating the new number of infections by
symptom or infection onset date would more closely align with the definition of
incidence as we know it \citep{jahja2022real}.

From the correlation analysis between daily infection estimates and
hospitalizations, a lag of 14 days gives the maximum average correlation 
across states. This is in agreement with the early estimates of the average time from
infection to hospitalization of 9.7 days (95\% CI: $[5.4, 17.0]$) for
cases reported in January, 2020 in Wuhan, China as well as with estimates from
across the pandemic in the UK that ranged from an average of 8.0 to 9.7
days, more precisely, 8.0 days (95\% interval: $[2.7, 18.5]$) for the first
wave to 9.7 days (95\% interval: $[4.1, 19.6]$) for the second wave,
\citep{ward2021understanding}. However, we should note the first study is based
on a small sample size for outbreak cases reported well before our study start
date. As well, both sets of estimates depend upon the healthcare system and the
population structure, amongst other things \citep{ward2021understanding}.
Nevertheless, their relative agreement with our estimate of 14 days for the US
states lends some credence to of our results. 

While we computed IHRs for all states, it is important to note that the IHR is
 also likely to vary within states and depend on additional variables such as
 age and the presence of major comorbidities \citep{russell2023comorbidities}.
 Therefore, it would be beneficial to account for such variables in the IHR
 calculations by, for example, stratifying infections and hospitalizations by
 age to produce age-specific estimates of the IHRs for each state (similar to
 \citealp{fox2023disproportionate} though with the additional element of being
 time-varying). We strongly believe this would be a worthwhile direction to
 pursue in future work should the necessary information be available. 

The remainder of our discussion consists of an in-depth look into the advantages and
limitations of our approach and of other comparable approaches, followed by a
high level summary of our work and its major contributions. 

Our approach offers a number of advantages.
% The development 
% of a way of modelling immunity and space-time-specific reporting ratios based on 
% seroprevalence data.
% Similar phrase on line 1268, though we may want to emphasize that point up here.
To the best of our knowledge, no other modelling approach has been used to 
reconstruct the infection time series
for every state over as much of the COVID-19 pandemic as in this study.
Furthermore, we aim to incorporate as much state-specific information as
possible when deriving our estimates. For instance, using variant circulation
and line list data, we are able to construct incubation and delay distributions
that are unique for each state. By using time-varying and state-specific
seroprevalence data, we are able to allow the reporting ratio to vary over both
time and state, which is an advantage over such ratios that are non-time varying
but state-specific and those that are time-varying but the same for all states
\citep{unwin2020state, uga2020covid19}. 
Existing approaches that use the delay distribution to generate infection
estimates often only construct one delay distribution that is used for all
states \citep{chitwood2022reconstructing, jahja2022real}. That is, they operate
under the assumption of geographic invariance, where it is assumed that all
states have the same patterns of delay from infection onset to case report,
which is unlikely to be true due to differences in reporting pipelines, pandemic
response, and variants in circulation, amongst other things. 

Another major limitation is that these models do not to account for
reinfections. Now, it may be contended that reinfections do not account for a
substantial fraction of the infections until later in the pandemic, so they are
not absolutely necessary to include in the earlier stages of the pandemic.
Still, at no stage did infection with the COVID-19 virus confer lifelong
immunity. Rather immunity is transient and wanes over time. And we believe it is
important to account for such defining characteristics of the virus when
tracking infections over time. Therefore, we account for reinfections and waning
of detectable immunity in our custom leaky immunity model. However, we 
acknowledge that the extent to which each of these are accounted for could be
improved upon in future work. 

Since the waning of detectable immunity is likely to be variant-dependent
\citep{pooley2023durability}, it follows that the leaky parameter may be better
posed as a mixture of parameters for different variants with weights determined
by the proportion of the variants circulating at the time in the state. Related
to this is the issue of how newer variants may escape detection
\citep{nih2022assessing, fda2023sars}. While in a retrospective analysis where
finalized data is used this is less likely to be an issue, this could very well
pose a problem for real-time estimates of infections.

As for reinfections, it would be ideal to have confirmed rates
over time for each US state. However, we are unable to find such data available
over the entire time period considered for even one state. So we have turned to
suspected reinfection data over time for Clark County, USA, as that surveillance
is amongst the most detailed that we have found for the United States.
Nevertheless, using such localized data raises questions of representativeness
and the applicability of such estimates to Nevada and all other states.
Furthermore, this data has no information available beyond suspected third
infections, which imposes an irremediable bias. However, based on the third
infection data available there, we expect that the probability of being
reinfected more than three times is likely very low for time frame considered
and so the omission of these would impact our infection estimates to a small
extent. 

The vast majority of issues we encountered when trying to reconstruct the
infection time series for each state are due to an absence or a lack of data.
Such is the primary issue we had with the restricted line list. In comparison to
the number of JHU cases (which we are treating as a gold standard) for the same
release date, we noted there are about $10$ million cases that are unaccounted
for in the CDC line list. Moreover, the missingness does not appear to be random
and uniformly distributed across states. Rather it is unequally distributed,
suggesting that the dataset is likely biased. However, more information on the
cases that are missing versus present would be required to determine the extent
the missing cases led to a nonrepresentative, and therefore, biased sample, and
could be a topic of further study.

Seroprevalence data also runs the risk of being nonrepresentative of the
intended population \citep{bajema2021estimated}. For example, in the blood donor
dataset some states have region specific-estimates, which clearly do not stand
for the entire state. Another source of systematic variation is in the
characteristics of the individuals who opt for blood tests versus those who do
not. For instance, there may be a healthy user bias, in which a number of those
who opt for blood tests are generally more inclined to partake in proactive
healthy behaviors (such as checking on basic health markers by taking an annual
blood test) than those who do not \citep{parsley2018blood}. Alternatively, a
number of individuals may be recommended for blood tests by their doctors due to
signs of ill-health (ex. mineral deficiencies or underlying medical
conditions). The extent that each such bias persists depends on the purpose of
the blood test and whether it was used as a proactive or reactive medical tool.
Since such information is unavailable to us, all we can conclude is that
participant-driven sources of bias impact the seroprevalence samples to an
undetermined extent. There are additional concerns about the performance of
antibody testing for individuals with mild or asymptomatic disease as well as
the loss of immunity over time \citep{kaku2021performance, seow2020longitudinal,
ibarrondo2020rapid}.

In this work, we do not attempt to directly address infection underascertainment
due to the increase in asymptomatic infections across variants
\citep{pho2023covid19}. We simply note that this would likely pose a greater
problem later in the pandemic, particularly during the Omicron era
\citep{fan2022sars}. We hope that such infections would be largely represented
by the seroprevalence and reinfection estimates, but there is undoubtedly
increasing reliance on such estimates to be able to do this over time (owing to
the simultaneous decline in the reporting cadence and the apparent rise in
asymptomatic infections over time) \citep{oph2022covid, garrett2022high,
blauer2022reduce, ren2021asymptomatic}. Consequently, there is an increasing
uncertainty over time that is not expressed by the model or the estimates.   

Due to such concerns with the seroprevalence data, one further area of research
is on investigating the utility of various sources to estimate the incidence of
infections. Intuitively, one might expect that leveraging data from multiple
sources would likely lead to more accurate and stable estimates than those from
using one source. Wastewater surveillance data is one
promising source that may be complementary to seroprevalence data, especially
when testing is low \citep{mcmanus2023predicting}. However, there has been
limited success in predicting incidence using such data and the extent that
wastewater concentration data is a useful in estimating COVID-19 incidence is
unclear owing to problems with viral occurrence and detectability in wastewater
that render detection inconsistent across locations (ex. due to temperature,
per-capita water use, and in-sewer travel time) \citep{mcmanus2023predicting,
hart2020computational, li2023correlation}. Sentinel surveillance streams for
influenza-like illness or acute respiratory infection may provide decent proxies
for COVID-19 incidence, especially when testing for mild cases of COVID-19 is
diminishing or has ceased completely. Finally, alternative surveillance streams
(potentially outside of public health) such as those from surveys, helplines, or
medical records could potentially be integrated if they provide at least a rough
indication of the disease intensity over time \citep{ecdc2020strategies}.

Overall, we adopt a relatively simple deconvolution-based approach and devote
much of our efforts to tailoring our approach to the available data. A
major result of this was the development of a way of to model immunity and
space-time-specific reporting ratios based on seroprevalence data.
In a way, our approach is built for the data rather than trying to force the data to fit
to an existing approach. However, our model is only as good as the quantity and 
the quality of the data provided to it. In our case, the lack of data is both a barrier to entry
and a continual roadblock. The assumptions we are required to make as a
consequence of this clearly limit the generalizability and call into question
the reliability of the results. So while we highlight some interesting trends and
numerical findings, these results are not definitive, but rather exploratory and
intended to stimulate discussion on the challenging task of estimating
infections. Despite these limitations, we are encouraged by the ability to use
routine data to produce sensible estimates of infections in the United States and the
plausibility of the apparent geospatial and temporal trends. 
 
Our approach is predicated upon having case, line list, viral circulation, and
seroprevalence data for each state, all of which are readily available (or
available upon request in the case of restricted line list data). As a result of
this, we are able to demonstrate the feasibility of estimating COVID-19
infections at the state level by using standard sources of data. 

Our framework is quite versatile as it lends itself to more localized, county or
community level estimates, or globalized, country-specific estimates.
Fundamentally, to produce estimates of infections for different geographic
regions, one would simply need to input the required data and re-run the
pipeline. In this way, one could readily adapt our approach to generate
estimates for the provinces in Canada or the regions in England.

Well-informed, localized estimates of COVID-19 infections over time can help us
to have a more clear and comprehensive understanding of the course of the
pandemic. Such estimates contribute important information on the timing and
magnitude of disease burden for each location and they highlight trends that may
not be visible from case data alone. Therefore, our infection estimates provide
key information for the ongoing debate on the true size and impact of the
pandemic.



\subsection*{Acknowledgements}

% Required Gisaid acknowledgement
We gratefully acknowledge all data contributors, i.e., the Authors and their
Originating laboratories responsible for obtaining the specimens, and their
Submitting laboratories for generating the genetic sequence and metadata and
sharing via the GISAID Initiative \citep{elbe2017data}, on which this research
is based.

%\bibliographystyle{naturemag} %% Change back to numeric references in line with requirements for Nature Communications articles (see pg 3 fro here: https://www.nature.com/documents/ncomms-formatting-instructions.pdf)
\bibliographystyle{rss}

\newpage
\bibliography{bibliography.bib}

\newpage
% Eventually make the supplement a separate document with its own title page 
\beginsupplement
\title{\supptitlefont Online Supplement}
\maketitle

\section{Additional information about estimation methodology}

\subsection{State space representation of the leaky immunity model}

To estimate the leaky immunity model, \autoref{eq:leakypr}, we express it as a 
Gaussian state space model (as in \citealp{durbin2012time, helske2017kfas}).

In general, for $t = 1, \dots, n$, we let $\alpha_t$ be the $m \times 1$ vector of latent
state processes at time $t$ and $y_t$ be the $p \times 1$ vector of observations
at time $t$. Under the assumption that $\eta$ is a $k \times 1$ vector, the
form of the linear Gaussian state space model is 
\begin{align}
y_t &= Z\alpha_t + \epsilon_t  \text{     (observation equation)} \label{eq:ss1}\\
\alpha_{t+1} &= T_t\alpha_t + R_t\eta_t  \text{     (state equation) \label{eq:ss2}}
\end{align}
where $\epsilon_t \sim N(0, H_t)$, $\eta_t \sim N(0, Q_t)$, and $\alpha_1 \sim
N(a_1, P_1)$ independently of each other \citep{helske2017kfas}. For notational
compactness, we let $\alpha = \left ( \alpha_1^\top, \dots, \alpha_n^\top \right )$
and $y = \left ( y_1^\top, \dots, y_n^\top \right )$.

The observation equation can be viewed as a linear regression model with the
time-varying coefficient $\alpha_t$, while the second equation is a first-order
autoregressive model, which is Markovian in nature \citep{durbin2012time}. 

The main idea of the two equations is that the system evolves over time
according to $\alpha_t$ (as in the second equation), but since those states are
not directly observed, we turn to the observations $y_t$ and use their
relationship with $\alpha_t$ (as in the first equation) to drive the system
forward \citep{durbin2012time}. So the objective of state space modeling is to
obtain the latent states $\alpha$ based on the observations $y$ and this is
achieved through Kalman filtering and smoothing. 

Kalman filtering gives the one-step-ahead predictions and prediction errors:
\begin{align*}
a_{t+1} &= \E[\alpha_{t+1}\given y_t, \dots, y_1] \\
v_t &= y_t - Za_t
\end{align*} with covariance, 
\begin{align*}
P_{t+1} &= \Var(\alpha_{t+1} \given y_t, \dots, y_1) \\
\Var(v_t) &= ZP_tZ^\top + H_t.
\end{align*}
Then, the state smoothing equations are run back in time to give
\begin{align}
\hat{a}_t &= \E[\alpha_{t}\given y_n, \dots, y_1] \label{eq:hatat}\\
V_t &= \Var(\alpha_{t}\given y_n, \dots, y_1). \label{eq:Vt}
\end{align}
The filtering and smoothing steps are based on recursions that are described in
Appendix A of \citet{helske2017kfas} as we use the R package KFAS to estimate
our model.

For our situation, the Kalman filter and smoothing approach offers a number of
advantages over the penalized regression approach. Perhaps most notably,
 the parameters are estimated all at once (so cross validating for model
parameter tuning is not necessary). Another major benefit is that it can handle 
unevenly spaced time series (refer to \citealp{durbin2012time} for further details).

To help manage the sparseness in the seroprevalence data, we convert our data 
to weekly by summing the reported infections and shifting the observed 
seroprevalence measurements to the nearest Monday. If there are multiple 
measurements in a week from a source, then the average of those 
measurements is used (and similarly for the weights). We denote these changes by 
changing the time-based subscript from $t$ to $m$ where $m$ indicates the Monday 
relative to our March 9, 2020 start date and $\Delta R_m$ is the change in weekly 
reported infections.

The leaky immunity model on weekly data can be written in state space form by 
defining the corresponding components in Equations \ref{eq:ss1} and 
\ref{eq:ss2} as follows:

% Probably move the below specification to the appendix

\begin{alignat*}{3}
R &= \begin{bmatrix}
1 & 0  \\ 
0 & 1 \\ 
0 & 0 \\ 
0 & 0 
\end{bmatrix} &\qquad 
Z &= \begin{bmatrix}
1 & 0 & 0 & 0 \\ 
0 & 1 & 0 & 0 
\end{bmatrix} &\qquad 
H_m &= \begin{bmatrix} %%
w_{m,c}\sigma^2_o & 0 \\ 
0 & w_{m,b}\sigma^2_o
\end{bmatrix} \\
\alpha_m &= \begin{bmatrix}
s_{m}\\
a_m\\ 
a_{m-1}\\ 
a_{m-2}
\end{bmatrix} & 
T_m &= \begin{bmatrix}
 \gamma & \Delta R_mn_m & 0 & 0\\ 
 0 & 3 & -3 & 1 \\ 
 0 & 1 & 0 & 0\\ 
 0 & 0 & 1 & 0
\end{bmatrix}  & 
Q &= \begin{bmatrix} 
\sigma^2_s & 0  \\ 
0 & \sigma^2_a
\end{bmatrix} \\
a_1 &= \begin{bmatrix}
\tilde{s}_{1}\\ 
\tilde{a}_1\\ 
\tilde{a}_1 \\
\tilde{a}_1
\end{bmatrix} & 
P_{1} &= \begin{bmatrix}
\sigma^2_{\tilde{s}_{1}} & 0 & 0 & 0 \\ 
0 & \sigma^2_{\tilde{a}_1} & 0 & 0\\ 
0 & 0 & \sigma^2_{\tilde{a}_1} & 0 \\ 
0 & 0 & 0 & \sigma^2_{\tilde{a}_1}
\end{bmatrix} 
\end{alignat*}
where $\sigma^2_o$ is the variance of observations,
$\sigma^2_s$ is the variance of the seroprevalence estimates (in terms of 
smoothness), and $\sigma^2_a$ is the trend variance. Since we expect the 
inverse ratios to be more variable than the seroprevalence estimates, 
we enforce that the estimate of $\sigma^2_a$ is a multiple of 
$\sigma^2_s$. Letting the subscripts $b$ and $c$ denote
the blood donor and commercial datasets, $w_{m,c}$ and $w_{m,b}$ are the
time-varying inverse variance weights computed from the commercial and blood
donor datasets, respectively. 

For each source, we compute the weights for the observed seroprevalence
estimates using the standard formula for the standard error of a proportion.
These weights are then re-scaled so they sum to the number of observed
seroprevalence measurements for the source. All days that are unobserved (i.e.,
lack seroprevalence measurements) are given weights of one. Finally, the ratio
of the average observed weights for the sources is used as a multiplier to scale
all of the weights for one source. For example, if the average weight of the
commercial source is double the average weight of the blood donor source (for an
arbitrary state), then we scale all of the weights in the commercial source
(including the ones) by two. The main purpose of this step is to ensure that
the source with a greater sample size contributes more weight in the model on
average. % Last sentence - Or is it simply that the source with larger
%weights on average will have more weight? 

The prior distribution for $\alpha_1$ is estimated using both data-driven constraints 
and externally sourced information. To obtain the initial value of the seroprevalence 
component, $\tilde{s}_{1}$, we extract the first observed seroprevalence measurement from 
each source, round down to two decimal places, and take the average to be the
estimated initial value $\tilde{s}_{1}$. The corresponding initial
variance estimate, $\sigma^2_{\tilde{s}_{1}}$, is taken to be the mean of the standard
errors of the two seroprevalence estimates. For all of the initial values of the trend 
components, we use the inverse of the ascertainment ratio estimate as of June 1, 2020 
for each state from Table 1 in \citet{unwin2020state} and denote this by $\tilde{a}_1$. The
initial variance estimate of $\sigma^2_{\tilde{a}_1}$ is based on the variance implied 
by the given inverse ascertainment ratio distribution.
% standard deviation implied by the interval in that table.  
% Update this last sentence if end up going with the standard deviation implied by the interval in Table 1
% instead of the variance implied by u_m ~ Beta(12,5) from the unwin2020state paper

The initial $\sigma^2_o$ is taken to be the average of the estimated variances
from the linear models for the sources where the observed seroprevalence
measurements are regressed on the enumerated dates. The initial value of 
the multiplier is set to be $100$ for all states. The $\sigma^2_s$ and $\gamma$ 
values are fixed and obtained by averaging the final values for all states on the real line. 

Following the maximum likelihood estimation of the two non-fixed parameters
we use the Kalman filtering and smoothing to obtain the
smoothed estimates of the weekly inverse reporting ratios and
their covariance matrices as shown in Equations \ref{eq:hatat} and \ref{eq:Vt}.
Forwards and backwards extrapolation is then used to estimate the ratios and covariance
outside of the observed seroprevalence range \citep{durbin2012time}, followed by linear 
interpolation to fill-in estimates for each day in our considered time period. 
After we obtain one vector of inverse reporting ratios for each state in this
way, we take each inverse reporting ratio and multiply it by the corresponding
deconvolved case estimate (that has undergone linear interpolation to correct
instances of $0$ reported infections) to obtain an estimate of new infections.
We are able to convert these numbers of infections to
infections per $100,000$ population by simple re-scaling (enabled by the fact
that normality is preserved under linear transformations).

The $50$, $80$, and $95\%$ confidence intervals are constructed by taking a
Bayesian view of the leaky immunity model (refer to the Online
Supplement~\ref{supp:bayesleaky} for the Bayesian specification of the model). 
That is, for each time, $t$, we obtain an estimate of the
posterior variance of $a_t$, apply the deconvolved case estimate as a constant
multiplier, and then use resulting variance to build a normal confidence
interval about the infection estimate. We additionally enforce that the lower
bound must be at least the deconvolved case estimate for the time under consideration.


\subsection{Bayesian specification of the leaky immunity
model}\label{supp:bayesleaky} 
In brief, the leaky immunity model where we let
$\beta = \left \{  \gamma, a_1,\dots, a_t \right \}$ and $X$ be the design
matrix, corresponds to a Bayesian model with prior 
\begin{align*}
    \beta \sim N \left( 0,  \frac{\sigma^2 }{ \lambda} \left( A^TD^TDA 
    \right)^{-1}  \right)
\end{align*} and likelihood 
\begin{align*}
    s|X,\beta \sim N \left( X\beta, \sigma^2W^{-1} \right),
\end{align*} where $A$ is indicator matrix save for the first column of $0$s 
(corresponding to $\gamma$), $D$ represents the discrete derivative matrix of 
order $3$, and $W$ is the inverse variance weights matrix. Then, the posterior 
on $a_t$ is normally distributed with mean 
\begin{align*}
    \left ( X^TWX + \lambda A^TD^TDA \right )^{-1}X^TWs
\end{align*} 
and variance 
\begin{align*}
    \sigma^2 (X^TWX + \lambda A^TD^TDA)^{-1}.
\end{align*}

\end{document}
